\hypertarget{beschaving}{%
\chapter{Beschaving}\label{beschaving}}

\vspace{-1em}
\begin{blockquotebox}
    De essentiële inzichten die de basis vormen voor onze samenwerking, maatschappij en beschaving\index{beschaving}, en die de mens van zijn dierlijke staat naar een wezen van beschaving hebben geëvolueerd, zijn de inzichten die bewijzen dat werken binnen een systeem van arbeidsdeling\index{arbeidsdeling} efficiënter is dan werk in isolatie, en dat het menselijk intellect in staat is deze realiteit te begrijpen. Zonder deze inzichten zouden mensen voortdurend als vijanden tegenover elkaar hebben gestaan, onverbiddelijke concurrenten in de strijd om een deel van de beperkte voorraad voedsel die de natuur te bieden heeft. Men zou genoodzaakt zijn geweest om iedereen als vijand te zien; de drang om in eigen behoeften te voorzien zou onvermijdelijk tot onophoudelijke conflicten met anderen hebben geleid. In zulke omstandigheden zou sympathie nooit hebben kunnen bloeien.\footnotemark
    \par\raggedleft--- Ludwig von Mises\index{Ludwig von Mises}
\end{blockquotebox}
\footautocite{213}

\lettrine{I}n dit boek heb ik getracht een overzicht te bieden van de economie, gezien door de bril van menselijk handelen\index{menselijk handelen}. Ik richt me vooral op hoe mensen acties ondernemen om in hun economische behoeften te voorzien, en hoe dit samenhangt met het vergroten van zowel de kwantiteit als de subjectieve kwaliteit van hun leven op aarde. Mensen zijn dankzij hun verstand in staat de voordelen van hun acties te herkennen en te evalueren. Dit verstand stelt individuen in staat hun leven zo in te richten dat ze acties ondernemen welke hen helpen hun persoonlijke doelen te realiseren en zich af te keren van activiteiten die daar niet aan bijdragen. Naarmate de tijd vordert, zullen handelingen, karaktertrekken en gedragspatronen die bijdragen aan economische vooruitgang in populariteit toenemen, omdat ze voordelig zijn voor diegenen die ze adopteren. Sociale organisatievormen die het mogelijk maken dat vreemden op een vreedzame, productieve en vrijwillige wijze met elkaar omgaan, zullen hun leden in staat stellen hun welzijn aanzienlijk te verbeteren door deelname aan een uitgebreidere en meer geavanceerde arbeidsverdeling\index{arbeidsdeling}. Beschaving kan worden opgevat als de uitgebreide sociale orde die voortkomt uit het gebruik van ons verstand, het verlagen van onze tijdsvoorkeur\index{tijdsvoorkeur} en de samenwerking om het leven in de loop der tijd te verbeteren.

Beschaving kent veel definities.\autocite{214} Deze variëren afhankelijk van de beschaving\index{beschaving} zelf en van tijdperk tot tijdperk, maar de fundamentele werkelijkheid die ten grondslag ligt aan alle opvattingen van beschaving\index{beschaving} is de verbetering van materiële omstandigheden. Hoewel de verbetering van materiële omstandigheden misschien niet het belangrijkste aspect van beschaving\index{beschaving} is, maakt het wel alle andere aspecten mogelijk, waardoor de menselijke samenleving kan genieten van hoge niveaus van productiviteit en een lange levensverwachting. Materiële omstandigheden vormen misschien niet het uiteindelijke doel van beschaving\index{beschaving}, maar zijn wel een onmisbaar middel daartoe. Beschaving biedt ons meer dan enkel materiële voordelen; het verschaft een ongekende manier om onze overlevingskansen en levenskwaliteit te verhogen. De verschuiving van barbaarsheid naar een geciviliseerde samenleving was geen toeval; daarachter lagen krachtige economische motieven. Door te kiezen voor een relatief vreedzamere sociale orde, kunnen mensen zich effectiever beschermen tegen natuurlijke bedreigingen en roofdieren. De reis van de vroegste menselijke gemeenschappen naar de hedendaagse, uiterst gespecialiseerde en technologisch geavanceerde wereldwijde economie was lang en complex, waarbij elke stap voorwaarts werd gedreven door de praktische economische waarde die het toevoegde.

Beschaving\index{beschaving} is diepgaand verweven met economische\index{economisch} methoden die de waarde en kwaliteit van ons leven op aarde verhogen. Deze processen vereisen een afname van de tijdsvoorkeur\index{tijdsvoorkeur}, kapitaalaccumulatie en arbeidsdeling\index{arbeidsdeling}, die op hun beurt afhankelijk zijn van vreedzame sociale samenwerking en menselijke creativiteit. Deze creativiteit stelt het menselijk verstand in staat om welke uitdagingen dan ook aan te gaan en streeft naar de best mogelijke uitkomsten. Het zijn precies deze drie processen die menselijke arbeid onderscheiden van dierlijke arbeid, waardoor we in staat zijn om beschaving\index{beschaving} te bouwen en onze meesterschap over onze omgeving en situaties steeds verder te ontwikkelen.

Het cruciale startpunt van menselijke beschaving\index{beschaving} is de daling van de tijdsvoorkeur\index{tijdsvoorkeur}. Deze verschuiving in het menselijk denken en haar gedrag stelt ons in staat om onze dierlijke instincten te bedwingen, en in plaats daarvan te vertrouwen op ons verstand. Het verlagen van de tijdsvoorkeur\index{tijdsvoorkeur} en het ontwikkelen van het vermogen om voldoening uit te stellen, is het startpunt voor spaargeld, wat zorgt voor kapitaalopbouw en een stijging van de productiviteit en levensstandaard. Naast sparen maakt een lage tijdsvoorkeur\index{lage tijdsvoorkeur} mensen waarschijnlijk beschaafder in hun gedrag. Ze denken sneller na over de gevolgen van hun handelingen en zijn zich dus meer bewust van de gedragspatronen die bevorderlijk zijn voor de groei van de arbeidsdeling\index{arbeidsdeling}, wat op zijn beurt een enorm krachtige manier is om het menselijk welzijn te vergroten. Zonder de arbeidsdeling\index{arbeidsdeling} is de mens alleen, overgeleverd aan de natuur. Dankzij arbeidsdeling\index{arbeidsdeling} neemt zijn productiviteit toe en kan hij meedoen aan de beschaafde maatschappij. Maar om dat te doen, moet hij eerst in staat zijn om mee te doen aan de arbeidsdeling\index{arbeidsdeling}, het economische fenomeen dat mensen samenbindt in de beschaving\index{beschaving} en hen onderling afhankelijk maakt, terwijl men op elkaar moet vertrouwen. Naarmate een persoon met steeds meer mensen in contact komt, wordt het noodzakelijker om sociale instituties en normen voor duidelijke en betrouwbare manieren te ontwikkelen voor de omgang tussen vreemden: dit zijn de beschaafde gewoontes en zeden. Menselijke instituties, cultuur, gewoonten en tradities draaien om het bevorderen van het menselijk gedrag passend bij een uitgebreide sociale orde. De meest doorslaggevende leerstelling waarop sociale samenwerking rust, is het respect voor eigendom. Veel van de sociale orde komt voort uit de noodzaak om mensen het beeld bij te brengen dat een beschaafde samenleving alleen mogelijk is door respect voor eigendom en het aannemen van beschaafde manieren.

De samenleving wordt gedefinieerd door de fenomenen die leiden tot een daling van de tijdsvoorkeur\index{tijdsvoorkeur}, aangezien deze de groei van de samenleving mogelijk maken, zowel in de vorm van kapitaalaccumulatie alsmede in een vreedzame samenwerking. De beschaafde samenleving bestaat in de mate waarin mensen hun tijdsvoorkeur\index{tijdsvoorkeur} verlagen, sparen, vreedzaam deelnemen aan de arbeidsdeling\index{arbeidsdeling} en hun menselijk verstand gebruiken. Hoe meer de arbeidsdeling\index{arbeidsdeling} groeit, hoe beschaafder we moeten worden omdat we met meer mensen omgaan. Beschaving komt naar voren als de manifestatie van een lagere tijdsvoorkeur\index{tijdsvoorkeur}. Hoe lager de tijdsvoorkeur\index{tijdsvoorkeur}, hoe beschaafder we worden.

In de mate waarin een sociale instelling overleeft en floreert, kan zij dit alleen bewerkstelligen door bij te dragen aan de menselijke beschaving\index{beschaving}. Sociale instellingen bereiken dit door hun leden voordelen te bieden die kenmerkend zijn voor beschaving: een lagere tijdsvoorkeur\index{tijdsvoorkeur}, verbeterde deelname aan arbeidsdeling\index{arbeidsdeling} en een verhoogde productiviteit. Het gezin speelt een cruciale rol in de maatschappelijke ontwikkeling. Het bevordert het vermogen om waardering te hebben voor wat er na ons eigen leven komt. De bezorgdheid over wat er met onze kinderen gebeurt na ons overlijden, verlaagt onze tijdsvoorkeur\index{tijdsvoorkeur}. Door zorg te dragen voor en ons sterk te identificeren met onze nakomelingen, rekken mensen de periode uit waarin de gevolgen van hun handelingen belangrijk zijn. Zonder de opoffering van direct plezier door vele generaties heen, zou de wereld van vandaag aanzienlijk minder kapitaal\index{kapitaal} hebben vergaard en zouden we in een veel primitievere staat verkeren. De zorg voor toekomstige generaties is van cruciaal belang voor het onderhoud van een beschaafde samenleving, en het krijgen van kinderen is een zeer effectieve manier om de tijdsvoorkeur\index{tijdsvoorkeur} te verlagen.

Het menselijke verlangen om onze kinderen een beter leven te bieden, is vaak de drijvende kracht achter onze betrokkenheid bij de samenleving en beschaving\index{beschaving}. Zonder de zorg voor kinderen en de toekomstige wereld, vermindert onze motivatie om rekening te houden met de gevolgen van ons handelen na onze dood. Het menselijk verstand onderscheidt ons van alle andere dieren door het vermogen om een dergelijke diepe band met onze nakomelingen te vormen. Als productieve individuen die deelnemen aan arbeidsdeling\index{arbeidsdeling}, kapitaal\index{kapitaal} vergaren en toekomstwaardering tonen, zijn we in staat om onze kinderen een beter leven te bieden. Veel aspecten die ons menselijk maken en onze ervaringen doorheen de geschiedenis zijn gecentreerd rond het verbeteren van het leven van onze nakomelingen. Dit verlangen is een krachtige motor van beschaving\index{beschaving} en een nuttig instrument voor het individuele leven, aangezien het mensen aanzet tot samenwerking en doordacht gedrag met een lage tijdsvoorkeur\index{lage tijdsvoorkeur}. Wanneer deze inzet generaties lang wordt volgehouden, accumuleert het tot een onschatbare erfenis waarmee elke persoon wordt geboren: de menselijke beschaving\index{beschaving}. Talen, religies, tradities, technologieën, ideeën, fysieke infrastructuur, en prachtige bouwwerken zijn allemaal nalatenschappen van voorouders die hun tijdsvoorkeur\index{tijdsvoorkeur} verlaagden, arbeid verdeelden en samenwerkten om beschaving\index{beschaving} op te bouwen. Het is niet alleen onze intelligentie, maar vooral ons vermogen tot samenwerking, het bouwen van beschaving\index{beschaving}, en het accumuleren van kapitaal\index{kapitaal}, zowel fysiek als in de vorm van ideeën, dat ons in staat stelt de natuur te temmen, veilig te leven, en gewelddadige dieren of mensen te overwinnen.

Beschaving kan gezien worden als de meest doeltreffende manier om de waarde van ons leven op aarde duurzaam te vergroten en te verrijken, een proces dat door elke generatie wordt voortgezet. Het is een langdurig traject – net zo oud als de mensheid zelf – dat zich richt op het vergaren van kennis en kapitaal\index{kapitaal} en het verhogen van de levenskwaliteit. Ieder beschaafd mens wijdt zijn of haar leven aan hard werken met het doel het eigen leven en, bij het stichten van een gezin, het leven van het nageslacht te verbeteren. De vooruitgang van de beschaving\index{beschaving} kan worden gezien als synoniem met duurzame economische groei op de lange termijn. Dit komt niet alleen omdat het leidt tot een hogere levensstandaard, maar ook omdat het bereikt wordt door toename van vreedzame interacties tussen steeds meer mensen, een lagere tijdsvoorkeur\index{tijdsvoorkeur} en door innovatie. Het voortduren van het beschavingsproces wordt gekenmerkt door opeenvolgende generaties die een beter leven leiden dan hun voorouders. Dit proces stokt echter wanneer opeenvolgende generaties minder goed af zijn dan hun voorgangers.

\hypertarget{de-prijs-voor-beschaving}{%
\section{De prijs voor beschaving}\label{de-prijs-voor-beschaving}}

De vruchten van de beschaving\index{beschaving} zijn onweerstaanbaar, en eenmaal geproefd, lijkt bijna iedereen er voor het leven aan verslaafd. Slechts een handjevol mensen heeft bewust de beschaving\index{beschaving} de rug toegekeerd om een solitair leven in de natuur te leiden, maar voor de meesten van hen is deze ervaring vaak kortstondig en weinig bevredigend. De giften van de beschaving\index{beschaving} zijn echter niet zomaar voor het grijpen; ze vereisen aanzienlijke offers, zoals het uitstellen van directe bevrediging en, meer in het algemeen, het temmen van onze menselijke instincten door ze te laten filteren door ons verstand. Dankzij ons vermogen tot rationeel denken kunnen we de verwachte resultaten van verschillende acties inschatten en zo de meest gunstige acties kiezen, zelfs als deze aanvankelijk negatieve resultaten opleveren. Zoals Mises aangeeft:

\begin{blockquotebox}
Rationeel gedrag betekent dat de mens, terwijl hij wordt geconfronteerd met het feit dat hij niet aan al zijn impulsen, verlangens en behoeften kan voldoen, moet afzien van de voldoening van hetgeen hij als minder dringend beschouwt. Om de werking van sociale samenwerking niet in gevaar te brengen, moet de mens afzien van het vervullen van die wensen waarvan de voldoening maatschappelijke instellingen zou schaden. Er is geen twijfel dat afstand nemen van bepaalde verlangens pijnlijk is. Echter, de mens heeft zijn keuze gemaakt. Hij heeft afgezien van het vervullen van sommige verlangens die onverenigbaar zijn met het sociale leven, en heeft voorrang gegeven aan het vervullen van de verlangens die alleen of op een meer overvloedige manier gerealiseerd kunnen worden in een systeem van arbeidsdeling\index{arbeidsdeling}. Hij heeft de weg naar beschaving\index{beschaving}, sociale samenwerking en welvaart gekozen.\footnotemark
\end{blockquotebox}
\footautocite{215}

Deze grootse structuur van economische samenwerking overspant duizenden jaren en omvat het werk van tientallen miljarden mensen. Het is gebaseerd op één basisprincipe: zelfbeschikking. Als je het idee van zelfbeschikking accepteert, dan kan je vreedzaam omgaan met anderen op een wederzijds voordelige manier en kun je goederen verkrijgen die je nooit zou kunnen bemachtigen als je hen in de plaats daarvan zou aanvallen. De ongelofelijke prestaties van de moderne beschaving\index{beschaving} waren alleen mogelijk doordat productieve vrije mensen wereldwijd hun werk coördineerden via vrije handel. Geen gewelddadige leider heeft ooit voor elkaar kunnen krijgen wat het vrijemarktkapitalisme in het moderne tijdperk heeft opgebouwd. Geen slaveneigenaar kon slaven ooit zover krijgen om de wonderen te produceren waartoe mensen die vrijwillig werken toe in staat zijn. Zelfs na decennia waarin miljoenen werden vermoord om gehoorzaamheid af te dwingen, produceerde de Sovjetindustrie niet meer dan geverfde roest en was het afhankelijk van handel met de kapitalistische wereld om zo lang te overleven. Het probleem is, zoals besproken in Hoofdstuk 12, niet het gebrek aan stimulans of een enkele specifieke fout; het probleem is de afwezigheid van een markt voor productiemiddelen. Zonder wijd verbreid eigendom van de productiemiddelen en de ontwikkeling van een markt voor kapitaalgoederen\index{kapitaalgoederen}, is er geen rationele manier om kapitaal\index{kapitaal} zo productief mogelijk toe te wijzen. De output van een moderne samenleving is niet iets dat kan worden geproduceerd door één toeziend oog -- het vereist wereldwijd miljarden mensen om vrijwillig te werken op een vrije markt, gebruikmakend van prijzen om de kosten en voordelen van alternatieve opties te berekenen om te beslissen welke de meest productieve en winstgevende is. Geen dwingende autoriteit kan dit nabootsen. Individuen moeten vrij zijn om de vruchten van hun arbeid te bezitten, maar ook de gevolgen van hun fouten ondergaan. Alleen dan kunnen ze de productiviteit en levensstandaard bereiken om in een beschaafde samenleving te leven. Zonder het concept van zelfbeschikking, zou elke samenleving zinken naar gewelddadige interne conflicten, waardoor productiviteit en levens worden vernietigd. Geweld vernietigt en zijn vruchten kunnen niet opwegen tegen de vruchten van productieve samenwerking onder arbeidsdeling\index{arbeidsdeling}.

Om deel te nemen aan een beschaving\index{beschaving}, moeten mensen afzien van veel handelingen die instinctief worden begeerd. De belangrijkste vereiste voor beschaving\index{beschaving} is respect voor eigendomsrechten. Om mensen bereid te krijgen tot samenwerking in een uitgebreide sociale orde, moeten ze accepteren dat andere mensen eigendomsrechten hebben over hun eigen lichaam en hun bezittingen. Zonder een alom geaccepteerde onwettigheid van agressie tegen vreemden, heeft het weinig zin om deel te nemen aan een beschaafde samenleving. Het heeft geen zin om de boom van beschaafd gedrag te planten als zijn vruchten constant voor het grijpen liggen. Het verbod op moord, mishandeling en diefstal vormt de basis van alle menselijke samenlevingen en is een belangrijk principe van religieuze en politieke instituties.

In een breder perspectief kunnen de gewoonten, tradities en morele waarden die een beschaving karakteriseren, het best worden geïnterpreteerd als gedragspatronen die essentieel zijn voor het benutten van de voordelen van economische handel en het beschaafde leven in dichtbevolkte gebieden. Eigenschappen als eerlijkheid, zorgvuldigheid en betrouwbaarheid vergroten de kans dat vreemden met elkaar in zee gaan, wat ten goede komt aan alle partijen. Seksuele terughoudendheid faciliteert de vorming van stabiele gezinnen, wat bijdraagt aan een verschuiving weg van directe bevrediging naar een focus op langetermijnwelzijn, en daarmee aan de ontwikkeling van de beschaving\index{beschaving}.\autocite{216} Immoraliteit, daarentegen, manifesteert zich in een minachting voor de toekomst, inbreuk maken op het eigendom en de rechten van anderen, bedrog, onbetrouwbaarheid, een gebrek aan geweten en een gebrek aan seksuele discipline. Deze gedragingen vergroten de kans op conflicten en ondermijnen de stabiliteit van langdurige instituties zoals huwelijken, steden en bedrijven.

Beschaafd gedrag is gericht op het verkrijgen van langetermijnvoldoening, ten koste van de onmiddellijke bevrediging die zou voortvloeien uit het volgen van onze instincten. Het impulsief toegeven aan dierlijke driften staat haaks op onze langetermijndoelen, terwijl het verstandige uitstel van bevrediging juist helpt bij het realiseren daarvan. De onbeschaafde barbaar, of het ongedisciplineerde kind dat in ieder van ons schuilt, zou het liefst direct agressie tonen tegenover iedereen die irritatie wekt, nemen wat gewenst is zonder acht te slaan op eigendomsrechten, liegen om eigen doelen te bereiken, of seksuele intimiteit afdwingen bij gewenste personen. En inderdaad, veel onbeschaafde en ongedisciplineerde individuen handelen op deze wijze. Het vereist jaren van onderwijs\index{onderwijs}, opvoeding en culturele ontwikkeling om mensen te leren hoe ze deze primitieve driften kunnen beheersen en in plaats daarvan vertrouwen op hun verstand in afwachting van toekomstige beloningen, in welke vorm dan ook. Dit is geen eenvoudige opgave, maar beschaving\index{beschaving} kan enkel voortbestaan als het menselijk verstand ons naar samenwerking leidt en onze instincten temt.

\hypertarget{het-pleidooi-voor-beschaving}{%
\section{Een pleidooi voor beschaving}\label{het-pleidooi-voor-beschaving}}

Moeten mensen zich inspannen om deel uit te maken van de beschaving\index{beschaving}? Waarom zouden mensen hun natuurlijke instinct om vreemden te bevechten en hun eigendom te bemachtigen opgeven? Is het de moeite waard om voor de materiële gemakken van economische groei onze volledige menselijkheid, inclusief alle hoogten en diepten, te verliezen? Mises geeft hierop een initiële reactie:

\begin{blockquotebox}
Biologie biedt geen standaard voor het beoordelen van veranderingen die optreden bij levende wezens, anders dan de vraag of deze veranderingen erin geslaagd zijn de individuen aan te passen aan de omstandigheden van hun omgeving en daarmee hun kansen in de strijd om te overleven te verbeteren. Het is een feit dat beschaving\index{beschaving}, beoordeeld vanuit dit perspectief, beschouwd moet worden als een voordeel en niet als iets kwaadaardigs. Het heeft de mens in staat gesteld stand te houden in de strijd tegen alle andere levende wezens, zowel de grote roofdieren als de nog schadelijkere microben; het heeft de middelen van levensonderhoud van de mens vermenigvuldigd; het heeft de gemiddelde persoon groter, flexibeler en veelzijdiger gemaakt en het heeft zijn gemiddelde levensverwachting verlengd; het heeft de mens de onbetwiste heerschappij over de aarde gegeven; het heeft de bevolkingscijfers vermenigvuldigd en de levensstandaard verhoogd tot een niveau dat de primitieve holbewoners van de prehistorische tijd nooit hadden durven dromen.\footnotemark
\end{blockquotebox}
\footautocite{217}

Dit betreft een samenvatting van het utilitaire en pragmatische argument ten gunste van de beschaving\index{beschaving}. Beschaving voorziet ons van meer materieel comfort en maakt een langer leven mogelijk. Hoewel dit voor velen een overtuigend argument lijkt, vormt het niet noodzakelijk een sluitend antwoord. Men zou kunnen stellen dat een korter, ruwer leven, waarin we onze dierlijke instincten volledig kunnen beleven, te prefereren valt boven de beperkingen van beschaafd gedrag. Het feit dat een beschaafd leven waarschijnlijk langer en comfortabeler is, impliceert niet automatisch dat het beter is dan een onbeschaafd bestaan. Uiteindelijk is waardering subjectief, en er bestaat geen objectieve, wiskundige manier om te beweren dat alle mensen de beschaving\index{beschaving} noodzakelijkerwijs boven barbaarsheid zullen verkiezen.

Een ander argument voor beschaving\index{beschaving} komt voort uit het concept van natuurlijke rechten. Mensen worden geboren met onvervreemdbare rechten en zonder het recht om die van anderen te schenden. Beschaving is simpelweg de orde die voortvloeit uit een samenleving waarin mensen hun verstand gebruiken om deze natuurlijke rechten te kennen en vervolgens elkaars natuurlijke rechten te respecteren. Dit is een argument dat overtuigend is voor mensen die al deel uitmaken van instituten van beschavingen -- vooral religies -- die hen een positieve waardering van beschaving\index{beschaving} bijbrengen. Maar dit argument is niet voor iedereen overtuigend en de meeste mensen, zelfs degenen die religieus zijn, slagen er niet in om consequent de natuurlijke rechten van anderen te respecteren en verzinnen veel manieren om hun agressie te rechtvaardigen.

Maar in plaats van beroep te doen op wiskunde of religie, kan er een zaak voor beschaving\index{beschaving} worden opgebouwd uit de methodologie van economie en dit boek: het bestuderen van menselijk handelen\index{menselijk handelen} en de gevolgen ervan afleiden, en wat dat ons vertelt over mensen. Het intellectuele brein formuleert verfijnde argumenten en wordt voornamelijk gebruikt voor amusement en rationalisering. Menselijk verstand wordt tot een bepaalde hoogte geregeerd en gereguleerd door gevolgen in de echte wereld.

Het overgrote deel van de mensen kiest ervoor om in een beschaving\index{beschaving} te leven, ook al bestaat het merendeel van het aardoppervlak uit ongetemde wildernis. Zeer weinig mensen besluiten om daadwerkelijk de beschaving\index{beschaving} te verlaten en de producten van de arbeidsdeling\index{arbeidsdeling} op te geven. Zich terugtrekken op een boerderij telt niet als het verlaten van beschaving\index{beschaving}, zolang het materieel van de boerderij het product is van kapitaalaccumulatie. Zeer weinig bevolkingsgroepen op aarde blijven afgezonderd en onwillig om contact te leggen met buitenstaanders, en zelfs deze enkele stammen zullen hun eigen geïsoleerde beschaving\index{beschaving} hebben, kapitaalaccumulatie, hoe primitief ook, in de vorm van speren en huizen, en een arbeidsdeling\index{arbeidsdeling}, hoe rudimentair dan ook.

Christopher Knight, ook wel de Kluizenaar van North Pond genoemd, was zo iemand die de maatschappij de rug toekeerde. Hij liet zijn leven achter zich en leefde meer dan 25 jaar alleen in de bossen van Maine. Maar zelfs hij was nog afhankelijk van de beschaving\index{beschaving}, omdat hij regelmatig moest stelen om te overleven. Hij verwierp de beschaving\index{beschaving} niet volledig, aangezien hij nog steeds afhankelijk was van haar producten; hij hield alleen op met zijn bijdrage eraan, wat hem tot een crimineel maakte.\autocite{218}

De meeste mensen hebben waarschijnlijk weinig kennis van de begrippen kapitalisme\index{kapitalisme} en zelfbeschikking en vinden het in de juiste context acceptabel om agressief te handelen tegen anderen en hun recht op eigendom en zelfbeschikking te schenden. Maar toch blijven kapitalisme\index{kapitalisme} en menselijke beschaving\index{beschaving} voortbestaan. Hun overleving komt niet voort uit het intellectuele begrip van de gemiddelde persoon, maar eerder uit hun zelfzuchtige verstand. Begunstigden van de beschaving\index{beschaving} kunnen lippendienst bewijzen aan het idee dat ze anderen niet nodig hebben of dat het initiëren van agressie prima is, maar ze handelen nog steeds vrijwillig met anderen voor het overgrote deel van hun leven. Ze zijn nog steeds afhankelijk van moderne technologische apparaten die alleen mogelijk zijn door een gedetailleerde arbeidsdeling\index{arbeidsdeling}. Zelfs criminelen en voorstanders van overheden die stellen dat hun agressie gerechtvaardigd is, vertrouwen om te overleven nog steeds op de producten die bestaan dankzij de arbeidsdeling\index{arbeidsdeling}, vreedzame ruilhandel\index{ruilhandel} en het wereldwijde kapitalisme\index{kapitalisme}. De wapens van de wereld worden niet geproduceerd door de meest agressieve en minst vreedzame mensen; ze worden vervaardigd door kapitalisten met een lage tijdsvoorkeur\index{lage tijdsvoorkeur} die decennialang investeren in uitgebreide kapitaalinfrastructuur, innovatieve ingenieurs gemotiveerd door kapitalistische salarissen, en productieketens die het werk van miljoenen mensen wereldwijd incorporeren. Vrijgesteld van de hypocrisie van het plukken van de vruchten van beschaving\index{beschaving}, om tegen beschaving\index{beschaving} te vechten, heeft de sterkste en meest agressieve onbeschaafde mens, net als het sterkste en meest agressieve dier, geen kans tegen een volwassene of kind in staat om de trekker over te halen van een door kapitalistische beschaving\index{beschaving} geproduceerd geweer.

Hoewel moderne intellectuelen en schrijvers uitvoerig kunnen discussiëren over de problematiek van de beschaving\index{beschaving} en de menselijke maatschappij, doen zij dit paradoxalerwijs vanuit de comfortabele omgeving van geciviliseerde kantoren en klaslokalen, binnen beschaafde gemeenschappen. Hun gedachten worden gedeeld via boeken, geproduceerd en verspreid dankzij een wereldwijde arbeidsverdeling\index{arbeidsdeling}, en bereiken lezers over de hele wereld door de samenwerking van talrijke bedrijven en hun werknemers. Niemand wordt gedwongen om binnen de beschaving\index{beschaving} te verblijven, maar ironisch genoeg lijken degenen die er het meest over klagen, zich er het minst van te kunnen losmaken.

Maar misschien wordt het meest doorslaggevende argument voor beschaving\index{beschaving} en eigendomsrechten ook ontleend aan de analyse van menselijk handelen\index{menselijk handelen} -- specifiek, de handeling van het argumenteren zelf.\autocite{219} De persoon die op zoek is naar een argument voor beschaving\index{beschaving} is een handelend mens, die graag met een ander mens wil redeneren. Het simpele feit van betrokkenheid bij een argumentatie en het zoeken naar een andere mening is een erkenning van het soevereine recht van de andere persoon op zijn of haar lichaam en bezit. Als jij, beste lezer, op een punt in jouw leven bent gekomen waar je dit boek hebt weten op te pakken -- een boek dat is geschreven, geproduceerd, gedrukt, en verspreid door talloze mensen wereldwijd in een verfijnde arbeidsdeling\index{arbeidsdeling}, met gebruik van zeer geavanceerde kapitaalgoederen\index{kapitaalgoederen} -- dan neem je deel aan een kapitalistische economische orde, waaraan je bijdraagt en waar je baat bij hebt. Het loutere feit dat je dergelijke onderwerpen kunt bespreken, is op zichzelf een afwijzing van de barbaarse bruutheid om aan al onze basale instincten toe te geven en is een manifestatie van rationeel gedrag. In plaats van gewoon te handelen vanuit een dierlijke drang om vijanden aan te vallen en hun eigendom te roven, ben je op zoek naar een rationele basis om de beschaving\index{beschaving} te ondersteunen. Je hebt een opvatting van goed en kwaad, en dus accepteer je dat je jouw wil niet zomaar aan de wereld kunt opleggen. Je erkent dat andere mensen het recht hebben op hun mening en gedachten, en je zoekt argumenten om met hen te discussiëren om te bepalen hoe je met hen omgaat.

\begin{blockquotebox}
    Het is absoluut niet mogelijk dat iemand iets zou kunnen voorstellen, of dat iemand overtuigd zou kunnen raken van een voorstel door middel van argumentatie, als het recht van een persoon om exclusief gebruik te maken van zijn fysieke lichaam niet al was verondersteld. Het is deze erkenning van elkaars wederzijds exclusieve controle over het eigen lichaam dat het onderscheidende karakter verklaart van de uitwisseling van proposities, waarbij het, hoewel men het oneens kan zijn over wat is gezegd, nog steeds mogelijk is om het er in ieder geval over eens te zijn dat de betrokken partijen niet akkoord gaan. Ook is het duidelijk dat zo\textquotesingle n eigendomsrecht van het eigen lichaam bij voorbaat gerechtvaardigd moet worden genoemd, want iedereen die zou proberen welke norm dan ook te rechtvaardigen, zou al het exclusieve recht van controle over zijn lichaam als een geldige norm moeten veronderstellen, simpelweg om ``Ik stel dit en dat voor'' te kunnen zeggen. Iedereen die zo\textquotesingle n recht zou betwisten zou verstrikt raken in een praktische contradictie, aangezien het argumenteren hiertegen al op de aanvaarding zou duiden van de norm die hij betwist.\footnotemark
\end{blockquotebox}
\footautocite{220}

Bij het schrijven van dit boek was het erg moeilijk om waardevrije economische analyse te scheiden van het pleiten voor vrije markten\index{markten} en individuele soevereiniteit en geweldloosheid als basis voor beschaafd leven. De economische argumenten voor individuele vrijheid zijn praktisch onlosmakelijk verbonden met het pleidooi voor beschaving\index{beschaving}. Het louter schrijven van een boek duidt al op het aanvaarden van het recht van anderen om hun eigen gedachten te bepalen.

Een gevolg van Hoppe\textquotesingle s argumentatie-ethiek is dat elk bezwaar tegen eigendomsrechten en arbeidsdeling\index{arbeidsdeling} alleen kan worden overwogen als deze wordt geuit zonder gebruik te maken van de vruchten van arbeidsdeling\index{arbeidsdeling} en eigendomsrechten. Elk argument tegen eigendomsrechten dat in een boek is geschreven, of besproken op televisie of het internet, moet vertrouwen op een zeer complexe beschavingsstructuur die alleen mogelijk is door eigendomsrechten en arbeidsdeling\index{arbeidsdeling}. Daarom is het gerechtvaardigd om te zeggen dat alle argumenten tegen eigendomsrechten en beschaving\index{beschaving} ongeldig zijn als ze op een andere manier worden gecommuniceerd dan door gewelddadig grommen. Het gewelddadige grommen ``argument'' is niets nieuws voor beschaafde mensen. Gewelddadig grommende dieren zijn een permanent kenmerk van de natuur, die de menselijke beschaving\index{beschaving} sinds het begin ervan hebben geplaagd, maar het ook hebben gedwongen zich aan te passen en te evolueren. Deze wilden kunnen schade en materiële en menselijke verliezen veroorzaken, maar ze zijn geen partij voor de intelligente mensen bewapend met zelfbeheersing en samenwerking onder de beschaafde arbeidsdeling\index{arbeidsdeling}. Gewelddadige dieren, zowel menselijke als niet-menselijke, zullen waarschijnlijk doorgaan met het initiëren van agressie tegen beschaafde mensen -- maar de beschaving\index{beschaving} zal hen blijven verslaan. Gewelddadige dieren kunnen de wapens die beschikbaar zijn voor een lid van de beschaving\index{beschaving} niet overwinnen. Ze zijn geproduceerd door de samenwerking van extreem grote netwerken van zeer productieve werkers en geaccumuleerd kapitaal\index{kapitaal}.

Het vinden van gebreken in het concept van zelfbeschikking, kapitalistische arbeidsdeling\index{arbeidsdeling} en beschaving\index{beschaving} is geen argument. Het is een verheerlijkte terugkeer naar apen die hun eigen uitwerpselen naar elkaar gooien--- een terugval naar een niet-menselijk dierlijk leven. Het probleem van een tegenstander van kapitalisme\index{kapitalisme} is dat je niets tegen kapitalisme\index{kapitalisme} kunt doen dat complexer is dan het gooien van je uitwerpselen zonder een kapitalist te worden. Elk wapen dat complexer is dan je eigen uitwerpselen, vereist uitgesteld genot en kapitaalaccumulatie. Elk wapen dat je niet met je blote handen kunt maken, vereist deelname aan de wereldwijde arbeidsdeling\index{arbeidsdeling}. De tegenstanders van kapitalisme\index{kapitalisme} vernietigen hun eigen vermogen om te produceren, te specialiseren en te innoveren, wat hen zwakker en minder invloedrijk maakt. Kapitalistische beschaving\index{beschaving} blijft winnen omdat haar tegenstanders ofwel machteloos hun uitwerpselen er naar gooien, of eraan deelnemen om het te proberen te bestrijden, wat effectief bijdraagt aan de vooruitgang ervan.

\hypertarget{schuldenslavernij-als-alternatief-voor-menselijke-beschaving}{%
\section{Schuldenslavernij als alternatief voor menselijke beschaving}\label{schuldenslavernij-als-alternatief-voor-menselijke-beschaving}}

De vastgelegde menselijke geschiedenis bevat vele perioden van de opkomst en de val van beschavingen, maar men kan stellen dat beschaving\index{beschaving} zich meestal in een periode van vooruitgang bevindt. Dit is te zien aan de toename van de productiviteit van werk in de loop van de tijd, de toename van het energieverbruik door de eeuwen heen en door de dalende kosten van energie. Het is ook te zien in de technologische vooruitgang van de kapitaalgoederen\index{kapitaalgoederen} waar mensen gebruik van maken. En het is te zien aan de langetermijntrend van dalende rentetarieven. Tijdsvoorkeur\index{tijdsvoorkeur} is de bepalende factor voor rentetarieven en aangezien rentetarieven op de lange termijn dalen, zoals werd besproken in Hoofdstuk 13, heeft een daling van de tijdsvoorkeur\index{tijdsvoorkeur} dit proces gestuurd. Maar dit beschavingsproces is geen vlotte lineaire verbetering geweest. Natuurrampen, oorlogen en maatschappelijk verval hebben lange perioden van terugval in de levensstandaard veroorzaakt. De wereldwijde marktorde van het Romeinse Rijk maakte een hoge mate van specialisatie en hogere productiviteit mogelijk, maar de ineenstorting van het rijk keerde dit om, en de bevolking van de oude wereld versplinterde in kleinere markten\index{markten} en kende eeuwenlang een lagere productiviteit. Er kan worden beargumenteerd dat de afgelopen eeuw getuige is geweest van een ommekeer in het proces van beschaving\index{beschaving} en een toename van de globale tijdsvoorkeur\index{tijdsvoorkeur}. Hoppe legt uit:

\begin{blockquotebox}
    In feite typeert een neiging naar dalende rentetarieven de supraseculaire ontwikkelingstrend van de mensheid. De minimale rentetarieven op \textquotesingle normale veilige leningen\textquotesingle{} waren rond 16 procent aan het begin van de Griekse financiële geschiedenis in de zesde eeuw voor Christus, en vielen tot 6 procent tijdens de Hellenistische periode. In Rome daalden de minimale rentetarieven van meer dan 8 procent tijdens de vroegste periode van de Republiek naar 4 procent tijdens de eerste eeuw van het rijk. In dertiende-eeuws Europa waren de laagste rentetarieven op \textquotesingle veilige\textquotesingle{} leningen 8 procent. In de veertiende eeuw daalden ze naar ongeveer 5 procent. In de vijftiende eeuw daalden ze tot 4 procent. In de zeventiende eeuw daalden ze naar 3 procent. En aan het einde van de negentiende eeuw waren de minimale rentetarieven verder gedaald tot minder dan 2,5 procent.
    \par\vspace{1em}\noindent
    Vanaf 1815 daalden de minimumrentetarieven in Europa en de westerse wereld gestaag naar een historisch dieptepunt onder de van ruim gemiddeld 3 procent tegen het eind van de eeuw. Met het aanbreken van het tijdperk van democratie en republiek, kwam deze vroegere trend tot stilstand en lijkt het van richting veranderd, waardoor het twintigste-eeuwse Europa en de VS als verzwakkende beschavingen zijn te zien. Een onderzoek naar de laagste gemiddelden van de rentetarieven per decennium voor bijvoorbeeld Groot-Brittannië, Frankrijk, Nederland, België, Duitsland, Zweden, Zwitserland en de VS, toont aan dat de rentetarieven in Europa gedurende de hele periode na de Eerste Wereldoorlog\index{Eerste Wereldoorlog} nooit zo laag of lager waren dan in de tweede helft van de negentiende eeuw. Alleen in de VS in de jaren vijftig kwamen de rentetarieven ooit onder het niveau van de late negentiende eeuw. Maar dit was een kortstondig fenomeen en zelfs toen waren de rentetarieven in de VS niet lager dan ze in Groot-Brittannië waren tijdens de tweede helft van de negentiende eeuw. In plaats daarvan waren de rentetarieven in de twintigste eeuw aanzienlijk hoger dan die in de negentiende eeuw overal ter wereld, en als er al een trend te zien was, dan was het een stijgende.\footnotemark
\end{blockquotebox}
\footautocite{222}

De Eerste Wereldoorlog\index{Eerste Wereldoorlog} was een keerpunt voor de mensheid, omdat je het kunt zien als het moment waarop de vooruitgang van de menselijke beschaving\index{beschaving} begon te stagneren en om te keren. De massale dood en vernietiging van de twintigste eeuw waren ongekend op historische schaal en dit werd mogelijk gemaakt door de vernietiging van vrijemarktgeld dat werd vervangen door overheidsschuld -- in wezen de beloning voor loyaliteit voor entiteiten wiens bestaansreden het initiëren van geweld is.\autocite{223} Niet alleen leidde dit tot een toename van geweld, maar de vernietiging van geld heeft de globale monetaire marktorde en de menselijke beschaving\index{beschaving} zelf langzaam met rot aangetast en heeft elke economische methode die mensen toepassen ondermijnd, zoals in detail besproken in eerdere hoofdstukken van dit boek. De rest van dit gedeelte past de analyse van mijn tweede boek, \emph{De Fiat Standaard}, toe op het economisch\index{economisch} handelen en uitgebreide marktorde die in dit boek worden besproken.

Door het vermogen van individuen om voor de toekomst te sparen te vernietigen, neemt fiatgeld\index{fiatgeld} de stimulans weg om voldoening uit te stellen, vermindert de kapitaalvorming en ondermijnt het startpunt van economische ontwikkeling en beschaving\index{beschaving}. In plaats van de zekerheid van geld te bieden om toekomstige onzekerheid op te vangen, maakt fiat\index{fiat} dat mensen schuldslaven worden van de bankierskartels van hun regering. Zoals besproken in \emph{De Fiat Standaard}:

\begin{blockquotebox}
    De huidige fiat\index{fiat}-tokens, of het nu contant geld\index{contant geld} is of op bankrekeningen staat, zijn constant onderhevig aan devaluatie doordat geldschieters nieuwe tokens kunnen creëren door krediet\index{krediet} te verlenen op basis van toekomstige ontvangst van fiat\index{fiat}-tokens. Het is daarom het meest logisch voor individuen, bedrijven en overheden om geen positieve saldi aan te houden, omdat ze zullen worden gedevalueerd door inflatie\index{inflatie}, maar in plaats daarvan te lenen. Gebruikers met negatieve saldi, dus degenen die schulden hebben, missen zekerheid en riskeren catastrofaal verlies. Financiële zekerheid, in de zin van een stabiele hoeveelheid liquide vermogen opgeslagen voor de toekomst, is niet langer beschikbaar in het huidige systeem. Je zult ofwel de afname van je vermogen waarnemen door inflatie\index{inflatie}, of je zult lenen en leven in de onzekerheid als je een paar betalingen mist. Fiat heeft effectief het sparen als financieel instrument vernietigd, met enorm negatieve gevolgen.\footnotemark
\end{blockquotebox}
\footautocite{224}

Zonder efficiënt spaargeld, en met een overheid\index{overheid} die financieel aangemoedigd wordt om meer voor iemands behoeften te zorgen, wordt de prikkel om in een gezin te investeren aangetast, en de gevolgen voor de samenleving zijn rampzalig geweest. Buiten de catastrofale impact op spaargeld schendt de opschorting van de goudstandaard\index{goudstandaard} de basis van de beschaafde samenleving: het natuurrecht. Het breekt flagrant het contract tussen de staat en eigenaars van geld om hun papieren goudbewijzen en banksaldi om te zetten in goud\index{goud}. De overheid\index{overheid} beschermt de banken die hun beloften niet nakomen en hervormt de wet om hen, en zichzelf, in staat te stellen verder te gaan met het veroorzaken van inflatie\index{inflatie} door het uitgeven van krediet\index{krediet}. Dit contract betreft iedereen in de kapitalistische economie. Zij moeten allemaal geld gebruiken om zich in de marktorde te bewegen. Wanneer geld in waarde daalt, daalt alles in waarde. Wanneer de overheid\index{overheid} -- die zich voordoet als de handhaver van contracten en hoeder van gerechtigheid -- een dergelijk solide contract verbreekt, zullen burgers onvermijdelijk hetzelfde pad volgen, minder betrouwbaar worden en oneerlijker, waardoor de basis van de beschaafde samenleving wordt ondermijnd. Wanneer het contract over geld wordt verbroken, concludeert iedereen in de kapitalistische economie dat de rechtsstaat niet voor iedereen geldt, en de samenleving verschuift van pogingen om het natuurrecht te volgen naar pogingen om deze voor persoonlijk voordeel te misbruiken. Fiat stelt de overheid\index{overheid} in staat om zichzelf steeds meer te financieren, wat resulteert in de monopolisering van de defensie- en rechtsindustrieën, en hun corruptie vervult niet de behoeften van de samenleving, maar beschermt een onproductief parasitair heersend regime.

Het toestaan van zacht geld\index{zacht geld}, in tegenstelling tot de vrije marktkeuze voor 'hard geld'\index{hard geld}, vernietigt de monetaire orde van de maatschappij door fiatgeld\index{fiatgeld}. Dit leidt tot economische cycli en de vernietiging van kapitaal\index{kapitaal}, een fenomeen dat zich uitermate vaak heeft voorgedaan in de eeuw van het centrale bankwezen. Het ondermijnt het banksysteem, hetzij door extreme hyperinflatie\index{hyperinflatie} of door het te veranderen van een cruciale instelling in de kapitalistische economie naar een beschermd monopolie dat zich richt op speculatief gokken. De winsten van dit gokken komen ten goede aan de overheid\index{overheid} en het bankkartel, terwijl de verliezen worden afgewenteld op de samenleving als geheel.

Fiatgeld verzwakt het kapitalistische economische systeem verder door het cruciale stuurmechanisme van economische calculatie te verstoren. Zodra de waarde van geld niet langer door de markt via vraag en aanbod wordt bepaald, wordt economische berekening voor ondernemers een proces vol fouten. Kapitaalmarkten transformeren meer en meer in reacties op de bevelen van monetaire autoriteiten. Als het agentschap dat de rentetarieven in de Verenigde Staten vaststelt, besluit deze te verlagen, stijgen de waarden van alle productiemiddelen—om vervolgens weer te dalen zodra de rentetarieven worden verhoogd. Economische berekeningen van winst en verlies van een bedrijf\index{bedrijf} worden een ondergeschikte overweging, en de allocatie van kapitaal\index{kapitaal} verwordt tot het speculeren op het monetaire beleid. Ondernemerschap en innovatie worden naar de achtergrond gedrongen door de decreten van de machthebbers van het fiatgeldcasino.

Fiat vernietigt ook geld als een marktgoed. We hebben geen geld meer in de zin van een alomvattend ruilmiddel\index{ruilmiddel} met hoge verkoopbaarheid\index{verkoopbaarheid} door de tijd en ruimte heen, zoals dat wel het geval was met goud\index{goud} voor de Eerste Wereldoorlog\index{Eerste Wereldoorlog}. Nu neemt een bonte verzameling van verschillende middelen de plaats in van het enige geldmiddel, waardoor het doel van geld teniet wordt gedaan en de wereld terugkeert naar een systeem van gedeeltelijke handel, waarin verschillende vormen van geld voor elkaar worden ingeruild en verschillende vormen van geld worden bewaard met verschillende overwegingen voor verkoopbaarheid\index{verkoopbaarheid}. De Amerikaanse dollar\index{Amerikaanse dollar} is het best verkoopbaar, dankzij het monopolie van de Amerikaanse Federal Reserve\index{Federal Reserve} op het wereldwijde banksysteem. Andere nationale valuta zijn beter verkoopbaar binnen de grenzen van de ``geldplantages'' van hun lokale centrale banken. Verkoopbaarheid op lange termijn en het vermogen om waarde over tijd vast te houden, zijn veel ingewikkelder. Obligaties, goud\index{goud}, vastgoed, kunst, aandelen en een eindeloze verscheidenheid aan andere middelen concurreren hier wereldwijd om, waardoor hun markten\index{markten} worden vervormd.

Bij economische calculatie gebaseerd op de steeds sterker wordende draaikolk van centraal geplande nationale valuta, die geoptimaliseerd zijn voor diefstal van de bevolking, wordt de economische calculatie van de voordelen van handel sterk vervormd, en de onzekerheid die ontstaat, ontmoedigt mensen om mee te doen aan wederzijds voordelige handel. Schommelingen in internationale wisselkoersen kunnen een winstgevende onderneming vernietigen of onrendabele bedrijven onverdiend belonen. De wereldwijde markt voor vreemde valuta verwerkt transacties die vele malen het wereldwijde bbp (bruto binnenlands product) waard zijn, omdat mensen buitenlandse valuta moeten aanschaffen om goederen uit het buitenland te kopen. Lokale en internationale handel worden ook geschaad door fiatgeld\index{fiatgeld}. Naarmate de prijzen stijgen, worden mensen voortdurend gedwongen de goederen die ze willen te vervangen door minderwaardige alternatieven, en de overheid\index{overheid} gebruikt haar inflatieprivilege om pseudowetenschappers te financieren om propagandawetenschap te produceren die beweert dat het vervangen van vlees\index{vlees} door soja, insecten en industrieel smeermiddel beter is voor de menselijke gezondheid.

In \emph{De Fiat Standaard} leg ik uit waarom inflatie\index{inflatie} van fiatgeld\index{fiatgeld} zowel de technologische vooruitgang als ons vermogen om meer energiebronnen te benutten, ondermijnt. Inflatie zorgt er tegelijkertijd voor dat zowel het spaargeld als de inkomsten van burgers devalueren, waardoor zij het vermogen verliezen om te investeren in moderne energiebronnen. Bovendien geeft het overheden de vrijheid om ongelimiteerd geld uit te geven aan propaganda, met als doel burgers af te leiden van hun dalende levensstandaard door de schuld te geven aan een reeks ongegronde zondebokken. De meest recente daarvan is het waanzinnige idee dat koolstofdioxide -- een atmosferisch gas dat essentieel is voor alle levende dingen en slechts in de minieme concentratie van 0,042\% in de atmosfeer voorkomt -- het weer op aarde om zeep helpt en apocalyptische schade aan de samenleving veroorzaakt. De enige manier om deze vermeende apocalyps te \textquotesingle repareren\textquotesingle{} is om, toevallig en wel heel heel erg van pas komend, de essentiële energietechnologieën op te geven die ons moderne leven mogelijk hebben gemaakt: koolwaterstoffen. Dezelfde energiebronnen waarvan de prijzen zeer gevoelig zijn voor inflatie\index{inflatie} vanwege hun onmisbaarheid. Fiat stelt ook regeringen in staat om onbegrijpelijk veel middelen uit te geven aan de krankzinnige missie om voldoende energie uit pre-industriële bronnen te halen om de moderne geïndustrialiseerde samenleving van stroom te voorzien. Duizenden miljarden zijn in de afgelopen decennia uitgegeven en het enige dat de \textquotesingle groene energie maffia\textquotesingle{} hiervoor kan laten zien, is een voortdurende toename van de prijs\index{prijs} en een afname van de beschikbaarheid van betrouwbare en essentiële energiebronnen.

Fiatgeld heeft ook de corruptie van wetenschappelijke kennis en onderwijsinstellingen mogelijk gemaakt. In plaats van kennis te blijven vergaren en technologisch vooruit te gaan, zijn universiteiten veranderd in instrumenten voor inflatiepropaganda en ingewikkelde smoesjes. \emph{De Fiat Standaard} bespreekt de corruptie binnen economie, voeding, en klimaatwetenschap als voorbeelden, maar de aantasting is waarschijnlijk veel omvangrijker. Er is misschien niets zo symbolisch voor de achteruitgang van de moderne academie en de morele corruptie die frauduleus afgedwongen geld oplegt aan de samenleving, dan het feit dat de meest geëerde en belangrijke econoom van de twintigste eeuw een zelfbenoemde ``immorelist'' was die zich bezighield met recreatieve kinderslavernij, zoals gedocumenteerd in \emph{De Bitcoin Standaard}. In de eigen woorden van Keynes:

\begin{blockquotebox}
    We verwierpen de persoonlijke verantwoordelijkheid om algemene regels te gehoorzamen. We eisten het recht op om elke individuele zaak op zijn waarde te beoordelen, en de wijsheid om dit succesvol te doen. Dit was een zeer belangrijk onderdeel van ons geloof, gewelddadig en agressief vastgehouden, en voor de buitenwereld was het onze meest duidelijke en gevaarlijke eigenschap. We verwierpen de gebruikelijke moraliteit, normen en traditionele wijsheid. We waren, in de strikte zin van het woord, immorelisten. De gevolgen als we betrapt zouden worden, moesten natuurlijk gezien worden voor wat ze waren. Maar we erkenden geen morele plicht voor ons en geen innerlijke sanctie om ons te conformeren of te gehoorzamen. Voor de hemelpoort eisten we zelf rechter te zijn in onze eigen zaak.\footnotemark
\end{blockquotebox}
\footautocite{225}

Met moraliteit die in slecht daglicht komt te staan en een criminele immorele persoon die wordt verheven tot geniale geleerde, begint de morele basis van een beschaafde samenleving te ontrafelen. Men kan het gebrabbel dat tegenwoordig doorgaat voor mainstream economische wetenschap, die nauwelijks meer is dan doorzichtige rechtvaardigingen voor overheidsinbreuken op natuurlijke rechten, niet begrijpen zonder te verwijzen naar het immorele karakter van zijn belangrijkste idool. Niet alleen het instituut van eigendomsrechten wordt constant door de overheid\index{overheid} geschonden in de vorm van inflatie\index{inflatie} en belasting, maar ons menselijk eigendom van onze eigen tijd wordt overtreden door het dwangmatig verbieden van de voornaamste technologie voor behoud van economische waarde door onze tijd geproduceerd: geld. We moeten steeds langer en harder werken, en met toenemende onzekerheid over de toekomst om het verlies door de diefstal van de vruchten van onze arbeid binnen de kapitalistische marktorde te compenseren. De afbraak van het gezin, samen met de toename van criminaliteit in grote steden, zijn slechts symptomen van de zich verdiepende malaise van een wereldwijde economie die zijn millennia oude kapitaal\index{kapitaal} verloren laat gaan in een proces dat de beschaving\index{beschaving} teniet doet.

Geld is de levensader van een economisch\index{economisch} systeem, de \emph{sine qua non} van economische calculatie en coördinatie. Door dit te scheiden van vrijwillige interactie en het in handen te geven van een gewelddadig monopolie, wordt het hele bouwwerk van de beschaving\index{beschaving} ondermijnd en op z'n kop gezet. Het is gemakkelijk om pessimistisch te worden over het lot van de menselijke beschaving\index{beschaving}, maar het is nog te vroeg om de hoop in de vindingrijkheid van mensen, hun technologieën, of een kapitalistisch creatief proces dat al millennia heeft overleefd en vele vijanden heeft overwonnen, op te
geven.

\hypertarget{verstand-heerst}{%
\section{Verstand heerst}\label{verstand-heerst}}

Geschiedenis toont aan dat veel mensen die in contact komen met beschaving\index{beschaving} deze niet langdurig kunnen behouden.\footnote{Ibn Khaldun, Abd Alrahman. Al-Muqaddima. 1377. Gibbon, Edward. The Decline and Fall of the Roman Empire. Alfred A. Knopf, 1994. Glubb, John. The Fate of Empires and Search for Survival. Blackwood, 1978} De begunstigden van het kapitalisme\index{kapitalisme} worden geboren in een relatieve welvaart: ze genieten van lange kinder- en adolescentiejaren, in die zin dat ze niet hoeven te werken om in hun eigen levensonderhoud te voorzien voor uitgebreide periodes van hun leven. Met de producten van het industriële kapitalisme\index{kapitalisme} binnen handbereik, kunnen gezinnen voor hun kinderen zorgen tot ze volwassen zijn, en soms zelfs daarna. Het wordt volkomen haalbaar voor leden van kapitalistische samenlevingen om geen productieve arbeid te verrichten tot ze twintig of dertig jaar oud zijn. Met zo’n loskoppeling van de realiteit van economische productie\index{productie} kunnen waanzinnige anti-beschavingsideologieën en bijgeloof gemakkelijk postvatten in de hoofden van burgers, waardoor de mentaliteit van lage tijdsvoorkeur\index{lage tijdsvoorkeur}, samenwerking en kapitalisme, die essentieel zijn voor de uitgebreide orde van economische productie\index{productie}, wordt ondermijnd. Is het mogelijk dat dergelijke ideeën de beschaving\index{beschaving} kunnen ontsporen?

De duurzaamheid en continuïteit van beschavingen zijn echter geworteld in de superieure organisatorische efficiëntie van kapitalistische economische calculatie, de enorme voordelen van vrijwillige specialisatie, en de onophoudelijke creativiteit van de mens. In fysieke oorlogvoering\index{oorlog}, net zoals in alle domeinen van menselijk handelen\index{menselijk handelen}, bevinden de tegenstanders van het kapitalisme\index{kapitalisme} zich altijd op een nadelige positie vanwege hun onvermogen om productie\index{productie} en de mobilisatie van middelen te organiseren op een wijze die vergelijkbaar is met die van het kapitalisme\index{kapitalisme}. Het gebrek aan prijsvorming en calculatie ondermijnt hun vermogen tot innovatie, terwijl de stimulans voor innovatie wordt gehinderd door restricties op winst. Ze missen toegang tot een arbeidsverdeling\index{arbeidsdeling} die zo omvangrijk en productief is als die van de grootste markt ter wereld: de wereldmarkt.

Je kunt de kapitalistische vrijemarkteconomie zien als een zeer krachtige machine, omdat zij dat in veel opzichten ook is. Alle machines en geïnvesteerd kapitaal\index{kapitaal} in private handen die gezamenlijk worden ingezet in productieprocessen, werken samen binnen één economisch\index{economisch} systeem: de uitgebreide orde van de vrijemarkteconomie. De mogelijkheid om wereldwijd miljarden machines in te zetten in productieprocessen die allemaal met elkaar zijn verbonden en van elkaar afhankelijk zijn, stelt ons in staat om veel hogere productieniveaus te bereiken dan eender welk alternatief. De mate waarin mensen zich kunnen verheffen boven het puur dag-tot-dag  overleven en zich bezig kunnen houden met economische handel wordt bepaald door de mate waarin zij de machines van het kapitalistische marktsysteem in hun leven inzetten. Deze machines bieden een enorm voordeel voor iedereen die betrokken is bij economische calculatie. Productieve mensen die deze machines maken, zullen manieren vinden om uit de buurt te blijven van fanatici, Luddieten en parasieten die ze willen vernietigen.

Maar de huidige vijand van het kapitalisme\index{kapitalisme} verschilt van de Luddieten van de vroege Industriële Revolutie, de Sovjet-boeman van de twintigste eeuw, en diverse andere disfunctionele socialistische totalitaire regimes. In tegenstelling tot expliciete externe vijanden, is de dreiging die het moderne kapitalisme\index{kapitalisme} onder ogen ziet intern, illegitiem en schijnbaar onvermijdelijker. Naarmate de wereldeconomie globaler is geïntegreerd, is zij steeds centraler komen te staan rond de Amerikaanse dollar\index{Amerikaanse dollar} en de Federal Reserve\index{Federal Reserve}. Bijna de hele wereldeconomie maakt gebruik van de Amerikaanse dollar\index{Amerikaanse dollar} of van valuta van centrale banken die de Amerikaanse dollar\index{Amerikaanse dollar} in reserve aanhouden. De overgrote meerderheid van de nationale banksystemen maakt gebruik van de Amerikaanse dollar\index{Amerikaanse dollar} en het internationale salderingsysteem van de Federal Reserve\index{Federal Reserve}. Dit betekent dat de overgrote meerderheid van de deelnemers in de wereldwijde markteconomie de waardedaling van hun geld ervaren om de uitgaven van de Amerikaanse regering en het fiat\index{fiat} bankkartel te financieren. Maar hoe inefficiënt, verspillend en ronduit crimineel dit systeem ook is, het blijft doorgaan omdat de vijanden ervan niet in staat zijn een alternatieve markteconomie van vergelijkbare grootte te exploiteren. Ondanks al zijn problemen, is het monetaire fiatsysteem nog steeds superieur aan autarkie en afzondering van de wereldeconomie. Een gewelddadig monopolie over de geldvoorraad stelt centrale overheden in staat een groot deel van de winsten van de kapitalistische vrije markt in beslag te nemen, waardoor zij profiteren van haar superieure productiecapaciteit, haar winsten gebruiken om hun controle over alle facetten van het economische leven te versterken en uiteindelijk de kapitalistische beschaving\index{beschaving} waarvan ze afhankelijk zijn, wurgen. Het kapitalisme\index{kapitalisme} heeft bewezen bedreven te zijn in het bestrijden van externe vijanden, maar hoe kan het standhouden tegen een interne parasiet die het hart controleert waarvan het kloppen het levensbloed ervan reguleert? Om te overleven, moet het kapitalisme\index{kapitalisme} een volledig onafhankelijk en alternatief hart voor het door parasieten geïnfecteerde hart uitvinden en implementeren. Dit lijkt een schijnbaar onmogelijke taak, maar het menselijk verstand zou er wel eens toe in staat zijn. Het hele beschavingsproces is gebaseerd op de systematische toepassing van verstand op menselijk handelen\index{menselijk handelen}, en naarmate de monetaire centrale planning meer een probleem wordt voor de beschaving\index{beschaving}, biedt de markt meer prikkels voor oplossingen voor dit probleem.

Kapitalistische economische calculatie ligt aan de basis van technologische innovatie. Wat een technologie succesvol maakt en uitgebreide adoptie mogelijk maakt, zijn de economische voordelen -- het vermogen om gebruikers een positief economisch\index{economisch} rendement te bieden op hun gebruik ervan. Kapitalisme is een onophoudelijk beloningsprogramma voor innovaties die problemen voor mensen oplossen. Hoe groter een probleem wordt, hoe groter de kosten die het aan de samenleving oplegt, hoe krachtiger de signalen om er een oplossing voor te vinden, en hoe groter de beloning voor het oplossen ervan. Naarmate de problemen van het huidige, aftakelende monetaire systeem duidelijker worden, zullen de meest geavanceerde technologieën, ingenieurs, en ondernemers steeds meer aangetrokken worden tot het aanpakken van deze problemen. Kapitalisme bewapent menselijk verstand ten dienste van innovaties die de mensheid ten goede komen, en beloont het naar de mate waarin het slaagt. Als fiat\index{fiat} de manier was waarop overheden het kloppende hart van de beschaving\index{beschaving} gevangen konden nemen, is kapitalisme\index{kapitalisme} het brein dat terugvecht door het verstand te stimuleren om een oplossing voor dit probleem te vinden.

Technologie is de som van hulpmiddelen die de mensheid heeft bedacht om de problemen waarmee de beschaving\index{beschaving} te maken heeft, het hoofd te bieden. De menselijke geest blijft het laatste bastion van vrijheid, en de meest geavanceerde technologie die vrije mensen kunnen produceren is software, als we deze meten in termen van productiviteit. Informatie in de vorm van letters en cijfers, op de juiste manier overgebracht, kan ervoor zorgen dat machines wereldwijd grote hoeveelheden werk kunnen verrichten en zo economische waarde creëren voor hun eigenaren. Machines doen het werk van honderden individuen, maar software laat miljoenen machines functioneren. Over de hele wereld hangt steeds meer economische productie\index{productie} af van software. Productiviteit in de industrie en software blijft groeien, maar wordt gehinderd door de afwezigheid van een vrije calculatie om nauwkeurige economische berekening mogelijk te maken om de beslissingen van kapitaaleigenaren wereldwijd te informeren. Terwijl informatie wereldwijd kan verstuurd worden, blijft geld functioneren via een enorm inefficiënt systeem dat gemanipuleerd wordt ten gunste van degenen die het bedienen. Aangezien software de meeste industrieën ter wereld binnendringt en dient als controle voor \textquotesingle s werelds industriële machines, lijkt het onvermijdelijk dat het de geldmarkt zal binnendringen en veroveren, vooral met de huidige gewelddadige en destructieve heersende fiattechnologie.

Het software-alternatief voor fiat\index{fiat} en centrale bankieren is bitcoin\index{bitcoin}, een gedecentraliseerd\index{gedecentraliseerd} peer-to-peer\index{peer-to-peer} betalingsnetwerk dat gebruik maakt van een eigen token, waarvan het aanbod beperkt is. Het belang van bitcoin\index{bitcoin} ligt in twee hoofdeigenschappen. Ten eerste biedt bitcoin\index{bitcoin} het enige werkende alternatief voor centraal bankieren voor de overdracht van geld over internationale grenzen. Ten tweede is het aanbod van bitcoin\index{bitcoin} strikt beperkt, wat betekent dat er geen manier is om de bestaande voorraad te devalueren ten gunste van een enkele persoon of instelling. Door iedereen ter wereld de mogelijkheid te bieden om te sparen in een vorm van geld dat niet kan worden gedevalueerd, kan bitcoin\index{bitcoin} het constante proces van de stijgende tijdsvoorkeur\index{tijdsvoorkeur} stoppen. Door iedereen de mogelijkheid te geven om geld internationaal te verzenden en te ontvangen zonder gebruik te maken van hun centrale bankmonopolie, stelt bitcoin\index{bitcoin} iedereen in staat om mee te doen aan de wereldwijde arbeidsdeling\index{arbeidsdeling}. Het is juist de centrale planning van deze twee markten\index{markten} -- geld en internationale overboekingen -- die aan de basis ligt van het probleem van het globale kapitalisme\index{kapitalisme}. De historische betekenis van bitcoin\index{bitcoin} is dat het een technologische oplossing biedt voor het probleem van centraal bankieren, en een technologisch en oneindig aantrekkelijker alternatief biedt dat centraal bankieren overbodig maakt.

Op dezelfde manier heeft het menselijke verstand ons opgetild van het gebruik van slaven naar paarden, naar auto\textquotesingle s, naar geavanceerde supersonische straalvliegtuigen -- en heeft het ons verheven van het gebruik van menselijke boodschappers naar postduiven, naar papieren post, naar e-mail, naar videogesprekken. En nu leidt het ons weg van het vertrouwen op monopolistische centrale banken naar betrouwbare open source software. Mensen hebben verstand en het is dat verstand dat ons uit de grotten heeft geleid en ons in staat heeft gesteld om onze omgeving te veroveren, de wildste beesten te temmen, en langer en beter te leven. Het huidige parasitaire overheidsbankmonopolie is slechts één van de vele uitdagingen die het menselijk verstand heeft geconfronteerd, en bitcoin\index{bitcoin} zou het middel kunnen zijn dat is bedacht om het te overwinnen. Met transparante regels die voor iedereen in de wereld te controleren zijn, en met een systeem dat volledig is gebouwd op verificatie in plaats van autoriteit, geeft bitcoin\index{bitcoin} de hele wereld een monetair marktgoed dat werkt zonder de noodzaak van dwingende politieke autoriteit. Het stelt ons in staat om een vreedzame basis te leggen voor de menselijke economische interactie, en brengt daarmee de productiviteit van het marktsysteem naar het monetaire terrein, waarmee we de gewelddadige hoge tijdsvoorkeur\index{tijdsvoorkeur} van het etatistische fiat\index{fiat} kunnen omkeren. Als het de menselijke beschaving\index{beschaving} kan bevrijden uit de greep van de fiatklauwen van de staat, zal bitcoin\index{bitcoin} worden herinnerd als dé meest significante stap voor de beschaving\index{beschaving} van onze tijd.
