\chapter{Eigendom}

\begin{blockquotebox}
    Alleen omdat er schaarste is, is er een probleem bij het opstellen van morele wetten; als goederen onbeperkt beschikbaar zijn (\enquote{vrije} goederen), is er geen conflict over het gebruik van goederen mogelijk en is er geen coördinatie nodig. Daaruit volgt dat elke ethiek die correct wordt opgevat, moet worden geformuleerd als een theorie van eigendom, oftewel een theorie van de toewijzing van rechten van exclusieve controle over schaarse middelen. Alleen zo wordt het mogelijk om anders onvermijdelijke en onoplosbare conflicten te vermijden.\footnotemark
    \par\raggedleft--- Hans-Hermann Hoppe
\end{blockquotebox}
\footautocite{47}


\section{Schaarste en Eigendom}

\lettrine{H}oofdstuk 3 behandelde het proces van economisch handelen als gevolg van de schaarste van menselijke tijd. Hoofdstuk 4 onderzocht hoe mensen hun tijd economiseren door een keuze te maken tussen vrije tijd en arbeid, en legde de basisprincipes uit van het productieproces. Hoofdstuk 5 onderzoekt het proces van economisch handelen met goederen en de economische reden voor het ontstaan van eigendom. Na uitleg over de economische betekenis van eigendom, behandelt dit hoofdstuk verschillende soorten eigendom, de toepassing van eigendom op zelfbeschikking en hoe eigendom als instituut helpt in de eeuwige zoektocht om de waarde en hoeveelheid menselijke tijd te vergroten.

Zoals besproken in het eerste hoofdstuk van dit boek, is schaarste het startpunt van economie en de oorsprong van al het economisch handelen. Het is het verschil tussen de gewenste en beschikbare hoeveelheid van een goed dat mensen dwingt om zorgvuldig met dat goed om te gaan, het in optimale conditie te houden om zijn functies te kunnen vervullen, en het te beschermen tegen het feit dat anderen het voor zichzelf kunnen nemen. Schaarste is wat ons dwingt om objecten te waarderen en door ze te waarderen ontwikkelen we er in de loop van de tijd controle. Schaarste is dan ook de oorsprong van eigendom. Zoals Menger uitlegt:

\begin{blockquotebox}
    Eigendom is net zoals de menselijke economie geen toevallige uitvinding. Het is eerder de enige praktisch mogelijke oplossing voor het probleem dat door de aard der dingen aan ons wordt opgelegd door het verschil tussen de behoeften aan, en de beschikbare hoeveelheden van alle economische goederen.\footnotemark
\end{blockquotebox}
\footautocite{48}


Een goed in eigendom hebben betekent de volledige controle uitoefenen over de diensten die daaruit kunnen voortvloeien. Menger definieert eigendom als \enquote{de totale som van goederen waarover een economiserend individu volledige beschikkingsmacht heeft voor de voldoening van zijn behoeften.}\autocite{49} De wetsgeleerde A. N. Yiannopolous schrijft:

\begin{blockquotebox}
    Eigendom kan worden gedefinieerd als een exclusief recht om de controle uit te oefenen over een economisch goed...; het is de naam van een concept dat verwijst naar de rechten en verplichtingen, privileges en beperkingen die de relatie van de mens met waardevolle dingen bepalen. Mensen verlangen overal en altijd naar het bezit van dingen die noodzakelijk zijn voor overleving of waardevol zijn volgens culturele normen en die, als gevolg van de vraag naar die zaken, schaars worden. Wetten, die worden afgedwongen door een georganiseerde samenleving, regelen de competitie om deze gewenste dingen en garanderen het genot ervan. Wat gegarandeerd van iemand is, is eigendom... [Eigendomsrechten] verlenen een direct en onmiddellijk gezag over een ding.\footnotemark
\end{blockquotebox}
\footautocite{50}

Eigendom is iets anders dan rijkdom, wat Menger definieert als \enquote{de volledige som van economische goederen die ter beschikking staan van een economiserend individu.} \autocite{51} Iemands eigendom bevat ook alle niet-economische goederen, maar rijkdom verwijst alleen naar economische goederen.

De economische reden voor het hebben van eigendom is duidelijk en eenvoudig. Als het gebruik van een economisch goed het niet verbruikt en het niet onbruikbaar maakt, kan het opnieuw worden gebruikt voor hetzelfde doel. De gebruiker zou het dan ook vanzelfsprekend het in eigendom willen houden tot hij of zij het weer nodig heeft. Een jager die een speer maakt om een konijn te doden, zal instinctief begrijpen dat de speer opnieuw kan worden gebruikt om een ander konijn te jagen en zal ervoor kiezen de speer in zijn bezit te houden. Heel weinig dieren hebben het instinct om objecten in bezit te nemen en misschien nemen niet-menselijke soorten alleen maar eigendom van hun woonplaatsen, nesten of holen. Door het superieure intellect van de mens kunnen wij op een veel meer verfijnde en complexe manier een bezitsdrang ontwikkelen en blijven dingen voor jaren, decennia en zelfs eeuwen het eigendom van meerdere generaties van dezelfde familie.

Door waardevolle objecten in eigendom te nemen van, kunnen mensen de kosten en tijd die nodig zijn om toekomstige taken uit te voeren, verminderen. De eigenaar van duurzame goederen kan haar doel bereiken met minder inspanning en kosten dan iemand die niet hetzelfde eigendom heeft. Het investeren van arbeid in het bouwen van een degelijk huis voor de langere termijn is een effectievere manier om onderdak te regelen dan elke dag een nieuwe provisorische oplossing te zoeken. Het temmen en houden van dieren kan een betrouwbaardere manier zijn om voedsel te verkrijgen, dan elke dag te jagen. Het cultiveren van je eigen bomen en gewassen kan betrouwbaarder en productiever zijn dan het elke dag zoeken naar planten. Dit zijn allemaal methoden waarmee mensen economiseren om hun overlevingskansen te verbeteren en de waarde van hun tijd te verhogen, met andere woorden, de hoeveelheid en de waarde van de tijd die ze op aarde hebben te vergroten.

We kunnen eigendom ook zien als een manier om de tijd die aan arbeid wordt besteed om te zetten in toekomstig nut. Door zijn arbeid te gebruiken om een duurzaam goed te produceren, ziet de mens af van directe voldoening, om zo een goed te maken dat continu nut levert over een toekomstige periode. De meest basale behoeften van de mens kunnen effectiever worden vervuld door te investeren in duurzaam eigendom. Het besteden van arbeid aan het bewerken van een stuk grond geeft een reden om op dat stuk grond te blijven en er blijvend van te profiteren. Het bezit van land maakt een langetermijninvestering en de verhoging van het nut mogelijk, meer dan wanneer het zonder eigenaar bleef, omdat het gebrek aan eigendom investeringen ontmoedigt.

Het belang van eigendom binnen de sociale context is dat het conflicten over schaarse middelen voorkomt. Zoals Stephan Kinsella het verwoordt:

\begin{blockquotebox}
    Er is altijd de mogelijkheid van conflict over betwistbare (schaarse) middelen. Dit zit in de aard van schaarse of rivaliserende middelen. Door aan elk goed een eigenaar toe te wijzen, stelt het juridische systeem van eigendomsrechten objectieve, publiekelijk zichtbare of waarneembare grenzen vast waar niet-eigenaars zich aan kunnen houden.\footnotemark
\end{blockquotebox}
\footautocite{52}


\section{Soorten Eigendom}

Fysieke goederen kunnen worden geclassificeerd in vier types: consumptiegoederen, duurzame goederen, kapitaalgoederen en monetaire goederen. Consumptiegoederen zijn de uiteindelijke doelen van het economisch handelen, ofwel de goederen die mensen aanschaffen voor eigen gebruik. Bijzondere soorten consumentengoederen zijn duurzame consumptiegoederen, die zich onderscheiden van andere consumptiegoederen doordat ze lange perioden in bezit worden gehouden, omdat hun consumptie over een langere duur kan worden uitgespreid. Voorbeelden van duurzame consumptiegoederen zijn huizen, auto's, televisies, of wasmachines. Kapitaalgoederen zijn goederen die worden aangekocht vanwege hun vermogen om consumptiegoederen te produceren. Monetaire goederen zijn goederen die niet worden bewaard om te consumeren of om consumptiegoederen mee te produceren, maar om later te worden ingewisseld voor andere goederen.

In een sociaal systeem dat bevorderlijk is voor individuen die economiseren en zo veel mogelijk conflicten proberen te vermijden, kunnen eigendomsclaims worden vastgesteld op basis van \enquote{het bestaan van een objectieve en intersubjectief verifieerbare link tussen de eigenaar en het geclaimde hulpmiddel}, zoals Hoppe het zegt.\autocite{53} In een vrije markt, of in een sociale orde zonder dwang, zijn er drie manieren voor individuen om legitiem eigendom te verwerven, zoals Rothbard uitlegt:

\begin{enumerate}
\def\labelenumi{\arabic{enumi}.}
\item Het claimen van objecten die nog geen eigenaar hebben
\item Producten die afgeleid zijn van deze objecten
\item Objecten op vrijwillige basis verkregen van rechtmatige eigenaren, hetzij door (ruil)handel, hetzij als een geschenk.
\end{enumerate}

\section{Zelfbeschikking}

Omdat mensen schaars zijn, en hun tijd ook, is het alleen maar vanzelfsprekend dat dezelfde implicaties van schaarste voor economische goederen ook op mensen van toepassing zijn. Eigendom is volgens Menger de \enquote{enige praktische oplossing}. Hoewel het idee van eigendom van mensen schokkend en moreel verkeerd klinkt, is het in economische termen onvermijdelijk. Aangezien mensen en hun tijd schaars zijn, moeten de beslissingen over hoe een mens zich gedraagt en wat hij met zijn tijd doet door iemand worden genomen. Dat is de essentie van eigendom. De persoon die beslist wat er met het lichaam en de tijd van een persoon moet gebeuren, bezit \emph{de facto} deze economisch gezien. De kwalificatie \enquote{afschuwelijk} is alleen van toepassing en gepast wanneer de vraag naar eigendom in het voordeel van iemand anders dan de persoon zelf wordt beantwoord.

Er zijn slechts drie mogelijke manieren om het eigendom van mensen te regelen:

\begin{enumerate}
\def\labelenumi{\arabic{enumi}.}
\item Via zelfbeschikking, waarbij een persoon zichzelf volledig bezit en anderen geen eigendomsrechten kunnen claimen over zijn lichaam en tijd.
\item Gemeenschappelijk eigendom, waarbij alle leden van de samenleving gezamenlijk al hun lichamen bezitten en gezamenlijk beslissen wat elk lichaam doet.
\item Slavernij, waarin een persoon het eigendom is van iemand anders en zijn eigenaren bepalen wat ze kunnen doen met het lichaam en de tijd van de slaaf. Dit varieert van het toewijzen van zijn tijd aan taken tot het toebrengen van lichamelijke schade. De rechten van slaveneigenaren gaan zelfs zo ver dat ze de slaaf mogen vermoorden. In een sociale orde met slavernij hebben sommige mensen eigendomsrecht over zowel zichzelf als anderen, terwijl anderen geen recht van eigendom hebben over zichzelf of anderen.
\end{enumerate}


De tweede optie is praktisch niet werkbaar buiten de omvang van een handjevol mensen die elkaar goed kennen, en zelfs dan zou het niet gemakkelijk zijn. Mensen zouden het erg moeilijk vinden om alle kennis te vergaren om te beslissen wat anderen met hun leven en tijd zouden moeten doen. De problemen om een mechanisme te bedenken voor informatieoverdracht, besluitvorming en uitvoering in een grootschalig sociaal systeem zijn praktisch onoverkomelijk.

De derde optie faalt op het gebied van consistentie, ethiek en al de gevolgen. Wat voor ethische basis kan er zijn om te rechtvaardigen waarom sommige mensen van zichzelf zijn, terwijl anderen in eigendom zijn van anderen? Er kan geen logisch en ethisch samenhangende manier zijn om dit drastisch verschil in toewijzing van eigendomsrechten te rechtvaardigen. Verder is dit verschil waarschijnlijk een recept voor conflicten. Het individu dat geen zelfeigendomsrechten heeft, zal proberen deze te verkrijgen en kan zich gerechtvaardigd voelen in het gebruik van geweld tegen degenen die hem bezitten. Overal waar slavernij als systeem heeft bestaan, heeft het geleid tot conflict.

Persoonlijke zelfbeschikking is de enige logische en ethisch consistente oplossing voor het probleem van menselijk eigendom, en het is de enige die waarschijnlijk zal leiden tot vreedzame samenwerking in plaats van gewelddadig conflict. Zelfbeschikking betekent dat een individu volledige aanspraak heeft op zijn eigen lichaam en tijd. Wanneer men het principe van zelfbeschikking accepteert, ontstaat er een coherent kader voor het begrijpen van rechten, rechtvaardigheid en non-agressie. Dit principe strekt zich uit tot wat een mens kan produceren als gevolg van deze keuzes, ofwel eigendom. Je kunt agressie zien als het gebruik of de dreiging van geweld om het lichaam of de tijd van een ander persoon te beheersen, en elke fysieke agressie tegen een individu zou een schending zijn van zijn recht op zelfeigendom.

Als men eigendomsrechten begrijpt als de enige werkende oplossing voor economische schaarste en als men vrede en beschaving waardeert, is het moeilijk om tegen zelfbeschikking en het systeem van eigendomsrechten te argumenteren. Elk dergelijk argument kan gezien worden als doorzichtig egoïstische hypocrisie. In plaats van een intelligent argument dat voortkomt uit het menselijk verstand, is dit argument niets meer dan een oproep om terug te keren naar de zeden van dieren die volledig door hun instincten worden beheerst, en niet in staat zijn om hun verstand te gebruiken. Argumenteren tegen zelfbeschikking is feitelijk argumenteren tegen je eigen persoonlijke identiteit, omdat het duidelijk maakt dat je geen respect hebt voor eigendomsrechten en geen deel kunt zijn van een beschaafde sociale orde. Het is een pleidooi om als een dier beschouwd te worden. Hoewel economische theorie geen politieke ideologie voorschrijft, zal begrip van economische schaarste en een waardering van vrede en beschaving een persoon ertoe aanzetten een libertarische kijk aan te nemen. oor zelfbeschikking bestaan geen alternatieven die niet tot conflict leiden en geen vijandigheid en wrok tussen individuen en groepen creëren.

Hoewel de meeste ideologieën niet expliciet voor slavernij zullen pleiten, volgen alleen libertariërs deze norm strikt tot in al zijn logische consequenties. Alle andere ideologieën geloven in minimaal een bepaalde vorm van slavernij, in de vorm van een legitieme aanspraak van anderen op iemands lichaam of tijd. Voorstanders van belasting, dienstplicht, drugsverboden of medische mandaten vinden het misschien niet leuk om zichzelf te zien als voorstanders van slavernij, maar ze geven een gedeeltelijk eigendomsrecht over iemands lichaam in handen van de staat. Ze doen dit omdat ze de staat steunen als ze haar burgers als eigendom behandelt. Dit gebeurt wanneer de staat hun inkomen met geweld afneemt, ze in de gevangenis opsluit voor het consumeren van drugs, of hen uitsluit van werk voor het niet gebruiken van door de staat verplichte farmaceutische producten.\autocite{54}

\section{Belang van Eigendomsrechten}

Door het concept eigendom te begrijpen, kan een individu beter en effectiever economiseren, en de productiviteit en waarde van zijn tijd verhogen. Door zijn arbeid te investeren in de productie van duurzame goederen, kan een mens langer van hun diensten profiteren. Dit verlaagt zijn tijdsvoorkeur en hij leert meer prioriteit te geven aan de toekomst.

Wanneer het accepteren van eigendomsrechten de geldende norm wordt in een samenleving, zijn individuen in staat om in kapitaalgoederen te investeren om met anderen te handelen. Hun productiviteit neemt verder toe en de markteconomische orde ontwikkelt zich. Dit wordt besproken in de volgende hoofdstukken. Eigendomsrechten kunnen worden begrepen als het sociale mechanisme dat mensen in staat stelt eigendom te houden in de nabijheid van anderen die het misschien ook willen bezitten. Zoals Mises het verwoordde, \enquote{Privaat eigendom van de productiemiddelen is het fundamentele instituut van de markteconomie. Het is de instelling wiens aanwezigheid de markteconomie kenmerkt. Waar het afwezig is, is er geen sprake van een markteconomie.}\autocite{55}

De markteconomie en de beschaving zelf zijn gebaseerd op het respect voor eigendomsrechten. Het is alleen bij veilige eigendomsrechten dat mensen een hoeveelheid kapitaal kunnen vergaren die significant groter is dan wat ze bij zich kunnen dragen voor hun eerste levensbehoeften. Een samenleving waarin eigendomsrechten niet worden gerespecteerd is er een waarin veel conflicten zijn. In zo'n maatschappij kunnen individuen zich niet veroorloven om hun kostbare arbeid in de toekomst te investeren, omdat al het eigendom dat deze waarde kan opslaan, riskant is om te bezitten. Een beschaafde samenleving is alleen mogelijk wanneer het recht op eigendom van jezelf en objecten in brede kring wordt gerespecteerd. Mensen kunnen dan verwachten hun eigendom zeker te stellen voor de toekomst.

In de context van een markteconomie legt Mises op prachtige wijze uit hoe het instituut van privaat eigendom zorgt voor een verantwoord beheer van middelen:

\begin{blockquotebox}
    De betekenis van privaat eigendom in de marktsamenleving is radicaal anders dan onder een systeem van autarkie van elk huishouden. In een situatie waarin elk huishouden economisch zelfvoorzienend is, dienen de productiemiddelen van het private eigendom uitsluitend de eigenaar. Alleen dit huishouden plukt de vruchten van het gebruik. In de marktsamenleving kunnen de eigenaren van kapitaal en land alleen genieten van hun eigendom door het te gebruiken om aan de behoeften van anderen te voldoen. Ze moeten in dienst zijn van de consumenten om enig voordeel te hebben van wat van hen is. Het feit dat ze productiemiddelen bezitten, dwingt hen zich te onderwerpen aan de wensen van het publiek. Eigendom is alleen een voordeel voor degenen die weten hoe ze het op de best mogelijke manier kunnen gebruiken ten behoeve van de consumenten. Het is een sociale functie.\footnotemark
\end{blockquotebox}
\footautocite{56}

De afwezigheid van privaat eigendomsrecht leidt tot conflicten tussen mensen, evenals tot de degradatie van economische goederen en natuurlijke hulpmiddelen. Wanneer economische goederen geen duidelijke eigendomsrechten hebben, zullen de individuen die ze op enig moment gebruiken en beheren, dit doen zonder de verwachting ze in de toekomst te kunnen gebruiken. Dit leidt ertoe dat ze de toekomstige staat van deze middelen niet als prioriteit zien. Deze sterke onderwaardering van de toekomst is een kenmerkend aspect van het gebruik van goederen zonder een duidelijke eigenaar. Privaat eigendom motiveert eigenaars om zich te bekommeren om de staat van hun eigendom over de lange termijn. Zoals Mises uitlegt:

\begin{blockquotebox}
    Als land in niemands bezit is, terwijl het formeel wettelijk als publiek eigendom kan worden beschouwd, wordt het gebruikt zonder enige aandacht voor de resulterende nadelen. Personen die in staat zijn om de opbrengsten voor zichzelf toe te eigenen -- het hout en wild uit de bossen, vis uit wateren en minerale afzettingen uit de bodem -- maken zich niet druk over de effecten van hun exploitatiemethoden op langere termijn. Erosie van de bodem, uitputting van eindige bronnen en andere schadelijke effecten op het toekomstig gebruik worden door hen gezien als externe kosten die geen deel uitmaken van hun berekening van input en output. Ze kappen bomen zonder nieuwe scheuten of herbebossing te overwegen. Bij jagen en vissen schrikken ze niet terug voor methoden die herbevolking van de jacht- en visgronden onmogelijk maken. In de vroege dagen van de menselijke beschaving, toen er nog volop grond van goede kwaliteit beschikbaar was, zagen mensen dergelijke roofzuchtige methoden niet als een probleem. Toen de effecten merkbaar werden in een afname van de netto-opbrengsten, verliet de boer zijn boerderij en verhuisde hij naar een andere plek. Pas als een land dichter bevolkt raakte en er geen ongebruikt land van topkwaliteit meer beschikbaar was, begonnen mensen dergelijke roofzuchtige methoden als verspilling te beschouwen. Op dat moment consolideerden ze het instituut van privaat eigendom van land. Ze begonnen met landbouwgrond en stapten vervolgens over op weiden, bossen en visgebieden.\footnotemark
\end{blockquotebox}
\footautocite{57}
