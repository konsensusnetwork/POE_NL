\chapter{Waarde}

\begin{blockquotebox}
    Waarde is dus niet inherent aan goederen. Het is geen eigenschap ervan, noch iets wat onafhankelijk op zichzelf bestaat. Het is een oordeel dat economisch\index{economisch} handelende mensen vellen over het belang van de goederen die ze tot hun beschikking hebben voor het behoud van hun leven en welzijn. Daarom bestaat waarde niet buiten het bewustzijn van mensen.\footnotemark
    \par\raggedleft--- Carl Menger\index{Carl Menger}
\end{blockquotebox}
\footautocite{13}

\lettrine{H}et eerste hoofdstuk was een methodologische inleiding tot het onderwerp economie en illustreerde de Oostenrijkse benadering waarin menselijk handelen\index{menselijk handelen} centraal staat. In dit hoofdstuk gaan we in op de kern van het vakgebied economie, de basisbegrippen en de belangrijkste vraagstukken die het vakgebied probeert te beantwoorden.

Aan het einde van de negentiende eeuw legde de Oostenrijkse econoom Carl Menger\index{Carl Menger} de fundamenten voor de moderne economie. Zijn verklaringen over de subjectieve natuur van waarde en economische keuzes, samen met de introductie van marginale analyse, zorgden voor een revolutionaire verandering binnen het domein. Deze concepten boden een stevige theoretische en methodologische grondslag die een systematische analyse mogelijk maakte van de manier waarop mensen economische\index{economisch} beslissingen nemen.\footnote{Nvdr. Economiseren, of economisch\index{economisch} handelen, houdt in dat men bewuste keuzes maakt met het doel middelen optimaal te benutten en een balans te vinden tussen behoeften en de beschikbare middelen.} Dit ondanks het feit dat economie als studiegebied al sinds de tijd van Aristoteles bestond. Mengers baanbrekende werk zorgde voor een dieper inzicht in de impact van economische activiteiten op het menselijk leven. Zijn leerboek \textit{Principles of Economics}, gepubliceerd in 1871, is wellicht het oudste economische leerboek dat vandaag de dag nog steeds van belang en leesbaar is. Dit hoofdstuk biedt een overzicht van enkele kernconcepten uit Mengers werk, waarbij zijn definities worden gebruikt als fundament voor de analyse van thema's die in latere hoofdstukken worden behandeld. Daarna worden de essentiële Mengeriaanse concepten besproken die ten grondslag liggen aan economische analyse: subjectieve waarde\index{subjectieve waarde} en marginale analyse.

\section{Nut en waarde}
\subsection{Goederen}

Menger omschrijft een goed als iets bruikbaars dat ingezet kan worden om menselijke behoeften te bevredigen. Voorwaarden om iets als een goed te beschouwen zijn: ten eerste moet er sprake zijn van een menselijke behoefte; ten tweede moeten de kenmerken van het goed in staat zijn deze behoefte te vervullen; ten derde is het noodzakelijk dat men zich bewust is van deze causale relatie; en ten slotte moet men over het goed kunnen beschikken in voldoende mate om de betreffende behoefte te kunnen bevredigen.

\subsection{Nut}

Nut is het vermogen van een goed om menselijke behoeften te bevredigen. Nut hangt af van ons vermogen om het verband te begrijpen tussen een goed en de behoefte die het vervult. Nut is een algemene voorwaarde om een object een goed te laten zijn. Alleen als iets nut kan bieden, kan het door mensen als een goed worden beschouwd.

\subsection{Schaarste}
Goederen kunnen worden onderverdeeld in twee categorieën, economische en niet-economische. Het onderscheid tussen de twee is \textbf{schaarste\index{schaarste}}: de vraag naar economische goederen is altijd groter dan het geleverde aanbod, terwijl bij niet-economische goederen het aanbod groter is dan de door mensen gevraagde hoeveelheden. 

Een \textbf{niet-economisch\index{economisch} goed} is een goed dat beschikbaar is in hoeveelheden die groter zijn dan de vraag ernaar, waardoor rivaliteit of concurrentie om het goed te bemachtigen uitgesloten is. Het beste voorbeeld is zuurstof, dat essentieel is voor het overleven van de mens, maar desondanks overal waar mensen leven in overvloed aanwezig is.\footnote{Het is schaars bij het onder water duiken en in de ruimte, en daarom wordt het in deze omgevingen een economisch\index{economisch} goed, waarvoor een geavanceerde infrastructuur nodig is om het beschikbaar te maken.} Zuurstof is daarom geen economisch\index{economisch} goed.

Omdat een \textbf{economisch\index{economisch} goed} schaars is, zal de vraag ernaar groter zijn dan het aanbod, en dit creëert rivaliteit om het te verkrijgen, waardoor mensen gedwongen worden om keuzes te maken tussen dit goed en andere goederen. De schaarste\index{schaarste} van economische goederen dwingt mensen om te \textbf{economiseren} en keuzes te maken tussen schaarse alternatieven. Economiseren, of het economisch\index{economisch} handelen, verwijst volgens Menger naar de neiging van mensen om zo groot mogelijke hoeveelheden te hebben van de goederen die hun behoeften kunnen bevredigen, om de nuttige functies van deze goederen te behouden, om hun meest dringende behoeften voorrang te geven boven minder dringende, en om de grootste voldoening te verkrijgen uit de hoeveelheid van een goed.

\subsection{Economie}

Economie is de \textbf{studie van menselijke keuzes bij schaarste\index{schaarste}}. Het richt zich op het analyseren van hoe mensen oplossingen proberen te vinden voor het probleem van het verschil tussen wat ze hebben en wat ze willen, en de gevolgen van hun keuzes. 

Omdat schaarste\index{schaarste} een permanent bestaansgegeven is, maken mensen voortdurend keuzes tussen verschillende manieren van handelen, verschillende goederen en verschillende behoeften om te bevredigen. De noodzaak om deze keuzes te maken dwingt ons om het nut dat we ontlenen aan verschillende goederen tegen elkaar af te zetten, zodat we weloverwogen keuzes kunnen maken.

\subsection{Waarde}

Waarde is onze subjectieve beoordeling van de voldoening die we uit goederen halen, of verwachten te halen. Het stelt ons in staat om economische beslissingen te nemen. Menger definieert waarde als “het belang dat individuele goederen of hoeveelheden goederen voor ons hebben, omdat we ons ervan bewust zijn dat we afhankelijk zijn van de beschikking erover voor de bevrediging van onze behoeften.”\autocite{15} Waarde is volgens Menger ook “het belang dat we eerst toekennen aan het vervullen van onze behoeften, namelijk aan ons leven en ons welzijn en als gevolg daarvan overdragen op economische goederen als de exclusieve oorzaken van de bevrediging van onze behoeften.”\autocite{16}

\subsection{Subjectieve waarde}

De basis van economische analyse, en een van de baanbrekende inzichten uit het werk van Menger, is dat \textbf{waarde subjectief is}. Het bestaat alleen in de geest van de persoon die de waarde bepaalt. Zoals Menger het stelde: “Waarde is dus niet inherent aan goederen, geen eigenschap ervan, noch een onafhankelijk iets wat op zichzelf bestaat. Het is een oordeel dat economiserende mensen vellen over het belang van de goederen die ze tot hun beschikking hebben voor het behoud van hun leven en welzijn.”\autocite{17}

Het is niet een inherente aard van goederen die ze waardevol voor ons maakt, maar alleen onze beoordeling van hun geschiktheid om aan onze behoeften te voldoen. Naarmate hun vermogen om aan onze behoeften te voldoen verandert, verandert ook hun waarde voor ons. Waarde is dus geen fysieke of chemische eigenschap van economische goederen; het is een psychische eigenschap die ze alleen krijgen wanneer mensen ze beoordelen. In de beroemde woorden van Menger: “Waarde bestaat niet buiten het bewustzijn van de mens.”\autocite{18}

Mijn favoriete voorbeeld om de subjectieve aard van waarde te illustreren is olie\index{olie}. Tot in de negentiende eeuw verminderde de aanwezigheid van olie\index{olie} de waarde van een stuk land, omdat de olie\index{olie} eerst weggehaald moest worden voordat het land gebruikt kon worden voor landbouw, handel of woningbouw. Zolang de mens olie\index{olie} als een vervuiling zag, had olie\index{olie} een negatieve economische waarde. Toen mensen zich realiseerden dat geraffineerde olie\index{olie} in een verbrandingsmotor gebruikt kon worden om machines aan te drijven die aan hun behoeften aan transport, elektriciteit en warmteopwekking voldeden, veranderde olie\index{olie} van hinderlijke overlast in een enorm waardevol en essentieel product, waar niemand in de moderne wereld nu zonder kan. Olie in het jaar 2020 verschilt chemisch en fysiek niet van olie\index{olie} in het jaar 1620, en toch is de waarde ervan veranderd van negatief naar positief. Hoewel onze bewuste inschatting van onze behoeften de fysische en chemische eigenschappen van olie\index{olie} niet kan veranderen, kan het de economische waarde ervan wel veranderen. Olie veranderde van een negatieve in een positieve waarde toen het menselijke bewustzijn het als nuttig herkende. Zoals Menger het stelt: “De waarde van goederen vloeit voort uit hun relatie tot onze behoeften, en is niet inherent aan de goederen zelf. Met veranderingen in deze relatie ontstaat en verdwijnt waarde.”\autocite{19}

Om dit punt verder te illustreren: terwijl dit boek in 2020 geschreven wordt, is een aanzienlijk deel van de wereldbevolking onderworpen aan regeringen die in de hele wereld aanzienlijke en verstikkende beperkingen van vrij maatschappelijk verkeer en economische productie\index{productie} opleggen. Olie wordt geproduceerd voor onmiddellijke consumptie\index{consumptie} en er is zeer weinig reservecapaciteit voor de opslag ervan in verhouding tot de enorme verbruikte hoeveelheden. Toen de industrie en het transport vrijwel tot stilstand kwamen, kon de overtollige olieproductie nergens heen, de olieprijs kelderde en werd zelfs een paar dagen negatief. Gezien het grote aanbodoverschot vergeleken met de vraag en het gebrek aan opslagcapaciteit, werd het bezitten van olie\index{olie} weer een last, zoals in het pre-industriële tijdperk, en de eigenaars moesten opnieuw betalen om het kwijt te kunnen. De olieprijs werd al snel weer positief en bleef stijgen. Er veranderde niets aan de inherente eigenschappen van olie\index{olie} toen de prijs\index{prijs} van negatief naar positief naar negatief naar weer positief ging; de omstandigheden van de mensen die de waarde vaststellen veranderden, en zo ook hun subjectieve waarderingen.

Zoals het voorbeeld van olie\index{olie} illustreert, kan waarde niet bestaan buiten de menselijke waardering en keuzes die hun voorkeuren weerspiegelen. Waarde kan geen constante eigenschap van objecten zijn; het is een bewust fenomeen in onze geest. Dit betekent niet dat waarde niet echt is. Waarde is echt en betekenisvol, en zij bepaalt onze handelingen en beslissingen die richting geven aan de productie\index{productie}, consumptie\index{consumptie} en het gebruik van de echte materiële objecten in onze wereld. Mengers erkenning van de subjectieve aard van waarde was een zeer belangrijk keerpunt in het economisch\index{economisch} denken. Economen uit het verleden hadden moeite om uit te leggen hoe goederen gewaardeerd werden en waarom bepaalde goederen waardevoller waren dan andere. Al deze mysteries en paradoxen rondom waardering werden pas opgelost met het Mengeriaanse inzicht van subjectieve waardering en marginale analyse.

\section{Waardering: ordinaal en kardinaal}

De eerste belangrijke implicatie van de subjectieve aard van waarde is dat deze niet objectief gemeten en uitgedrukt kan worden. Aangezien de menselijke beoordeling van waarde subjectief is en voortdurend verandert op basis van onze behoeften en ons begrip van de mogelijkheden van goederen om aan onze behoeften te voldoen, variëren waarderingen van persoon tot persoon; ook veranderen individuele waarderingen voortdurend afhankelijk van individuele omstandigheden. Om enige meting objectief vast te stellen, is een wetenschappelijke eenheid nodig als standaardmaat waartegen verschillende objecten worden beoordeeld, zoals besproken in Bijlage 1.

Gewicht, lengte, temperatuur en andere wetenschappelijke maten worden bepaald in objectief definieerbare eenheden die een nauwkeurige vergelijking tussen verschillende objecten mogelijk maken. Maar een dergelijke eenheid kan niet bestaan voor menselijke waardering, omdat de waarde van een goed geen inherente objectieve eigenschap van het goed is. Het is een subjectieve psychische eigenschap die afhankelijk is van de persoon die de waardering uitvoert, afhankelijk van de steeds veranderende omstandigheden die het nut van dat goed bepalen met betrekking tot het bevredigen van behoeften. Er is geen objectieve standaard waarmee de voldoening van mensen vergeleken kan worden, omdat de individuen zelf de beoordelaars van de waarde zijn. Met andere woorden, er is geen manier om de voldoening die een persoon uit een goed haalt objectief te meten in termen van de voldoening die een andere persoon uit hetzelfde goed haalt.

Zonder een standaard objectieve eenheid is meten onmogelijk, en waardering kan niet worden uitgedrukt in objectieve numerieke \textbf{kardinale} termen, waardoor het onmogelijk is om economische waarde met wiskundige precisie te meten. Zonder een constante eenheid als referentie voor waarde die voor iedereen vast te stellen is, is het niet mogelijk om de economische waarde van verschillende goederen ten opzichte van elkaar uit te drukken. Het is mogelijk om de lengte van verschillende voorwerpen te meten, omdat ze allemaal gemeten kunnen worden aan de hand van de constante referentie van een inch, voet, mijl of meter. Iemand die een koelkast in een keuken wil installeren, kan de toegewezen ruimte van de koelkast meten in centimeters en vervolgens de afmetingen van de koelkast opzoeken om te zien of deze zou passen. Zo’n meting is zinvol en nuttig omdat de klant en de fabrikant van de koelkast een zeer nauwkeurige en precieze gedeelde definitie hebben van wat een centimeter is. Zonder overeenstemming over een gemeenschappelijke constante eenheid zou het onmogelijk zijn om, zonder de koelkast te plaatsen, te weten of hij zou passen.

Zonder een gemeenschappelijke constante eenheid is de enige manier waarop we waarde kunnen uitdrukken \textbf{ordinaal}, waarbij goederen met elkaar worden vergeleken en gerangschikt op basis van de voorkeur van het waarderende individu, maar niet expliciet kwantitatief worden gewaardeerd. Het is mogelijk voor een individu om zijn voorkeur voor het ene goed boven het andere te kennen, omdat er een constante is voor deze vergelijking – het individu dat de waardering doet. Het is dus mogelijk om goederen te \textit{vergelijken} in termen van waarde, aangezien een individu gemakkelijk kan bepalen of hij meer waarde hecht aan goed A dan aan goed B, en meer aan goed B dan aan goed C. Maar deze waardering is puur subjectief, uitgedrukt in verhouding tot het nut dat de persoon die het waardeert, ervaart. Het is onmogelijk voor de persoon om deze voorkeuren in kwantitatieve en kardinale termen uit te drukken, zoals het waarderen van goed A met een precieze numerieke waarde uitgedrukt in dezelfde eenheid waarmee de voorkeur voor goed B wordt uitgedrukt. In de echte economie kan er niet zoiets bestaan als een verklaring die de waarde van goederen weergeeft, zoals \enquote{de waarde van A = 14,372x, de waarde van B = 4,258x, en de waarde van C = 1,273x,} waarbij x een objectieve waarde-eenheid is die gebruikt kan worden voor persoonlijke en interpersoonlijke nutsvergelijkingen.

Zoals Mises het stelt:

\begin{blockquotebox}
    Er is een meer en een minder in het wegnemen van gevoelens van ongemak; maar hoeveel de ene voldoening de andere overtreft, kan alleen maar gevoeld worden; het kan niet op een objectieve manier vastgesteld en bepaald worden. Een waardeoordeel meet niet, het ordent in een schaal van graden, het rangschikt. Het drukt een voorkeursvolgorde uit, maar geen maat en gewicht. Alleen de rangtelwoorden kunnen erop worden toegepast, maar niet de kardinale getallen.\footnotemark
\end{blockquotebox}
\footautocite{20}

Denk aan de manier waarop je persoonlijk dingen ten opzichte van elkaar waardeert. Ben je in staat om ze uit te drukken in één eenheid die ze allemaal meet? Kunnen alle dingen die je waardeert, van materiële goederen tot vriendschappen, familie en geluk, gemeten worden in dezelfde eenheid? Bestaat er een vaste wisselkoers\index{wisselkoers} tussen een familielid en materiële goederen? Kun je jouw kind waarderen op basis van een geldhoeveelheid? Hoeveel auto’s heeft een mens nodig om voor zijn kind te ruilen? Menselijke waarden kunnen niet gemeten worden met één gestandaardiseerde eenheid. Menselijke waarderingen kunnen alleen vergeleken worden, maar ze kunnen niet opgeteld, afgetrokken of vermenigvuldigd worden. Zonder een gemeenschappelijke en constante eenheid zijn metingen en wiskundige bewerkingen niet mogelijk.


\section{Waarde en prijs\index{prijs}}

De waarde van economische goederen staat los van hun prijs\index{prijs} en mag hier ook niet mee verward worden. De prijs\index{prijs} van een economisch\index{economisch} goed is niet de objectieve waardering ervan, noch de subjectieve waardering van een van de partijen die handel drijven. De prijs\index{prijs} waartegen een verkoop plaatsvindt, illustreert alleen dat de verkoper het goed minder waardeert dan de prijs\index{prijs}, terwijl de koper het meer waardeert. Als dit niet het geval was geweest, zou de transactie niet hebben plaatsgevonden.

Een veelgemaakte fout in de economie is om waarde en prijs\index{prijs} door elkaar te halen. Die fout gaat gepaard met het idee dat waarde inderdaad objectief gemeten kan worden, uitgedrukt in monetaire eenheden. Maar dat kan niet kloppen, omdat marktprijzen alleen een grens aangeven voor de waarde van goederen, die strikt zijn gebonden aan een bepaalde tijd en plaats. Wanneer iemand ermee instemt om een goed voor \$1.000 te verkopen, geeft ze daarmee aan dat ze het goed op minder dan \$1.000 waardeert. Als ze het voor meer dan \$1.000 had gewaardeerd, zou ze niet geïnteresseerd zijn geweest om het voor \$1.000 te ruilen. Alleen als haar waardering lager is dan \$1.000, zou een bod van \$1.000 haar overhalen om te verkopen. Omgekeerd, als de koper \$1.000 uitgeeft om dat goed te kopen, is het enige dat we over zijn waardering van het goed kunnen zeggen dat die hoger is dan \$1.000, anders zou hij dat bedrag er niet voor betaald hebben. Het is niet mogelijk om de precieze waardering van een individu te bepalen op basis van zijn transactie, maar alleen de boven- of ondergrens ervan. Alleen al de ruil\index{ruil} vertelt ons veel over waardering.

\section{Vrije ruil\index{ruil}}

Wanneer twee mensen er vrij voor kiezen om economische goederen te ruilen, moet het noodzakelijkerwijs waar zijn dat ze allebei geloven dat ze voordeel zullen halen uit de ruil\index{ruil}; anders zouden ze niet ruilen. Wederzijds voordelige ruil\index{ruil} geeft aan dat elke partij iets heeft ontvangen waar ze meer waarde aan hechten dan wat ze hebben opgegeven. De enige manier waarop dit mogelijk is, is als we begrijpen dat ze allebei verschillende subjectieve waarderingen van het geruilde goed hebben. Als de waarde van deze goederen objectief zou zijn, zou deze niet van persoon tot persoon verschillen, en zou de ruil\index{ruil} niet mogelijk zijn, omdat geen van beiden vrijwillig zou kiezen om het goed met de objectieve lagere waarde te accepteren in ruil\index{ruil} voor het goed met de hogere objectieve waarde. Dit wordt meer in detail besproken in Hoofdstuk 9 over handel, waarin de voordelen van ruilhandel\index{ruilhandel} worden geïllustreerd.

\section{Bepalende factoren van waarde}

Het fundamentele verschil tussen economen van de Oostenrijkse School\index{Oostenrijkse School} en andere scholen is dat Oostenrijkers waarde als subjectief beschouwen, terwijl andere scholen waarde als iets objectiefs zien, of objectief meetbaar. Om die schijn in stand te houden, definiëren sommige moderne economieboeken waarde als een functie van nut, wat gemeten wordt in een denkbeeldige en ongedefinieerde eenheid genaamd \textit{util}.\footnote{Nvdr: Nutseenheid} Er is geen standaard voor wat een util is, en geen manier om iets te meten in termen van utils. Sommige moderne wiskundige economen drukken waarde uit in expliciete numerieke termen, gemeten in monetaire eenheden, waarbij ze waarde verwarren met prijs\index{prijs} en niet kunnen verklaren waarom mensen transacties zouden aangaan om voorwerpen te ruilen als beide voorwerpen identieke waarden hebben. Marxisten daarentegen denken dat waarde bepaald wordt door de arbeid die in de productie\index{productie} van een goed gaat zitten. Dit is een absurde stelling volgens welke dingen waardevol worden als er werk in gestoken wordt om ze te produceren, ongeacht of iemand ze wil bezitten. Als je evenveel tijd zou besteden aan het bakken van een normale cake als aan het bakken van een cake van modder, dan zou de marxist beweren dat beide cakes dezelfde waarde zouden hebben.

Er gaat een intuïtieve aantrekkingskracht uit van het idee dat arbeid waarde bepaalt. We kunnen zien dat economische goederen altijd een bepaald element van arbeid vereisen om ze aan de menselijke behoeften te laten voldoen. Zelfs fruit dat in het wild groeit, vereist dat de mens de nodige arbeid verricht om het te plukken en op te eten voordat het aan zijn behoeften kan voldoen. Het is niet mogelijk om goederen te bedenken die menselijke behoeften bevredigen zonder dat er arbeid aan wordt besteed, en dit brengt de aanhangers van de arbeidstheorie tot de conclusie dat het arbeid is die waarde geeft aan goederen en dat waarde gemeten kan worden aan de hoeveelheid arbeid die eraan wordt besteed. Echter, dit is een onhoudbaar idee. 

Men waardeert goederen alleen op basis van hun capaciteit om in onze behoeften te voorzien. Bij een aankoop interesseert de tijd en moeite die in het vervaardigen van een product is gestoken de koper niet; hij kijkt enkel naar de diensten en het nut dat het product voor hem heeft. Producenten zetten arbeid in met de verwachting dat dit de consument waarde zal brengen, maar arbeid leidt niet automatisch tot waardevermeerdering. Arbeid kan men verspillen aan een productieproces\index{productieproces} dat faalt en geen bruikbaar product voortbrengt. De inspanning maakt de output niet waardevol; de nutteloosheid ervan maakt het onwaardeerbaar voor wie de waarde ervan probeert te bepalen. De hoeveelheid arbeid die men in productie stopt, garandeert niet de waarde ervan. Hoewel arbeiders hun arbeid kunnen over- of onderschatten, is het uiteindelijk de keuze van de consument op de markt die de waarde van goederen bepaalt. Producenten en arbeiders investeren hun arbeid in productieprocessen in de hoop waardevolle goederen te creëren. Als de kosten van de inputs lager zijn dan de marktprijs van de output, maakt de producent winst\index{winst}, wat aantoont dat haar investering maatschappelijk productief was omdat de input minder kostte dan de waarde van de output. Echter, als de marktprijs van het product lager is dan de kosten van de inputs, krijgt de producent een signaal dat haar productieproces destructief is, en hoe langer dit doorgaat, des te meer kapitaal\index{kapitaal} ze verspilt.

In de Oostenrijkse School\index{Oostenrijkse School} is waarde subjectief en afhankelijk van het tijdstip en de plaats waarop de waardering plaatsvindt. Waarde wordt afgeleid van menselijke keuzes die noodzakelijk zijn vanwege schaarste\index{schaarste}. Waarde wordt door individuen aan elke eenheid toegekend op het moment en de plaats waarop ze beslissingen nemen, maar het is geen universele eigenschap van het goed. Zonder een subjectieve opvatting van waarde is het niet mogelijk om coherente verklaringen te vinden voor waarom en hoe mensen de economische keuzes maken die ze maken. Hoe consumenten de subjectieve waarde\index{subjectieve waarde} van objecten bepalen, is aan hen. Hetzelfde individu zal hetzelfde goed op verschillende tijden en plaatsen verschillend waarderen, afhankelijk van vele factoren; met name hun bestaande voorraad van dat goed.

\section{Marginalisme}

Mengers andere significante bijdrage aan de economie is het concept van het marginalisme, ook wel bekend als de grensnutschool. Nadat Menger had vastgesteld dat de waarde van goederen niet inherent is aan de goederen zelf, maar eerder subjectief en afhankelijk van hun vermogen om aan onze behoeften te voldoen, paste hij dit toe op de studie van de waarde van verschillende eenheden van hetzelfde goed en legde daarmee de basis voor de moderne economische analyse. 

Aangezien de waarde van goederen wordt afgeleid van hun vermogen om aan onze behoeften te voldoen, en aangezien verschillende voldoeningen een ongelijke waarde voor ons hebben, zal de waarde van verschillende eenheden van hetzelfde goed ook ongelijk zijn, omdat deze afhangt van de behoeften waaraan ze voldoen. Hetzelfde goed zal voor dezelfde persoon een verschillende waarde hebben, afhankelijk van de behoefte waarin het op een bepaald moment voorziet.

Individuen gebruiken de eerste eenheid van een goed om te voldoen aan de belangrijkste en dringendste behoeften die ermee verbonden zijn. Ze zullen de tweede eenheid gebruiken om aan de op één na dringendste behoefte te voldoen. Naarmate de hoeveelheid van het goed dat ze bezitten toeneemt, worden de behoeften waaraan wordt voldaan minder waardevol en minder dringend. Met andere woorden, identieke goederen hebben verschillende waarden voor individuen, omdat het nut dat ze opleveren niet identiek is. De eerste eenheden zijn het waardevolst, en naarmate het aantal geconsumeerde eenheden toeneemt, is elke marginale eenheid minder waardevol dan de vorige.

Menger illustreerde dus dat de waarde die we aan goederen hechten niet afhankelijk is van hun totale of algemene nut en dat hun nut niet iets is dat inherent is aan deze goederen in abstracte zin, ongeacht hun hoeveelheden. Het belang dat we aan goederen hechten is onlosmakelijk verbonden met de hoeveelheid van die goederen, en hun hoeveelheid in verhouding tot het bestaande aanbod van het goed dat we tot onze beschikking hebben. Mensen nemen beslissingen niet op basis van het totale of abstracte nut van een object, maar op basis van het nut dat specifieke hoeveelheden van het goed bieden en hun vermogen om onze verschillende behoeften te bevredigen.

\section{Marginaal nut}

Hoewel Menger de term zelf nooit gebruikte, zou zijn leerling Friedrich von Wieser later de term “marginaal nut\index{marginaal nut}” introduceren om te verwijzen naar het belang dat gehecht wordt aan de minst belangrijke behoefte die door één eenheid van de beschikbare hoeveelheid van een goed wordt verzekerd. Mises definieert het door te zeggen: “We noemen dit gebruik van een eenheid van een homogeen aanbod waarvan de eigenaar `n' eenheden heeft, maar dat aanbod niet zou gebruiken als zijn voorraad één eenheid minder was, `n-1' eenheden, het minst dringende gebruik of het marginale gebruik. Het nut dat ervan wordt afgeleid wordt het marginale nut genoemd.”\autocite{21}

De eerste eenheid voedsel die iemand eet, is bijvoorbeeld extreem waardevol, omdat het het verschil is tussen verhongeren en overleven. De tweede eenheid voedsel zal het verschil zijn tussen louter overleven en goed gevoed zijn. Hoewel het nog steeds zeer waardevol is voor het individu, is de tweede eenheid niet zo waardevol als de eerste. Er zullen nog meer eenheden voedsel worden aangeschaft om van de smaak te genieten of voor sociale bijeenkomsten, die weliswaar waardevol zijn, maar niet zo waardevol als de vorige eenheden die werden gebruikt om te overleven en gezond te blijven. Als de voedselconsumptie van een individu blijft toenemen, komt hij uiteindelijk op een punt waarbij hij geen waarde meer hecht aan een extra eenheid voedsel en het liever niet heeft, zelfs als hij het gratis krijgt aangeboden. Een toename van het aantal geconsumeerde eenheden leidt ertoe dat de eenheden worden ingezet om aan minder dringende behoeften te voldoen, wat betekent dat elke opeenvolgende eenheid een lager nut heeft dan de vorige eenheid en dus een lagere waardering voor individuen. 

Met dit belangrijke inzicht weerlegde Menger het idee dat de waarde van goederen inherent is aan de goederen zelf. Hij illustreerde dat de waarde afhankelijk is van de behoeften waarin de goederen voorzien, die op hun beurt afhankelijk zijn van de overvloed en schaarste\index{schaarste} van de goederen, en alleen voor de persoon die de waarde bepaalt. Niemand wordt ooit gevraagd om het totale aanbod van een goed te waarderen, of om een goed abstract te waarderen. Economische beslissingen hebben alleen betrekking op individuele eenheden van goederen, en individuen nemen op elk moment in de tijd met name beslissingen over de volgende eenheid van een goed dat ze willen consumeren, niet over hun levenslange voorraad ervan, noch over het de abstracte versie van het goed zelf.

\section{Wet van afnemend marginaal nut\index{marginaal nut}}

Een belangrijke implicatie van Mengers benadering van waardering is de wet van afnemend marginaal nut\index{marginaal nut}. Deze wet stelt dat de waardering van een individu en het nut van een goed afnemen naarmate de hoeveelheid van het goed toeneemt. Aangezien individuen de eerste eenheden van een goed dat ze verwerven, gebruiken om de meest dringende behoeften te vervullen waarin het kan voorzien, moet hieruit volgen dat de eerste eenheid van een goed het hoogst gewaardeerd zal worden door dat individu. Naarmate hun bezit van dat goed toeneemt en elke marginale eenheid de minder dringende behoefte vervult, zal elke marginale eenheid een lagere waarde hebben voor het individu. De waarde die iemand aan een goed toekent, hangt op elk moment af van de behoefte die het vervult. Daarom zal de waardering voor iets verminderen naarmate men er meer van verkrijgt.

Dat het marginale nut van een goed afneemt naarmate de hoeveelheid toeneemt, is een belangrijk inzicht in de individuele besluitvorming. Iedereen die wel eens een dure aankoop heeft gedaan, kan zich dit voorstellen. Op de eerste dag dat je een nieuwe auto of nieuw speelgoed hebt, is de nieuwheidsfactor overweldigend en ben je erdoor gefascineerd. Dit wordt na verloop van tijd minder naarmate je meer gewend raakt aan de vele functies en eigenschappen. Wat nieuw was, wordt gewoon en verliest de allure die het had voordat je het ervoer. Je beleeft nog steeds plezier aan het besturen van de auto of het spelen met het speelgoed, maar het specifieke plezier neemt af met elk extra gebruik.

De wet van het afnemende marginale nut herinnert ons er nog eens aan dat er niet zoiets bestaat als een objectieve waarde van goed X, omdat die waarde verandert afhankelijk van de overvloed van goed X en de behoeften die ermee bevredigd worden. Er is altijd alleen een subjectieve waarde\index{subjectieve waarde} van de volgende (marginale) eenheid van goed X voor de persoon die de waarde bepaalt. Dit is afhankelijk van de subjectieve voorkeuren van het individu dat het waardeert en de schaarste\index{schaarste} van het goed.

\section{Waardering door het minst waardevolle gebruik}

Een andere implicatie van Mengers benadering van waardering: als individuen hun inventaris van een goed inzetten om aan hun meest dringende behoeften te voldoen, dan zal hun waardering van de marginale eenheid, hun waardering van de minst belangrijke bevrediging die dit goed verzekert, weerspiegelen. Bij het nemen van aankoopbeslissingen zal de waardering van een goed dus een weerspiegeling zijn van de waardering van de minst belangrijke bevrediging die het goed biedt. Iemand die besluit te betalen voor een maaltijd, zal dit niet betalen op basis van hoeveel waarde hij hecht aan voedsel op zichzelf of hoeveel waarde hij hecht aan al het voedsel dat hij in zijn leven heeft gegeten. Hij zal betalen naar de waarde die hij hecht aan de eerstvolgende maaltijd. De werkelijke waarde van al het voedsel zijn voor de man irrelevant. Als iemand heel zijn leven voldoende voedsel heeft gehad om gezond te blijven en op dit moment een nieuwe maaltijd kan eisen, dan waardeert hij de volgende eenheid voedsel niet net als al het voedsel dat hij tot nu toe heeft gegeten. Hij waardeert het niet alsof het het verschil is tussen leven en dood, want dat is het niet. De beslissing over de volgende maaltijd wordt gewaardeerd op basis van de behoefte die de volgende maaltijd voor deze persoon bevredigt, die, omdat het maar één maaltijd is, aanzienlijk lager zal zijn dan de waarde van voedsel dat hem in het algemeen in leven houdt of de waarde van alle voorgaande maaltijden die hem tot op heden in leven hielden. We kunnen dan zien hoe, wanneer we een keuze moeten maken over een bepaald goed, we het waarderen in het licht van het minst waardevolle gebruik dat mogelijk is, omdat dat de enige marginale keuze is die bestaat. Aan alle waardevollere toepassingen werd al voldaan met eerdere eenheden van eerder genuttigd voedsel. 

De persoon die overweegt om bijvoorbeeld een fles water van een restaurant te kopen, zal niet betalen op basis van de waarde die hij of zij aan water geeft om te overleven of om in zijn of haar dagelijkse basisbehoeften te voorzien. Ze beslissen gewoon over de marginale (volgende) eenheid water die ze gebruiken, nadat ze al andere eenheden water hebben toegewezen aan hun meer dringende behoeften. De prijs\index{prijs} die voor water wordt betaald, komt niet in de buurt van de waarde die het individu aan overleven hecht, omdat de beslissing om een fles water te kopen in een moderne stad alleen betrekking heeft op de consumptie\index{consumptie} van een extra fles water, en niet op overleven. Omdat water essentieel is voor het overleven van de mens, ontstaan alle menselijke samenlevingen alleen op plaatsen waar genoeg water is om aan de essentiële behoeften van de mensen te voldoen. Als deze behoeften verzekerd zijn, zal de prijs\index{prijs} van marginale eenheden niet de waarde van de basisbehoeften weerspiegelen, maar eerder de waarde van de minder dringende behoeften. Dit helpt ons te begrijpen waarom water relatief goedkoop is, ook al is het essentieel. De essentiële aard ervan zorgt ervoor dat mensen er meestal grote hoeveelheden van hebben en hun marginale aankoopbeslissingen baseren op de marginale eenheden die naar minder dringende behoeften gaan.

We kunnen zien waarom goederen die essentieel en belangrijk zijn om te overleven, meestal goedkoop zijn. In de moderne wereld betalen mensen niet voor water op basis van de waarde die ze hechten aan het overleven dankzij dat water. Ze leven al in een tijd en plaats die hun belangrijkste behoeften aan water tegen zeer lage prijzen veilig stelt. Hun individuele aankoopbeslissingen hebben betrekking op het verkrijgen van marginale hoeveelheden water die misschien een lichte dorst kunnen lessen, maar die niet nodig zijn om te overleven of gezond te blijven. Maar als je een individu in een situatie zou plaatsen waarin ze een paar dagen lang niet in staat is om water te kopen voor haar levensbehoeften, dan zou het minst waardevolle gebruik dat het haar zou bieden nog steeds het verschil zijn tussen leven en dood, en dat zou ervoor zorgen dat ze er veel waarde aan hecht. Zoals Mises uitlegt:

\begin{blockquotebox}
    De handelende mens bevindt zich niet in een positie waarin hij moet kiezen tussen al het goud\index{goud} en al het ijzer. Hij kiest op een bepaalde tijd en plaats onder bepaalde omstandigheden tussen een strikt beperkte hoeveelheid goud\index{goud} en een strikt beperkte hoeveelheid ijzer. Zijn beslissing om te kiezen tussen 100 ounce goud\index{goud} en 100 ton ijzer hangt helemaal niet af van de beslissing die hij zou nemen als hij zich in de hoogst onwaarschijnlijke situatie zou bevinden dat hij moet kiezen tussen al het goud\index{goud} en al het ijzer.
    \par\vspace{1em}\noindent
    Wat alleen telt voor zijn feitelijke keuze is of hij onder de bestaande omstandigheden de directe of indirecte voldoening die 100 ounce goud\index{goud} hem zou kunnen geven groter of kleiner acht dan de directe of indirecte voldoening die hij zou kunnen ontlenen aan 100 ton ijzer. Hij spreekt geen academisch of filosofisch oordeel uit over de absolute waarde van goud\index{goud} en ijzer; hij bepaalt niet of goud\index{goud} of ijzer belangrijker is voor de mensheid; hij perverteert niet als schrijver van boeken over de filosofie van de geschiedenis of over ethische principes. Hij kiest gewoon tussen twee voldoeningen die hij niet allebei tegelijk kan hebben.
    \par\vspace{1em}\noindent
    Wanneer de mens geconfronteerd wordt met het probleem van de waarde die moet worden toegekend aan één eenheid van een homogeen aanbod, beslist hij op basis van de waarde van het minst belangrijke gebruik dat hij maakt van de eenheden van het hele aanbod; hij beslist op basis van marginaal nut\index{marginaal nut}.\footnotemark
\end{blockquotebox}
\footautocite{23}

\section{Water-diamantparadox}

Een belangrijke uitkomst van Mengers marginale analyse is dat het de eerste economische verklaring bood voor de water-diamantparadox\footnote{Ook bekend als de klassieke waardeparadox.}, een vraagstuk dat economen eeuwenlang had gepuzzeld. Hoe was het mogelijk dat water, onmisbaar voor het menselijk bestaan, vaak zeer goedkoop of zelfs gratis is, terwijl diamanten, die louter luxegoederen zijn en geen essentiële behoefte vervullen, zeer kostbaar zijn? Als waarde werkelijk subjectief is, waarom hechten mensen dan zoveel waarde aan zaken zoals diamanten die ze niet nodig hebben, terwijl ze relatief weinig waarde toekennen aan essentiële zaken zoals water? Zou een arbeidswaardetheorie, die stelt dat diamanten waardevoller zijn omdat hun productie\index{productie} meer arbeid vereist, niet logischer zijn?

Zoals eerder besproken, is de marktwaarde niet gebaseerd op een inherente eigenschap van het goed of de waarde van het totale aanbod; het is gebaseerd op de minst belangrijke behoefte die het goed vervult. Omdat drinkwater doorgaans in overvloed beschikbaar is op plekken waar mensen leven, zijn de meest urgente behoeften aan water al vervuld en worden op de markt beslissingen genomen over eenheden die veel minder dringende behoeften vervullen. Wanneer iemand in een moderne stad besluit geen fles water te kopen, ziet hij af van een kleine, op dat moment minder belangrijke behoefte aan water. Hij heeft nog steeds toegang tot het water dat nodig is voor zijn meest urgente en belangrijkste behoeften, zoals overleven en hygiëne. Diamanten daarentegen zijn uiterst zeldzaam en beschikbaar in zeer beperkte hoeveelheden, en worden aangeschaft door mensen die ze voor hun meest waardevolle doeleinden gebruiken.

Men kan zich een scenario voorstellen waarin zowel water als diamanten uiterst schaars zijn, en de beschikbare marginale eenheden van beide zouden worden ingezet om in de meest urgente behoeften te voorzien. Een persoon die gestrand is in de woestijn en al dagen geen water heeft gedronken, zou bereid zijn veel meer te betalen voor een eerste eenheid water dan voor een eerste eenheid diamant, omdat water in zijn situatie het verschil tussen leven en dood betekent.

Het is daarom niet correct om te stellen dat diamanten waardevoller zijn dan water. De water-diamantparadox benadrukt het belang van de individuele omstandigheden bij het bepalen van de subjectieve waarde\index{subjectieve waarde}. In situaties waarin water overvloedig aanwezig is en diamanten schaars zijn, is water dat gebruikt wordt voor de minst waardevolle doeleinden minder waardevol dan diamanten, waarvan de schaarste\index{schaarste} ervoor zorgt dat zelfs de minst waardevolle doeleinden nog steeds van grote waarde zijn. In situaties waar water schaars genoeg is dat de marginale eenheid gebruikt wordt voor overleving, is water onmiskenbaar waardevoller dan diamanten.
