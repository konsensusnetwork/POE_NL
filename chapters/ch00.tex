\chapter{Inleiding}

De overgrote meerderheid van de studieboeken die tegenwoordig op universiteiten worden onderwezen, volgen de gangbare Keynesiaans-Samuelson economische traditie, die studenten meer in verwarring brengt dan verlicht. Ik heb jarenlang lesgegeven uit deze universitaire studieboeken en gezien hoe talloze intelligente studenten de klas met meer vragen verlieten dan waarmee ze binnenkwamen. Ze worstelden om de betekenis van obscure vergelijkingen die ze bestudeerden te begrijpen of om een overtuigende reden te zien om hun uitkomsten te geloven. Door de jaren heen heb ik met tientallen hoogintelligente studenten en afgestudeerden gesproken die een vergelijkbare ervaring rapporteerden: Ze deden wat ze moesten doen om het gewenste cijfer te halen, maar geen enkel deel van de stof hield steek. Ze probeerden zichzelf met verbazing te overtuigen om de logische sprongen te maken die nodig waren om zin te geven aan de irrelevante vergelijkingen, alleen om nooit meer de ideeën van de cursus te overwegen. Als studenten leren uit het gangbare studieboek, leren ze theoretische modellen te begrijpen die slechts een zwakke link hebben met de werkelijkheid. Succes in de cursussen bestaat uit het begrijpen van de modellen, niet de realiteit.

Tijdens mijn lessen introduceerde ik inzichten uit de Oostenrijkse school van de economie. Studenten vonden dit steevast de meest praktische en intellectueel interessante delen van de cursus, en het deel dat hen blijvende waarde bood na het behalen van een diploma. In de meeste universiteiten van vandaag worden Oostenrijkse ideeën bijna volledig genegeerd. Moderne studieboeken vermelden zelden de Oostenrijkse school, laat staan dat ze hun ideeën uitwerken. Ik moest voortdurend mijn toevlucht nemen tot diverse lezingen over verschillende onderwerpen. De meest vooraanstaande Oostenrijkse studieboeken en verhandelingen, zoals Mises' \textit{Human Action} en Rothbards \textit{Man, Economy, and State}, zijn voor de meeste moderne lezers moeilijk te verteren. Helaas besteden ze te veel tijd aan het proberen te argumenteren met de gangbare denkwijze, wat op een gegeven moment het verduidelijken van het Oostenrijkse perspectief belemmert.

Ik heb altijd een heldere, beknopte en leesbare behandeling gewild van de belangrijkste economische ideeën in de Oostenrijkse traditie, uitmondend in een begrip van het beschavingsbelang van de uitgebreide monetaire marktordening. Ik begon de contouren van zo'n studieboek te ontwikkelen voor mastercursussen en senior cursussen die ik gaf aan de Libanese Americaankse Universiteit. Na het publiceren van \textit{The Bitcoin Standard} en het vinden van een ontvankelijk lezerspubliek dat mijn schrijven over economie waardeerde, besloot ik mijn aandacht uitgebreid te richten op het schrijven van het studieboek dat ik altijd had willen onderwijzen. In 2019 besloot ik mijn universitaire baan op te zeggen en zelfstandig te gaan onderwijzen en publiceren, op mijn website saifedean.com. In 2019 en 2020 ontwikkelde ik twee Principles of Economics-cursussen, ECO11 en ECO12, die de ideeën verder uitwerkten die zouden uitgroeien tot dit boek.

Door te doceren en te interageren met honderden studenten van over de hele wereld en bevrijd te zijn van de steeds meer archaïsche en esoterische tijdschriften en uitgevers van de academische publicatiemolen, kon ik me nu concentreren op het schrijven voor de lezer, niet voor commissies van academici. Na twee decennia van studeren en leren van economie op universitair niveau, vertegenwoordigt dit boek de kennis van economie die ik graag had willen hebben toen ik 17 was. Het is wat ik hoop dat mijn kinderen zullen lezen wanneer ze nieuwsgierig worden naar economie.

Dit boek vormt een introductie tot de principes van economie en de economische denkwijze -- een krachtig instrument van mentale planning dat voor iedereen nuttig is om te begrijpen. Op een universiteit zou ik dit boek gedurende twee semesters onderwijzen om studenten een breed beeld te geven van het onderwerp economie en de economische denkwijze. Meer dan alleen een universitair studieboek, is dit een boek geschreven voor een algemeen publiek dat geïnteresseerd is in economische ideeën. Zelfs als je geen economie studeert aan een universiteit, neem je elke dag van je leven economische beslissingen. Voor deze lezer hoop ik dat dit boek een beknopte en toepasbare samenvatting biedt van de meest nuttige inzichten van de economische denkwijze, wat nuttig zou zijn bij persoonlijke en zakelijke besluitvorming.

Dit boek hanteert onverontschuldigend een Oostenrijkse benadering. Het gebruikt eenvoudige geschreven tekst om uit te leggen wat veel economen door de geschiedenis heen hebben beschouwd als de krachtigste methoden om economische verschijnselen te begrijpen. Het past de benadering van menselijke actie toe om de belangrijkste concepten en onderwerpen in de economie uit te leggen, voortbouwend op het werk van de economen van de Oostenrijkse school. Het behandelt belangrijke economische concepten en onderwerpen op een onafhankelijke manier, maar in een logische volgorde die als doel heeft de lezer een begrip van economie op individueel en maatschappelijk niveau te bieden, evenals de wijdverbreide implicaties van economie als onderwerp. Het eerste deel van het boek introduceert de fundamentele concepten in de economie en de Oostenrijkse methode van dit boek. Het tweede deel van het boek, Economie, introduceert de acties die individuele mensen uitvoeren om te economiseren. Deel III, De Marktordening, onderzoekt het economiseren in de sociale context, waarom de kapitalistische economie zich ontwikkelt, en de rol van geld. Deel IV, Monetaire Economie, onderzoekt tijd, rente, en monetaire en financiële economie. Deel V, Beschaving, onderzoekt de economie van geweld en veiligheid en wat ze impliceren voor de mogelijkheid om de menselijke beschaving te bevorderen.

Elk hoofdstuk van dit boek bespreekt een belangrijk economisch concept en kan worden gelezen als een op zichzelf staand essay over het onderwerp. Maar het boek is ook gestructureerd als een monografisch verhaal, waarbij deze concepten in een logische volgorde worden gepresenteerd. Het eerste hoofdstuk introduceert de Oostenrijkse methodologische benadering van de economie, geeft een voorbeeld en vergelijkt dit met de methodologische benadering van de natuurwetenschappen. Hoofdstuk 2 introduceert het fundamentele concept van waarde, legt de subjectieve aard ervan uit, evenals de concepten van nut en marginale analyse, gebaseerd op het werk van Carl Menger, de grondlegger van de Oostenrijkse school. Hoofdstuk 3 benadrukt het belang van tijd in de economie, de unieke aard van het economiseren van tijd, en hoe alle economische handelingen kunnen worden gezien als pogingen om de hoeveelheid en subjectieve waarde van onze tijd op aarde te verhogen. Dit hoofdstuk introduceert ook de cruciale concepten van opportuniteitskosten en tijdsvoorkeur.

Het tweede gedeelte van het boek introduceert de belangrijkste handelingen die mensen uitvoeren om individueel te economiseren. In elk van de hoofdstukken van dit gedeelte wordt een sleutelconcept geïntroduceerd en geanalyseerd in termen van de redenen waarom mensen zich erin begeven, het probleem dat het oplost en hoe het hen helpt tijd te besparen. Het eerste en meest basale concept is arbeid, het onderwerp van Hoofdstuk 4. Hoofdstuk 5 legt de economie van eigendom uit, waarom het ontstaat, het probleem dat het oplost en het concept van zelfeigendom. Hoofdstuk 6 introduceert een specifiek type eigendom, kapitaal, dat bestaat uit goederen die worden gebruikt voor de productie van andere goederen, en bespreekt de kosten van kapitaal, de productiviteit ervan en de verbinding met tijdsvoorkeur.

Hoofdstuk 7 bespreekt technologie als een economisch concept, waarom het de arbeidsproductiviteit verhoogt en de unieke status als een immaterieel economisch goed dat niet schaars is. Het hoofdstuk sluit af met een bespreking van het concept van intellectueel eigendom, en hoe de niet-schaarse aard van informatie verschilt van andere productiegoederen.

Energie, het onderwerp van Hoofdstuk 8, is geen conventioneel onderwerp in de meeste economische leerboeken. Echter, ik geloof dat het begrijpen van de economie van energie essentieel is voor het begrijpen van economie, met name omdat de moderne kapitaalintensieve en technologisch geavanceerde markteconomie niet mogelijk zou zijn zonder aanzienlijke toenames in de kracht van de moderne mens - het vermogen om grote hoeveelheden energie in korte tijd te gebruiken. Bovendien is het benaderen van economie via de Oostenrijkse methode, door middel van marginale analyse, essentieel om de realiteiten van energieproductie in de wereld van vandaag te begrijpen.

Terwijl het tweede gedeelte van het boek individuele economiseringsacties onderzoekt, bekijkt het derde deel van het boek economiseren in een sociale context, door andere mensen in de analyse te introduceren en de implicaties te verkennen. Zodra er een andere persoon aanwezig is, wordt handel mogelijk en hebben beide partijen een stimulans om eraan deel te nemen, omdat het hen beiden ten goede komt. Hoofdstuk 9 legt de rationale van handel uit, de voordelen ervan en de implicaties van de groei van de markt waarin de arbeidsdeling plaatsvindt.

Hoofdstuk 10 introduceert het concept geld, waarbij wordt uitgelegd welke problemen het oplost, hoe deze problemen de gewenste kenmerken van geld vormgeven en hoe geld mensen helpt te bezuinigen en de waarde en productiviteit van hun tijd te vergroten. Het hoofdstuk legt uit dat geld een product van de markt is en niet van de staat, zoals vaak maar ten onrechte wordt onderwezen in economische leerboeken. Hoewel dit hoofdstuk geld introduceert, wordt de bredere discussie over monetaire economie overgelaten aan Deel IV, zodat het de discussie over kapitaalmarkten kan volgen, een essentieel onderwerp in monetaire economie.

De sociale orde waarin individuen vreedzaam deelnemen aan alle eerder genoemde economiseringsacties wordt een marktorde genoemd. Hoofdstuk 11 onderzoekt hoe individuele voorkeuren en economiseringsacties leiden tot het ontstaan van prijzen, waarvan het essentiële belang voor het marktproces wordt uitgelegd. Hoofdstuk 12 legt de term kapitalisme uit in de Misesean-traditie en hoe het een ondernemerssysteem is dat onlosmakelijk verbonden is met privébezit en economische berekening. We onderzoeken Mises' lakmoesproef voor het bepalen of een samenleving een markteconomie heeft en hoe het ons kan helpen de economische geschiedenis te begrijpen.

Deel IV, Monetaire Economie, benadert het onderwerp geld vanuit een Oostenrijks perspectief en dus begint Hoofdstuk 13 met tijdvoorkeur, en de relatie ervan met sparen, geld en kapitaalaccumulatie, wat krediet en bankieren mogelijk maakt, de onderwerpen van Hoofdstuk 14, dat ook rentetarieven uitlegt en of ze kunnen worden geëlimineerd. Hoofdstuk 15 onderzoekt het Oostenrijkse begrip van de conjunctuurcyclus door de onderliggende oorzaak ervan te onderzoeken, monetaire uitbreiding via de uitgifte van circulatiekrediet.

Zoals de eerdere delen illustreren over de functie en vorm van een kapitalistische markteconomie, en hoe deze alleen kan werken in een systeem dat respect heeft voor privébezit, onderzoekt het vijfde en laatste deel van het boek, Beschaving, de levensvatbaarheid van de kapitalistische beschaving tegen de dreiging van gewelddadige agressie. Hoofdstuk 16 onderzoekt de economie van geweld, zowel in privé- als in overheidsvorm, terwijl Hoofdstuk 17 de economie van verdediging onderzoekt en laat zien hoe dit slechts een ander marktgoed is, dat tegenwoordig voornamelijk op de markt wordt geleverd.

Het laatste hoofdstuk van het boek bespreekt het concept van beschaving vanuit een economisch perspectief. Beschaving wordt gezien als een orde die ontstaat wanneer een samenleving vredig, productief, laag in tijdvoorkeur, coöperatief en innovatief genoeg kan blijven om intergenerationele verbeteringen in levensstandaard te ondersteunen. De kosten van deze monumentale onderneming worden besproken, evenals de kansen voor het voortbestaan van de kapitalistische beschaving in het licht van de formidabele bedreigingen waarmee ze wordt geconfronteerd.

Dit boek wordt aangevuld door zijn webpagina, saifedean.com/poe, waar je een volledige bibliografie kunt vinden met live links naar de lezingen die in dit boek zijn vermeld. Omdat het internet zo alomtegenwoordig is geworden, besloot ik dat het logisch zou zijn om de papieren versie van dit boek te optimaliseren voor de leeservaring door url's uit verwijzingen te verwijderen en een live volledige bibliografie op saifedean.com/poe te bewaren. Na het afronden van dit boek zal ik nog een online cursus aanbieden op saifedean.com om deze stof dieper te bestuderen.

Dit boek heeft enorm geprofiteerd en is sterk verbeterd als gevolg van de feedback van Ross Stevens, Jeff Deist, Per Bylund, Conza, Allen Farrington, Jonathan Newman, Peter Young en Thomas Semaan. De laatste twee leverden ook uiterst waardevolle onderzoeksassistentie gedurende het schrijven van dit boek. Ik bedank ook van harte de uitstekende redacteuren wiens grondige en nauwgezette redactie dit manuscript enorm heeft verbeterd: Alex McShane, Steve Robinson, Chay Allen, Renata Sielecki, Magda Wojcik, Evan Manning en Elizabeth Newton. Ik bedank ook Tamara Mikler voor het produceren van de graphics en Max DeMarco voor het bewerken van het audioboek. Ik ben ook erg dankbaar voor het saifedean.com-team van Pavao Pahljina, Marko Pahljina, Dorian Antešić, Flora Fontes en Valentino Cnappi voor alle moeite die ze hebben gestoken in het runnen van de website en het regelen van de publicatie.

Dit boek zou niet mogelijk zijn geweest zonder de steun, aanmoediging en feedback van leden van mijn online leerplatform saifedean.com. Ik ben hen zeer dankbaar dat ze me in staat hebben gesteld om productief te werken aan het afronden van mijn werk. In het bijzonder gaat mijn oprechte dank uit naar mijn lezers die de publicatie van dit boek hebben gesteund door de gesigneerde exemplaren in voorbestelling te kopen. Dank je wel A Patel, Aaron Macy, Abdulla Al Abbas, Abdullah Almoaiqel, Ágúst ragnar Pétursson, Aidan Campbell, AJ Garnerin, Alex, Alex Bowe, Alex Voss, Alistair Milne, Amit Barkan, Anderson Thees, Andrea Bortolameazzi, Andrew Brasuell, Andrew Rosener, Andrew Stange, Anthony Clavero, Antonio Caccese, Ashok Atluri, ben johnson, Bertrand Marlier, BitcoinTina, Blake Canfield, BowserKingKoopa, brian daucher, Brian Kim, Brian Lockhart, Bronson Moyen, Browning Hi-Power 9mm, Bryan Matthieu, Bryan Wilson, Burcu Kocak, Carlo Barbara, Carlos Chida, Caspar Veltheim, Cedric Youngelman, Chase Oleson, Chen YH, Chris Cowlbeck, Christian Amadasun, Christof Mathys, Christopher Lamia, Christopher P Valle, Christopher Pogorzelski, Christopher To, Cletus Reynolds, Dale Williams, Dan Skeen, Dane Bunch, Daniel Ostermayer, Daniel Smith, Dave Hudson, David Heller, David Lawant, Dirk Seeber, Domingo Ochotorena, Dylan Parker, Ed Becker, Eduardo Lima, Edward Cosgrove, Ernest Huttel, Fabian von Schilcher, Federico Quintela, Francisco Reyes, Frank Acklin, Gary Lau, Gary Speed, Gen Shin, Glenn Thomas, Greg Doyle, Haris M, Harlan Robinson, Hayden Houser, hugh  starr, Hunter Hastings, Jaap Willems, Jackson Forelli, Jaeger Hamilton, James Seibel, James Weaver, Jason DiLuzio, Jawad Barlas, Jerrold Randall, Jesse Powell, Jim Patterson, Joachim Boudet, John A. Krpan, John Brier, John Dixon, Jon E, Jonas Karlberg, Jonas Konstandin, Jonathan Camphin, Jonathas Carrijo, Jordan Wilby, Jose Areitio Arberas, José Niño, Jules, Julio Neira, Justin Schwartz, Keith G, Kelly Lannan, Kenneth Gestal, Kevin Coffin, Kim Butler, Lachie McWilliam, Larry Salibra, Leo Smith, Luis Alonso, Maksymilian Korzuchowski, Manuel Tomasi, Marco Daescher, Marcus Dent, Marius Kjærstad, Marius Reeder, Martin Brochhaus, Matija Grlj, Matt, Matthew Robin, Matthew Sellitto, Max Cash, Maximiliano Guimarães, Michael Atwood, Michael Culhane, Mike Clear, Mitch Soboleski, Mitchell Vanya, Nate Kershner, Nathan Smith, Neal Nagely, Nelson, Nicholas Sheahan, Nick Giambruno, Niko Laamanen, The Noded Podcast with Pierre Rochard and Michael Goldstein, Odi Kosmatos, Oleg Mikhalsky, Paweł Sławniak, Petar, Petr Zalud, Prince Filip Karađorđević, Raycheslav Karagyozov, Rene Bos, Richard Duke, Robert Koonce, Robin Dea, Ronald Zandstra, Rosie Featherby, Ross Stevens, Rowais Hanna, Ryan Nadeau, Ryan Sandford, Saagar Singh Sachdev, Sam Dib, Sam Shams, Samuel Douglass, Scott Manhart, Scott Schneider, Scott Shell, Seb Walker, Shakti Chauhan, Shaun McFarlane, Simonna Pencev, Stefano D'Amiano, Stephen Labb, Subhan Tariq, Tanner Dowdy, Thierry Thierry, Thomas Jenichen, Tom Karadza, Travis Tripodi, Trevor Smith, vik, Wendy Hiam, Wilfred Tannr Allard, Will Phillips, William Green, William Johnston, Wityanant Thongsawai, Yani Eberding, Yoism, Zachary Hollinshead, Zarak Ortega, Zsuzsanna Glasz