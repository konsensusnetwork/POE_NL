\chapter{Voorwoord door Tom van Lamoen}

Saifedean Ammous is meesterlijk in staat om de principes van de Oostenrijkse School op een concrete manier te presenteren en toegankelijk te maken voor iedereen die er voor het eerst mee in aanraking komt. Echter, dit werk tekortdoen door het te bestempelen als `De Oostenrijkse School voor Dummies' doet geen recht aan de diepgang ervan. Op een bijzonder elegante wijze weet Ammous de essentie van deze economische benadering en de filosofie van het libertarisme samen te vatten tot een coherent geheel waarin geen enkele lacune te ontdekken valt. Hij opent zijn betoog met de opmerking dat hij dit boek graag gedurende zijn hele loopbaan als economisch docent bij de hand had gehad, om adequaat de vragen van zijn studenten te kunnen beantwoorden.

In de huidige samenleving vertonen alle vormen van menselijke interactie scheefgroei, voornamelijk doordat educatie over de basis van menselijk handelen in het huidige economische en monetaire systeem volledig ontbreekt. Voor mij heeft educatie en het verspreiden van het gedachtegoed over de huidige status quo, evenals mogelijke oplossingen, de hoogste prioriteit. Zijn boek \emph{De Bitcoin Standaard} is voor mijzelf het meest geschonken boek, om dezelfde reden.

\subsection{Vrijwillig handelen}

Vrijwillig handelen staat centraal in Hoofdstuk 1, waar we direct de diepte ingaan van de filosofische basis van praxeologie, het menselijk handelen, geïnspireerd door het werk van Ludwig von Mises --- \emph{Human Action}. Als ik dan de kans krijg om mijn eigen visie te delen, benader ik dit vanuit mijn persoonlijke overtuiging. Ik beschouw de spirituele en morele kant van vrijwillig handelen als de basis van het `voluntarisme'. De Oostenrijkse School toont aan dat de mens altijd bewust handelt, wat Mises `doelgericht gedrag' noemt. Het lijkt de weg van de minste weerstand te volgen, maar het kan niet het volledige spectrum van handelen omvatten, aangezien menselijke acties niet voorspelbaar zijn.

In onze samenleving trachten diverse autoritaire krachten het menselijk handelen te beïnvloeden, terwijl de staat hierin een leidende rol speelt. Het bestaansrecht van de staat verkrijgt zij door dwang (belasting), waardoor alle regelgeving de keuzes van mensen beperkt. Dit vormt een ernstige bedreiging voor de vrijheid van het individu en is schadelijk voor de samenleving en economie. Onder het gezag van de staat kan er volgens mijn overtuiging geen vrije marktwerking bestaan, en bijgevolg geen vrijwillige actie.

\subsection{Vrijwillig eigendom}

Vrijwillig eigendom is een ander kernprincipe van het libertarisme, geworteld in het eigendomsrecht. Zelfbeschikking over lichaam, arbeid en bezit is absoluut, zonder uitzondering of verwatering. Als je geen volledige zeggenschap hebt over je eigen lichaam, arbeid of bezit, ontbreekt zelfbeschikking automatisch. Vrijwilligheid ligt ook hier aan de basis. Het is altijd mogelijk om een deel van je lichaam, arbeid of bezit over te dragen aan een ander, maar alleen als dit voortkomt uit vrijwillige interactie valt dit onder zelfbeschikking. Zelfs bij een minimale overschrijding van de vrijwillige basis kan er sprake zijn van aanranding, slavernij of diefstal.

Vrijwilligheid vormt dus de basis van elke vorm van vrijheid. Als we kijken naar het concept van privacy, citeer ik graag uit het fantastische document genaamd \emph{The Cypherpunk Manifesto}: `Privacy is noodzakelijk voor een open samenleving in het elektronische tijdperk. Privacy is niet hetzelfde als geheimhouding. Een privékwestie is iets wat men niet wil dat de hele wereld weet, maar een geheime zaak is iets wat men niet aan iemand wil onthullen. Privacy is de kracht om selectief delen van zichzelf aan de wereld te openbaren.' Het selectief delen van informatie kan alleen vrijwillig plaatsvinden.

\subsection{Vrijwillige handel}

Vrijwillige handel is naar mijn overtuiging enkel mogelijk in een vrije markt, waar alle transacties op volledig vrijwillige basis plaatsvinden. De vrije markt, zoals ik die voor ogen heb, is een manifestatie van de chaostheorie, ogenschijnlijk onvoorspelbaar van dichtbij, maar altijd geworteld in een rationele basis. Het is een optelsom van vrijwillig menselijk handelen, waarbij individuen gezamenlijk, volgens het principe van `wisdom of the crowd', bepalen wat de vraag, het aanbod en de bijbehorende prijs voor een product of dienst zullen zijn. In deze dynamiek verschaffen marktsignalen constant inzicht in de actuele vraag en aanbod, waardoor de prijs een eerlijke weerspiegeling is van de huidige situatie.

Echter, als de vraag of het aanbod gemanipuleerd kunnen worden, kunnen bepaalde spelers in dit spel valsspelen. Dit betekent dat de signalen die de prijs bepalen verstoord worden, waardoor de prijs gemanipuleerd kan worden. Deze vorm van manipulatie komt op diverse manieren voor in de huidige markt, waarbij monopolies hun marktpositie misbruiken om zelf de prijs te kunnen bepalen. Dit is echter alleen mogelijk in een onvrije markt. Grote bedrijven lobbyen vaak voor overheidsinvloed om voordelen te verkrijgen ten opzichte van concurrentie, zoals belastingkortingen, subsidies, regulering, patenten of contracten met de overheid.

Een andere vorm van schadelijke en obscure overheidsinvloed, naast de eerder genoemde, is geld. Geld fungeert als de meetlat voor vraag en aanbod, en het is essentieel voor effectieve handel. Maar kunnen we erop vertrouwen dat het signaal dat geld ons geeft een rationeel beeld is? Onze overheid bewaakt het monopolie op ons geld, de euro, en banken en centrale banken kunnen dit fundament van onze samenleving manipuleren. Geld wordt ondersteund door arbeid, aangezien zo goed als alle mensen arbeid leveren voor hun inkomsten. Echter kan een kleine groep mensen eenvoudigweg op een knop drukken om ditzelfde geld te creëren, wat hen enorme macht geeft over alle facetten van de samenleving. Inflatie en de cyclus van economische boom en bust verstoren de signalen van vraag en aanbod, wat leidt tot bekende fenomenen zoals bubbels.

Het is duidelijk dat er geen sprake kan zijn van een vrije markt zolang we onderhevig zijn aan de manipulatie van een centrale bank. Dit zijn slechts enkele voorbeelden; men kan hele boeken vullen over dit onderwerp. Saifedean Ammous, als expert op dit gebied, slaagt erin om diepgaand alle facetten van deze economische filosofie toegankelijk en rationeel te presenteren.

Mijn laatste oproep is dan ook eenvoudig: gebruik dit boek als een wapen. Zorg ervoor dat de volgende generaties zich bewust worden van het systeem waarin we leven, zodat we kunnen bouwen aan een menselijke toekomst met een ware vrije markt.


\chapter{Inleiding}
De meerderheid van de leerboeken die vandaag op universiteiten worden gebruikt, zijn geworteld in de gangbare, maar vaak verwarrende Keynesiaans-Samuel\-soniaanse economische traditie. Ik heb jarenlang lesgegeven met deze academische teksten en heb gezien hoe talloze intelligente studenten de klas verlieten met meer vragen dan antwoorden. Ze worstelden om het praktisch nut in te zien van de complexe formules die ze moesten leren of om de resultaten ervan te vertrouwen op basis van logische gronden. In de loop der tijd heb ik met tientallen zeer intelligente studenten en afgestudeerden gesproken die een soortgelijke ervaring hadden. Ze deden wat nodig was om de gewenste cijfers te behalen, maar begrepen de essentie van wat ze leerden niet. Ze dwongen zichzelf om de benodigde mentale sprongen te maken en de logica achter de schijnbaar irrelevante formules te begrijpen, om daarna nooit meer terug te denken aan de concepten die tijdens de cursus aan bod kwamen. Wanneer studenten leren uit een algemeen aanvaard leerboek, maken ze kennis met theoretische modellen die vaak slechts een losse verbinding hebben met de werkelijkheid. Het succesvol afronden van de cursussen is gebaseerd op het begrijpen van deze modellen, niet noodzakelijk van de realiteit zelf.

In mijn colleges integreerde ik inzichten uit de Oostenrijkse School van economie, die door studenten consequent als de meest praktische en intellectueel boeiende segmenten van de cursus werden ervaren. Deze delen boden hen waarde die verder reikte dan het behalen van hun diploma. Tegenwoordig worden Oostenrijkse ideeën op de meeste universiteiten grotendeels genegeerd. Hedendaagse leerboeken maken zelden gewag van de Oostenrijkse School, laat staan dat ze hun concepten in detail behandelen. Ik was genoodzaakt om continu naar verschillende bronnen te grijpen voor informatie over deze onderwerpen. De voornaamste Oostenrijkse teksten, zoals Mises' \textit{Human Action} en Rothbards \textit{Man, Economy, and State}, blijken vaak moeilijk te verteren voor de moderne lezer. Helaas besteden deze werken veel tijd aan het bekritiseren van de heersende economische opvattingen van toen, wat uiteindelijk het duidelijk uiteenzetten van het Oostenrijkse perspectief in de weg staat.

Ik heb altijd naar een heldere, beknopte en toegankelijke samenvatting verlangd van de voornaamste economische concepten uit de Oostenrijkse traditie, die zou leiden tot inzicht in het belang van de uitgebreide monetaire marktordening binnen een beschaving. Deze wens leidde tot het ontwikkelen van de contouren van een dergelijk leerboek voor master- en seniorcolleges die ik gaf aan de Libanese American University. Na de publicatie van `De Bitcoin Standaard' en het ontdekken van een enthousiaste doelgroep die mijn economische inzichten waardeerde, besloot ik om mijn focus uitgebreid te verleggen naar het schrijven van het leerboek waaruit ik altijd al had willen lesgeven. In 2019 nam ik de beslissing om mijn universitaire positie op te geven en over te stappen naar zelfstandig doceren en publiceren via mijn website saifedean.com. Gedurende 2019 en 2020 ontwikkelde ik twee Principles of Economics-cursussen, ECO11 en ECO12, die de ideeën verder vormgaven die uiteindelijk de basis zouden vormen voor dit boek.

Door interactie te hebben met honderden studenten van over de hele wereld, en door me los te maken van de steeds verouderende en elitaire academische tijdschriften en uitgevers, kon ik mijn aandacht richten op het schrijven voor de lezer, in plaats van voor academische commissies. Na twee decennia van studie in de universitaire economie, vertegenwoordigt dit boek de inzichten die ik op mijn 17e had willen bezitten. Ik hoop dat mijn kinderen dit boek kunnen lezen wanneer ze interesse krijgen in economie.

Dit boek vormt een introductie tot de grondbeginselen van de economie en de economische denkwijze – voor iedereen een krachtig en nuttig hulpmiddel voor persoonlijke inzichten. Op een universiteit zou ik dit boek gedurende twee semesters onderwijzen om studenten zo een breed beeld te geven van het onderwerp economie en de economische denkwijze. Dit is meer dan alleen een universitair leerboek, het is geschreven voor een algemeen publiek dat geïnteresseerd is in economische ideeën. Zelfs als je geen economie studeert aan een universiteit, neem je elke dag van je leven economische beslissingen. Ik hoop dat dit boek een beknopte en toepasbare samenvatting biedt van de meest nuttige inzichten van de economische manier van denken. Een manier van denken die nuttig kan zijn bij persoonlijke en zakelijke besluitvorming.

`Grondbeginselen van de Economie' hanteert een benadering vanuit de Oostenrijkse school en maakt gebruik van helder en eenvoudige taal om uit te leggen wat door veel economen door de geschiedenis heen is gezien als de meest effectieve methoden om economische fenomenen te doorgronden. Het toepassen van de benadering van menselijk handelen staat centraal bij het verklaren van de kernconcepten en thema's binnen de economie, voortbouwend op de inzichten van economen van de Oostenrijkse School. Het boek richt zich op het in een logische volgorde behandelen van essentiële economische concepten en thema's, met het doel om lezers zowel een individueel als maatschappelijk economisch inzicht te bieden, inclusief de diverse implicaties van economie als vakgebied. Het eerste deel introduceert de fundamentele economische concepten en de methodologie van de Oostenrijkse School. Het tweede deel, `Economie', belicht de economische beslissingen van individuen. Het derde deel, `De Marktorde', verkent economisch handelen binnen een sociale context, de ontwikkeling van de kapitalistische economie, en de functie van geld. Het vierde deel, `Monetaire Economie', behandelt thema's als tijd, rente, en de monetaire en financiële economie. Het vijfde en laatste deel, `Beschaving', duikt in de economie van geweld en veiligheid en de betekenis ervan voor de toekomstige ontwikkeling van de menselijke beschaving.

Elk hoofdstuk in dit boek behandelt een essentieel economisch concept en is opgevat als een zelfstandig essay. Tegelijkertijd is het boek zodanig opgebouwd dat deze concepten in een logische reeks worden aangeboden. In het openingshoofdstuk wordt de methodologische aanpak van de economie volgens de Oostenrijkse School geïntroduceerd, inclusief een voorbeeld en een vergelijking met de methodologie van de natuurwetenschappen. Het tweede hoofdstuk werpt licht op het fundamentele concept van waarde, gebaseerd op het werk van Carl Menger, de grondlegger van de Oostenrijkse School. Het behandelt de subjectieve natuur van waarde, evenals de concepten van nut en marginale analyse. In het derde hoofdstuk wordt het belang van tijd binnen de economie benadrukt, de unieke kenmerken ervan bij de behandeling als economisch goed, en hoe economische activiteiten gezien kunnen worden als inspanningen om zowel de hoeveelheid als de subjectieve waarde van onze tijd op aarde te vergroten. Dit hoofdstuk introduceert eveneens de belangrijke begrippen van opportuniteitskosten en tijdvoorkeur.

Het tweede gedeelte van het boek introduceert de belangrijkste handelingen die mensen verrichten om op individueel vlak economische keuzes te maken. In elk van de hoofdstukken van dit gedeelte wordt een essentieel concept geïntroduceerd en geanalyseerd op basis van de redenen waarom mensen zich er mee bezig houden, het probleem dat het oplost en hoe het mensen helpt om economisch om te gaan met hun tijd. Het eerste en meest basale concept is arbeid, wat wordt behandeld in Hoofdstuk 4. Hoofdstuk 5 legt de economie van eigendom uit, waarom het ontstaat, het probleem dat het oplost en het concept van zelfbeschikking. Hoofdstuk 6 introduceert een specifiek type eigendom, kapitaal, dat bestaat uit goederen die worden gebruikt voor de productie van andere goederen. Ook de kosten van kapitaal, de productiviteit ervan en het verband met tijdsvoorkeur worden besproken.

Hoofdstuk 7 behandelt technologie als een economisch concept, legt uit waarom het de arbeidsproductiviteit verhoogt en waarom het beschouwd wordt als een immaterieel economisch goed dat niet aan schaarste onderhevig is. Dit hoofdstuk wordt afgerond met een analyse van het concept intellectueel eigendom, en de manier waarop de niet-schaarse aard van informatie afwijkt van andere productiefactoren. Energie, het thema van Hoofdstuk 8, wordt zelden besproken in traditionele economische leerboeken. Ik ben echter van mening dat een goed begrip van de energiemarkt cruciaal is voor een alomvattend begrip van de economie, vooral omdat de huidige kapitaalintensieve en technologisch geavanceerde markteconomieën niet mogelijk zouden zijn zonder significante toenames in het vermogen van de mens om in korte tijd grote hoeveelheden energie te benutten. Daarnaast is het toepassen van de Oostenrijkse benadering via marginale analyse fundamenteel om de huidige realiteit van energieproductie te begrijpen.

In het tweede deel van het boek wordt individueel economisch handelen onderzocht, terwijl het derde deel zich richt op economisch handelen binnen een sociale context. Dit gebeurt door de introductie van andere individuen in de analyse en het verkennen van de implicaties daarvan. Met de aanwezigheid van een ander persoon wordt ruil mogelijk, waarbij beide partijen worden gestimuleerd om deel te nemen omdat dit voor beiden voordelig is. Hoofdstuk 9 verklaart de logica van ruilhandel, de voordelen ervan, en de gevolgen van marktgroei waarin arbeidsdeling plaatsvindt. Hoofdstuk 10 introduceert het concept van geld, waarbij wordt uitgelegd welke problemen het oplost, hoe deze problemen de gewenste kenmerken van geld bepalen en hoe geld mensen helpt economisch te handelen en de waarde en productiviteit van hun tijd te vergroten. Het hoofdstuk legt uit dat geld een product van de markt is en niet van de staat, zoals vaak ten onrechte wordt onderwezen in economische leerboeken. Hoewel dit hoofdstuk geld introduceert, wordt de bredere discussie over monetaire economie overgelaten aan deel vier van het boek, zodat het de discussie over kapitaalmarkten kan volgen, een essentieel onderwerp in monetaire economie.
De sociale orde waarin individuen vreedzaam deelnemen aan alle eerder genoemde economische handelingen wordt een marktorde genoemd. Hoofdstuk 11 onderzoekt hoe individuele voorkeuren en economische handelingen leiden tot het ontstaan van prijzen, waarvan het essentiële belang binnen het marktproces wordt uitgelegd. Hoofdstuk 12 legt de term kapitalisme uit in de Misesiaanse traditie en hoe het een systeem van ondernemen is dat onlosmakelijk verbonden is met privaat eigendom en economische berekeningen. We onderzoeken Mises' krachtige test om te bepalen of een samenleving een markteconomie heeft en hoe het ons kan helpen de economische geschiedenis te begrijpen.

Het vierde deel van het boek, getiteld `Monetaire Economie' behandelt het concept van geld vanuit een Oostenrijks standpunt. Hoofdstuk 13 opent met het thema tijdsvoorkeur en de relatie hiervan met sparen, geld en kapitaalaccumulatie, welke op hun beurt kredietverlening en bankwezen mogelijk maken - dit zijn de onderwerpen van Hoofdstuk 14. Dit hoofdstuk gaat tevens in op rentetarieven, inclusief de discussie of deze uitgesloten kunnen worden. In Hoofdstuk 15 wordt het Oostenrijkse inzicht in de conjunctuurcyclus verkend door de fundamentele oorzaak te analyseren: monetaire expansie door het verstrekken van commercieel krediet.

In de voorgaande delen wordt de functie en structuur van een kapitalistische markteconomie belicht, alsook hoe deze enkel kan functioneren binnen een systeem dat privaat eigendom waardeert. Het vijfde en laatste deel van het boek, getiteld `Beschaving', toetst de levensvatbaarheid van de kapitalistische samenleving tegenover de dreiging van dwang en agressie. Hoofdstuk 16 behandelt de economie van geweld, zowel op persoonlijk als op overheidsniveau, terwijl Hoofdstuk 17 ingaat op de economie van defensie, en toont aan dat dit ook gewoon een marktproduct is dat tegenwoordig voornamelijk door de markt wordt aangeboden. Het afsluitende hoofdstuk bespreekt het concept van beschaving vanuit een economisch standpunt. Beschaving wordt voorgesteld als een orde die zich vormt wanneer een samenleving vreedzaam, productief, met een lage tijdvoorkeur, coöperatief, en innovatief genoeg is om een verbetering van de levensstandaard over generaties heen te garanderen. De kosten van deze enorme uitdaging worden onderzocht, alsook de overlevingskansen van de kapitalistische beschaving te midden van de grote bedreigingen waarmee zij geconfronteerd wordt.

Het boek wordt verder ondersteund door de website saifedean.com/poe, waar een volledige bibliografie beschikbaar is met links naar de lezingen die in het boek worden genoemd. Gezien de alomtegenwoordigheid van het internet, werd besloten de printversie van het boek te optimaliseren voor leesgemak door URL's uit de verwijzingen te verwijderen en een volledige bibliografie te bewaren op \href{saifedean.com/poe}{saifedean.com/poe}. Na het voltooien van dit boek, zal er ook een online cursus aangeboden worden op saifedean.com om de stof verder te verdiepen.

Dit boek heeft enorm baat gehad bij en is sterk verbeterd als gevolg van de feedback van Ross Stevens, Jeff Deist, Per Bylund, Conza, Allen Farrington, Jonathan Newman, Peter Young en Thomas Semaan. De laatste twee leverden ook uiterst waardevolle hulp bij het onderzoek tijdens het schrijven van dit boek. Ik bedank ook van harte de uitstekende redacteuren wiens grondige en nauwgezette redactie dit manuscript enorm heeft verbeterd: Alex McShane, Steve Robinson, Chay Allen, Renata Sielecki, Magda Wojcik, Evan Manning en Elizabeth Newton. Ik bedank ook Tamara Mikler voor het grafische ontwerp en Max DeMarco voor het redigeren van het audioboek. Ik ben ook erg dankbaar voor het team achter saifedean.com van Pavao Pahljina, Marko Pahljina, Dorian Antešić, Flora Fontes en Valentino Cnappi voor alle moeite die ze hebben gestoken in het runnen van de website en het regelen van de publicatie.

Dit boek zou niet mogelijk zijn geweest zonder de steun, aanmoediging en feedback van de leden van mijn online leerplatform saifedean.com. Ik ben hen zeer dankbaar dat ze me in staat hebben gesteld om productief te kunnen werken aan het afronden van mijn werk. In het bijzonder gaat mijn oprechte dank uit naar mijn lezers die de publicatie van dit boek hebben gesteund door de gesigneerde exemplaren in voorbestelling te kopen. Dank aan A Patel, Aaron Macy, Abdulla Al Abbas, Abdullah, Almoaiqel, Ágúst ragnar Pétursson, Aidan Campbell, AJ Garnerin, Alex, Alex Bowe, Alex Voss, Alistair Milne, Amit Barkan, Anderson Thees, Andrea, Bortolameazzi, Andrew Brasuell, Andrew Rosener, Andrew Stange, Anthony Clavero, Antonio Caccese, Ashok Atluri, Ben Johnson, Bertrand Marlier, BitcoinTina, Blake Canfield, BowserKingKoopa, brian daucher, Brian Kim, Brian Lockhart, Bronson Moyen, Browning Hi-Power 9mm, Bryan Matthieu, Bryan Wilson, Burcu Kocak, Carlo Barbara, Carlos Chida, Caspar Veltheim, Cedric Youngelman, Chase Oleson, Chen YH, Chris Cowlbeck, Christian Amadasun, Christof Mathys, Christopher Lamia, Christopher P Valle, Christopher Pogorzelski, Christopher To, Cletus Reynolds, Dale Williams, Dan Skeen, Dane Bunch, Daniel Ostermayer, Daniel Smith, Dave Hudson, David Heller, David Lawant, Dirk Seeber, Domingo Ochotorena, Dylan Parker, Ed Becker, Eduardo Lima, Edward Cosgrove, Ernest Huttel, Fabian von Schilcher, Federico Quintela, Francisco Reyes, Frank Acklin, Gary Lau, Gary Speed, Gen Shin, Glenn Thomas, Greg Doyle, Haris M, Harlan Robinson, Hayden Houser, Hugh Starr, Hunter Hastings, Jaap Willems, Jackson Forelli, Jaeger Hamilton, James Seibel, James Weaver, Jason DiLuzio, Jawad Barlas, Jerrold Randall, Jesse Powell, Jim Patterson, Joachim Boudet, John A. Krpan, John Brier, John Dixon, Jon E, Jonas Karlberg, Jonas Konstandin, Jonathan Camphin, Jonathas Carrijo, Jordan Wilby, Jose Areitio Arberas, José Niño, Jules, Julio Neira, Justin Schwartz, Keith G, Kelly Lannan, Kenneth Gestal, Kevin Coffin, Kim Butler, Lachie McWilliam, Larry Salibra, Leo Smith, Luis Alonso, Maksymilian Korzuchowski, Manuel Tomasi, Marco Daescher, Marcus Dent, Marius Kjærstad, Marius Reeder, Martin Brochhaus, Matija Grlj, Matt, Matthew Robin, Matthew Sellitto, Max Cash, Maximiliano Guimarães, Michael Atwood, Michael Culhane, Mike Clear, Mitch Soboleski, Mitchell Vanya, Nate Kershner, Nathan Smith, Neal Nagely, Nelson, Nicholas Sheahan, Nick Giambruno, Niko Laamanen, The Noded Podcast with Pierre Rochard and Michael Goldstein, Odi Kosmatos, Oleg Mikhalsky, Paweł Sławniak, Petar, Petr Zalud, Prince Filip Karađorđević, Raycheslav Karagyozov, Rene Bos, Richard Duke, Robert Koonce, Robin Dea, Ronald Zandstra, Rosie Featherby, Ross Stevens, Rowais Hanna, Ryan Nadeau, Ryan Sandford, Saagar Singh Sachdev, Sam Dib, Sam Shams, Samuel Douglass, Scott Manhart, Scott Schneider, Scott Shell, Seb Walker, Shakti Chauhan, Shaun McFarlane, Simonna Pencev, Stefano D’Amiano, Stephen Labb, Subhan Tariq, Tanner Dowdy, Thierry Thierry, Thomas Jenichen, Tom Karadza, Travis Tripodi, Trevor Smith, vik, Wendy Hiam, Wilfred Tannr Allard, Will Phillips, William Green, William Johnston, Wityanant Thongsawai, Yani Eberding, Yoism, Zachary Hollinshead, Zarak Ortega, Zsuzsanna Glasz.
