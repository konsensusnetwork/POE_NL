\hypertarget{technologie}{%
\chapter{Technologie}\label{technologie}}

\vspace{-1em}
\begin{blockquotebox}
    Het probleem van onze tijd ligt in de algemene onbekendheid met de invloed die het beleid van economische vrijheid heeft gehad op de technologische vooruitgang in de afgelopen tweehonderd jaar. Men maakt de foutieve aanname dat de verbeteringen in productiemethoden slechts toevallig samengingen met het laissez-fairebeleid.\footnotemark
    \par\raggedleft--- Ludwig von Mises\index{Ludwig von Mises}
\end{blockquotebox}
\footautocite{71}

\vspace{-0.5em}
\lettrine{V}oordat het proces van economische productie\index{productie} plaatsvindt in de echte wereld, plant de ondernemer het in zijn geest. Menselijk verstand stelt ons in staat om concepten en ideeën te ontwikkelen om economische resultaten te bereiken. Technologie kan gezien worden als het plan voor economisch\index{economisch} handelen, en als het mechanisme waarmee de mens zijn doelen bereikt. Technologie is vergelijkbaar met een recept voor het koken van een maaltijd; het is niet een fysiek deel van de maaltijd, maar de cognitieve kennis die alles bij elkaar brengt. Ideeën, recepten en technologie zijn vormen van kapitaal\index{kapitaal}, omdat ze de productiviteit van het productieproces\index{productieproces} verhogen. Ze zijn echter niet-materiële vormen van kapitaal\index{kapitaal}, wat ze overvloedig maakt. Als een persoon een technologie of een idee gebruikt, vermindert hij daarmee niet de mogelijkheid voor anderen om er gebruik van te maken, noch vermindert hij de productiviteit ervan. De implicaties van de niet-fysieke aard van deze vorm van kapitaal\index{kapitaal} zijn significant.

Het proces van technologische vooruitgang is de voortdurende ontwikkeling en toepassing van nieuwe en betere ideeën en methoden in het productieproces\index{productieproces}. Dit leidt tot een progressieve toename van de productie\index{productie} per tijdseenheid. Zonder technologische vooruitgang zal kapitaalaccumulatie snel te maken krijgen met afnemende opbrengsten. Als de visser een hengel begint te gebruiken, neemt zijn productie\index{productie} toe. Zonder technologische vooruitgang zou hij blijven investeren in meer hengels, tot op het punt dat hij geen behoefte meer had aan meer hengels, en de extra investering die hem enkel hengels opleverde die hij nooit nodig had om te gebruiken. Hij zou uiteraard op dat moment stoppen met investeren.

Maar als de visser in staat is om zijn verstand te gebruiken en nieuwe technologische ideeën te bedenken om kapitaal\index{kapitaal} mee te creëren, kan hij nieuwe kapitaalgoederen\index{kapitaalgoederen} produceren die productiever zijn dan de hengel. Het proces van kapitaalaccumulatie zal dan de productiviteit blijven verhogen zonder dat de marginale opbrengsten afnemen. De rede van de visser doet hem vermoeden dat het vissen productiever zal zijn als hij dit vanuit een boot kan doen in plaats van vanaf de kustlijn. Hij investeert zijn tijd en productie\index{productie} in het bouwen van de boot en probeert van alles uit. Zoals besproken in het vorige hoofdstuk, is deze investering duur en onzeker. Het vereist uitstel van consumptie\index{consumptie}, het is onderhevig aan waardevermindering en het kan mislukken. Maar als het lukt, zal zijn productiviteit toenemen. Het doorgaan met het investeren in meer van dezelfde boten zal ook leiden tot afnemende opbrengsten, maar het menselijk verstand zal blijven zoeken naar nieuwe technologieën. Bij elke nieuwe technologie en uitvinding duiken nieuwe beperkingen op voor de productie\index{productie} en kan kapitaal\index{kapitaal} worden ingezet om ze te verbeteren. Een betere, grotere, snellere en veiligere boot en nieuwe gespecialiseerde apparatuur kunnen blijven uitgevonden worden, zolang er kapitaal\index{kapitaal} wordt opgebouwd om ze te financieren. De nieuwe technologieën stellen je niet alleen in staat om meer vissen binnen te halen, maar stellen je ook in staat om soorten vissen te vangen die eerder niet te vangen waren.


\hypertarget{technologie-en-arbeid}{%
\section{Technologie en arbeid}\label{technologie-en-arbeid}}

\begin{blockquotebox}
    De vervanging van minder efficiënte productiemethoden door efficiëntere methoden leidt niet tot een overvloed aan arbeid, op voorwaarde dat er nog steeds materiële factoren beschikbaar zijn waarvan het gebruik de menselijke welvaart kan vergroten. Integendeel, het verhoogt de productie\index{productie} en daarmee de hoeveelheid consumentengoederen. ``Arbeidsbesparende'' apparaten verhogen het aanbod. Ze veroorzaken geen ``technologische werkloosheid''.\footnotemark
    \par\raggedleft--- Ludwig von Mises\index{Ludwig von Mises}
\end{blockquotebox} 
\footautocite{72}

De opkomst van de industrialisatie en het gebruik van grote hoeveelheden energie in de economische productie\index{productie} is gepaard gegaan met steeds terugkerend geklaag dat technologie arbeid vervangt. Intuïtief en oppervlakkig gezien lijkt dit logisch. Hoe meer machines worden gebruikt om de productie\index{productie} en productiviteit te verhogen, hoe minder producenten afhankelijk zijn van werknemers om hetzelfde productieniveau te genereren. Als fabrieken machines aanschaffen ontslaan ze overbodige werknemers. Misschien het meest bekende en originele voorbeeld van woede tegen de machine in reactie op de vrees voor verlies van banen kwam van de luddieten, die campagnes organiseerden om geautomatiseerde weefgetouwen kapot te maken, die volgens hen het levensonderhoud van de Britse textielarbeider zouden vernietigen. Gemechaniseerde landbouw zou boeren werkloos maken. De stoommachine zou grote delen van de beroepsbevolking overbodig maken. Telegrafisten waren nodig om telefoongesprekken te verbinden toen telefoons voor het eerst werden uitgevonden en gebruikt, maar toen geautomatiseerde schakelborden werden uitgevonden, stortte de vraag naar telegrafisten in. Recentelijk zetten veel fastfoodrestaurants steeds geavanceerdere geautomatiseerde kassa\textquotesingle s in die hun behoefte aan werknemers verminderen. Deze manier van denken is ook centraal in de marxistische leer, aangezien Marx beargumenteerde dat de winsten van mechanisatie zouden doorsijpelen naar de kapitalisten ten koste van de arbeiders, van wie het loon niet zou stijgen en die in aantallen zouden afnemen naarmate de roofzuchtige kapitalisten hen achterlieten in werkloosheid.

Hadden de luddieten gelijk? Zou verdere automatisering resulteren in de werkloosheid van grote delen van de bevolking, met vreselijke maatschappelijke consequenties tot gevolg? De overeenkomsten tussen hun geklaag en marxistische theorieën zijn een duidelijke waarschuwing voor het tegendeel. Bovendien ondersteunt empirische observatie de beweringen van de luddieten niet. Maar het definitieve antwoord kan alleen worden verkregen door een economische manier van denken.

Na ruim twee eeuwen van automatisering en industrialisatie is de situatie nu zo dat het merendeel van de Britse volwassenen die willen werken, werk kunnen vinden met lonen die aanzienlijk hoger liggen dan die van de luddieten. Het klopt dat er weinig tot geen Britten zijn die dezelfde soort eenvoudige banen hebben als hun achttiende-eeuwse voorouders. Echter, ze hebben wel werk. Ondanks de groeiende bevolking van Groot-Brittannië, is het aantal banen ook toegenomen. De Britten van nu verdienen meer en werken onder veel betere omstandigheden dan hun voorouders in de achttiende eeuw. Als de voorspellingen van de luddieten en marxisten correct waren geweest, zou men kunnen verwachten dat na twee eeuwen van technologische vooruitgang vandaag de dag niemand meer werk zou hebben, laat staan beter werk.

De verwarring van de luddieten was dat ze arbeid als een consumptiegoed\index{consumptiegoed} behandelden, aangeschaft voor het nut dat het biedt, in plaats van een productiegoed, aangeschaft voor de productie\index{productie} van consumptiegoederen\index{consumptiegoed}. Een consumptiegoed\index{consumptiegoed} waarvoor een superieur alternatief gevonden kan worden, is niet langer gewild en kan zijn economische waarde verliezen, zoals gebeurde met typemachines na de uitvinding van computers. Maar de vraag naar een productiegoed is niet noodzakelijk afhankelijk van het nut ervan voor de koper; het is afhankelijk van het nut van het goed voor productie\index{productie}. Zelfs als een productiefactor in één productieproces\index{productieproces} vervangen zou worden, zou het nog steeds waardevol zijn als het in een ander productieproces\index{productieproces} gebruikt kon worden.

Arbeid is de minst specifieke factor voor productie\index{productie}, en kan worden ingezet in andere banen of industrieën. En arbeid, bestaande uit menselijke tijd, is ook het ultieme hulpmiddel, waarvan de schaarste\index{schaarste} de basis vormt voor de schaarste\index{schaarste} van alle andere middelen. Alles is gemaakt met de input van menselijke arbeid, en we leven in een wereld van schaarste\index{schaarste} waar er altijd een grote marginale vraag is naar meer goederen en diensten. Naarmate technologische vooruitgang de productiviteit van arbeid verhoogt, en daardoor de waarde van arbeid verhoogt, maakt het de productie\index{productie} van meer economische goederen mogelijk, waarmee het de schaarste\index{schaarste} verlicht. Het kan echter de schaarste\index{schaarste} niet opheffen, die in feite de schaarste\index{schaarste} van menselijke tijd zelf is. Zolang de mensen onvervulde behoeften hebben, zullen er manieren zijn om menselijke arbeid in te zetten om aan die behoeften te voldoen. Hoeveel de menselijke productiviteit ook toeneemt, de menselijke behoeften kunnen altijd meer stijgen, en het menselijk verstand kan betere oplossingen voor de problemen van schaarste\index{schaarste} blijven bedenken. Er kunnen altijd betere producten, technologieën en veiligere productiemethoden worden bedacht, wat nieuwe vraag aanwakkert. We zullen nooit ``banen tekortkomen'', omdat we altijd meer mensen kunnen gebruiken om meer schaarse producten te maken om aan de altijd toenemende behoeften van andere mensen te voldoen. Schaarste kan nooit worden afgeschaft, omdat tijd altijd schaars is. Arbeid kan nooit stoppen, en de mens kan alleen kiezen aan welke taken hij prioriteit geeft. Hoe meer taken hij aan machines kan delegeren, hoe meer tijd hij heeft om vele van het oneindige aantal taken uit te voeren dat hij graag zou willen maar niet kan uitvoeren vanwege de schaarste\index{schaarste} van zijn tijd.

Er was een tijd dat het verplaatsen van mensen of bagage alleen kon worden bereikt door andere mensen in te huren om ze te dragen. Een sterke, gezonde man kon een andere man of enkele tientallen kilogrammen aan gewicht dragen, en ze enkele kilometers op een dag verplaatsen. Het werk van het dragen van zware dingen zonder de steun van kapitaal\index{kapitaal} had een zeer lage productiviteit. Het was zo onaangenaam om te doen dat het voornamelijk voor slaven was weggelegd. Alleen degenen die slaven konden bezitten, konden zich dit soort arbeid regelmatig veroorloven. De overgrote meerderheid van de bevolking kon echter alleen hun eigen lichaam en spullen verplaatsen, zo ver en snel als hun eigen voeten ze konden dragen.

Toen mensen het wiel ontwikkelden, breidden de mogelijkheden om zware dingen te verplaatsen zich uit. Door een kar met wielen te trekken, kon de arbeider nu meer gewicht over langere afstanden verplaatsen; met andere woorden, zijn productiviteit nam toe. De productiviteit van de arbeider zou nog verder toenemen door de kar te combineren met een paard. Met de komst van de Industriële Revolutie, en de uitvinding van de trein, auto, vrachtwagen, zeecontainer, en vliegtuig, overtrof de productiviteit van modern vervoer ver het niveau van voor de industrialisatie. Eén vrachtwagenchauffeur kan nu tot 50.000 kilogram aan gewicht verplaatsen met een snelheid van 100 km/u voor 16 uur per dag. Een handvol bemanningsleden kan een Airbus A380 van 575 ton, waarvan 300 ton vracht, met een snelheid van 903 km/u laten vliegen. Met een bemanning van 20 tot 40 mensen kan het grootste containerschip ter wereld, de HMM Algeciras, 24.000 containers van elk tot 25.400 kilogram, met een totaalgewicht van ongeveer 672.000 ton met een snelheid van 15.2 knopen of 28 km/u verplaatsen.

Vanaf het temmen van het paard tot de bouw van de HMM Algeciras, zijn er opeenvolgende uitvindingen geweest -- het wiel, de wagen, koetsen, vrachtwagens, treinen en vliegtuigen -- en toch zijn banen in de transportsector nog niet verdwenen. Integendeel, er is vandaag de dag zelfs een groter percentage voltijdse banen in de transportsector dan voordat het wiel werd uitgevonden. In primitieve samenlevingen die het wiel nog niet kenden, kon het niveau van specialisatie niet bestaan dat een groot aantal verschillende carrières in het transport mogelijk zou hebben gemaakt, omdat iedereen het merendeel van hun werktijd moest besteden aan het voorzien in hun eigen basisbehoeften. Met lage kapitaalniveaus, een beperkt gebruik van niet-menselijke energiebronnen en primitieve technologische ontwikkeling, was de arbeidsproductiviteit bijna gelijk aan het niveau nodig om te kunnen overleven. In zo\textquotesingle n wereld moeten de meeste mensen werken aan het produceren van hun eigen voedsel, en heel weinig mensen kunnen zich specialiseren in andere banen. Gezien de zeer lage productiviteit van transporttechnieken voor het wiel was uitgevonden, is het onwaarschijnlijk dat veel mensen een overschot hadden aan economische productie\index{productie} om iemand fulltime in te huren voor transport, omdat de opportuniteitskosten van die persoon een significant deel van het voedsel zouden vertegenwoordigen dat ze anders voor zichzelf zouden produceren. Alleen iemand die tot slaaf was gemaakt en geen vrije wil had, zou gedwongen worden tot dit soort werk.

Naarmate de technologie vordert en productiviteit toeneemt, overtreft de productie\index{productie} van elk individu boven hun dagelijkse behoeften om te kunnen overleven. Er ontstaat dan ruimte voor specialisatie, omdat meer arbeiders gevoed kunnen worden door de inspanningen van anderen. Dit bevrijdt hen van de noodzaak om zich bezig te houden met de arbeid om puur te overleven, en het stelt hen in staat om complexere goederen te produceren. Door de toegenomen productiviteit in de transportsector, werd het haalbaar voor mensen om vrijwillig in transport te gaan werken. Met de verdere verbetering van technologie en productiviteit, verbeterden de omstandigheden en lonen voor banen in transport ook voortdurend.

Veel mensen vinden steeds meer werk in transport naarmate de productiviteit van transport toeneemt. In plaats van één werknemer die één persoon draagt, hebben we nu één werknemer die een schip vaart dat duizenden mensen vervoert of een vliegtuig dat honderden mensen vervoert. De hoeveelheid werk dat gedaan wordt, neemt evenredig toe met de toename in productiviteit. Meer mensen reizen, er wordt meer werk gedaan, er vindt meer handel plaats en er worden meer behoeften vervuld. Hoe meer kapitaal\index{kapitaal} er in transport wordt geïnvesteerd, hoe productiever transportmedewerkers worden, en hoe meer ze betaald krijgen.

Voor luddieten en marxisten zou de uitvinding van het wiel overkomen als een absolute ramp -- denk maar aan alle verloren banen in de industrie voor het dragen van rugbrekend zware dingen! Maar in werkelijkheid was het een zegen voor de mensheid, omdat het mensen ontlastte van het dragen van zware lasten en hen in staat stelde om zich te concentreren op meer productieve banen.

De waarde van goederen komt, zoals besproken in Hoofdstuk 1, voort uit hun eigenschap om menselijke behoeften te vervullen. De menselijke behoefte aan beweging en transport kan niet worden geëlimineerd door efficiënter te worden. Mensen zijn mobiel en blijven niet graag lang op dezelfde plek. Afnemende opbrengsten zijn een resultaat van het verblijven op dezelfde plek en individuen willen zich verplaatsen. Handel vereist het verplaatsen van goederen en hoe groter de reikwijdte voor handel, hoe groter productiviteitsverbeteringen kunnen worden behaald. Deze economische realiteiten maken transport een behoefte die in alle tijden en plaatsen heeft bestaan, en men heeft geen reden om te verwachten dat ze binnenkort zal worden geëlimineerd. Elke baan in transport vertegenwoordigt op elk moment de meest productieve en technologisch geavanceerde oplossing voor het probleem van transport tot op dat moment. Wanneer een nieuwe technologie wordt uitgevonden, elimineert deze de behoefte aan transport niet; het stelt arbeid in staat om te worden gericht op een meer productieve oplossing voor transport.

Het is dus geen toeval dat de economische omstandigheden van de mensheid blijven verbeteren met technologische vooruitgang. Hoe productiever onze technologie is, hoe beter het met ons gaat. Als de mensheid zou luisteren naar de luddieten en technologische vooruitgang zou bestrijden, zouden we allemaal geen tijd hebben om iets van de enorm productieve dingen te doen die we in de moderne samenleving van vandaag doen. We zouden te druk bezig zijn met de meest primitieve taken, zoals het dragen van zware lasten, om iets anders te kunnen doen.

Het slechte nieuws voor luddieten is dat hun tegenstander veel sterker is dan ze beseffen. Ze staan niet tegenover hebzuchtige kapitalisten die proberen arbeiders te bedriegen; ze staan tegenover de volle kracht van economische werkelijkheid en het menselijke handelen in reactie op economische prikkels. De waarde die voor de mensheid voortvloeit uit nieuwe uitvindingen die onze productiviteit verhogen, is veel te groot en verleidelijk om door wetgeving en machinebrekers te worden overwonnen. De luddieten zijn altijd voorbestemd om te verliezen van wie technologie waardeert, omdat de gebruikers ervan het kunnen gebruiken om een veel hogere productiviteit te bereiken.

Hoewel de luddieten van de vroege negentiende eeuw erin slaagden om veel machines en enkele fabrieken te vernietigen, waren hun overwinningen tegen menselijke vooruitgang onbeduidend. Hun campagne stierf uit en hun ideeën werden belachelijk gemaakt, terwijl de technologische vooruitgang het leven voor iedereen bleef verbeteren. Ze konden niet voorkomen dat de vindingrijkheid van miljarden mensen het leven voor ons allemaal beter maakte. Zodra een wiel, weefgetouw, auto, vliegtuig, of softwarecode is uitgevonden, erkennen mensen de waarde die het biedt in termen van verhoogde productiviteit. Strenge beperkingen kunnen erin slagen om deze technologieën te vertragen, maar ze verhogen ook de opbrengsten voor degenen die er succesvol omheen werken. De individuen, bedrijven, of regio\textquotesingle s die een productieve technologie benutten die elders niet wordt gebruikt, kunnen tegen lagere prijzen produceren.

Technologische vooruitgang schaft de vraag naar arbeid niet af. Er is echter overtuigend bewijs dat het wel slavernij afschaft. \autocite{73} Naarmate specialisatie en productiviteit toenemen samen met de accumulatie van kapitaal\index{kapitaal}, wordt de productie\index{productie} van een werknemer steeds waardevoller, waardoor hij een waardevollere beloning voor zijn arbeid kan krijgen. De markt buit werknemers niet uit, maar stelt hen juist in staat om met de hoogste productiviteit te produceren. Dit maakt hen waardevoller voor degenen die hen in dienst hebben en vermindert het rendement op het tot slaaf maken. Naarmate de productiviteit van werknemers toeneemt, groeien de voordelen van wederzijdse samenwerking.

Slavernij en zeer productieve kapitaalgoederen\index{kapitaalgoederen} gaan niet samen. Het gebruik van zeer productieve kapitaalgoederen\index{kapitaalgoederen} maakt de bereidwillige medewerking van de werknemer steeds waardevoller, omdat ze zeer dure apparatuur, die vele malen meer waard is dan het loon dat ze betaald krijgen, vrijwillig of door nalatigheid kunnen saboteren. Tenzij hij genoeg betaald krijgt om vrijwillig te werken, brengt het dwingen van een slaaf om dure kapitaalgoederen\index{kapitaalgoederen} te beheren een groot risico\index{risico} met zich mee. Op deze manier moedigt het kapitalisme\index{kapitalisme} de opkomst van meer wederzijds voordelige handel aan ten koste van dwangmatige regelingen zoals slavernij. \autocite{74}

Kapitaalaccumulatie en de arbeidsdeling\index{arbeidsdeling} hebben ook geleid tot de ontwikkeling van geavanceerde energiebronnen. Dit maakt het mogelijk om steeds grotere hoeveelheden energie in te zetten om aan onze behoeften te voldoen. Zoals in het volgende hoofdstuk zal worden besproken, was het menselijk energieverbruik voor de inzet van moderne, kapitaalintensieve energiesystemen gebaseerd op koolwaterstoffen, wat zeer dicht bij de menselijke energieproductie lag. In een pre-kapitalistische wereld werd de meeste energie geproduceerd door de eigen handen en benen van een persoon. In zo\textquotesingle n wereld is de dienst van een ander mens erg waardevol. Met zeer weinig energie om aan de behoeften van een persoon te voldoen, heeft de energieproductie van een tweede persoon een enorme marginale waarde, wat slavernij economisch\index{economisch} aantrekkelijk maakt en slaven waardevol. Maar naarmate het energieverbruik toeneemt met nieuwe technologieën, tot het punt waarop de gemiddelde burger van een rijk land nu zoveel energie verbruikt als de productie\index{productie} van 200 slaven, kan het meeste werk dat door slaven werd gedaan nu aan machines worden uitbesteed. Deze machines zijn veel productiever, betrouwbaarder en nauwkeuriger. Met honderden mechanische slaven die energie leveren, wordt de marginale waarde van één extra menselijke slaaf steeds lager. Naarmate we meer machines hebben, wordt de economische logica van slavernij steeds minder overtuigend. Het is niet overdreven om te zeggen dat technologische innovatie en kapitaalaccumulatie de slavernij overbodig hebben gemaakt en slaven hebben bevrijd.

Toen er weinig of geen kapitaal\index{kapitaal} was, was transport een baan die alleen acceptabel was voor slaven. Toen er koetsen kwamen, waren er vrije mensen die bereid waren om een baan in het transport te accepteren, omdat de productiviteit hoog genoeg was om hen voldoende te compenseren voor hun tijd. Dit stelde hen in staat om voldoende voedsel te kopen van anderen die gespecialiseerd waren in de productie\index{productie} van voedsel. Met de introductie van de auto werd het werk van een taxichauffeur of vrachtwagenchauffeur nog beter beloond en werd werken als chauffeur een aantrekkelijk beroep voor miljoenen mensen over de hele wereld. Hoe meer de technologie ontwikkelt, hoe meer kapitaal\index{kapitaal} er in een baan wordt geïnvesteerd, hoe productiever de baan wordt en hoe lonender het werk is. Vandaag de dag werken veel hooggeschoolde ingenieurs, technici en verschillende andere professionals in de scheepvaart- en transportindustrie en hun productiviteit is hoog, waardoor ze een hoge levensstandaard hebben.

\hypertarget{technologie-en-productiviteit}{%
\section{Technologie en productiviteit}\label{technologie-en-productiviteit}}

\begin{blockquotebox}
    We hebben van onze voorouders niet alleen een voorraad aan producten geërfd, die de bron van onze materiële rijkdom zijn; we hebben ook ideeën en gedachten, theorieën en technologieën geërfd waaraan ons denken zijn productiviteit dankt.\footnotemark
    \par\raggedleft---Ludwig von Mises\index{Ludwig von Mises}
\end{blockquotebox}
\footautocite{75}

Naarmate betere technologieën worden gebruikt, stijgen de productiviteit en levensstandaard. Maar de niet-schaarse aard van technologie maakt het uniek als een methode om de waarde van menselijke tijd te verhogen. Terwijl arbeid, eigendom, kapitaal\index{kapitaal}, energie en geld schaars zijn, zijn ideeën dat niet. Toen de uitvinder van het wiel dit begon te gebruiken, nam zijn productiviteit toe. Toen zijn buren hem kopieerden, konden ook zij hun productiviteit verhogen zonder de productiviteit van de uitvinder te verlagen. Naarmate mensen een uitvinding namaken, profiteren zij ervan en neemt ieders productiviteit toe. Naarmate meer mensen profiteren van de uitvinding van het wiel, is het waarschijnlijk dat zij er innovaties aan toevoegen, waardoor iedereen kan profiteren van de hogere productiviteit die dergelijke innovatie met zich meebrengt.

De niet-schaarse aard van technologie maakt het in grote lijnen de fundamentele drijfveer van economische groei op lange termijn. Werk is duur, want het kost ons vrije tijd. Naarmate ons inkomen groeit, kunnen we ons meer vrije tijd veroorloven. Kapitaal is ook duur, omdat het ten koste gaat van toenemend waardevolle consumptie\index{consumptie}. Zonder technologische vooruitgang heeft het onvermijdelijk te maken met afnemende rendementen. Er zijn namelijk maar een bepaald aantal hengels die je kunt gebruiken. Handel en specialisatie lopen tegen hun grenzen aan als ze niet samen gaan met technologische vooruitgang. Deze voortgang kent zelf geen grenzen en maakt onbeperkte stijgingen in economische productiviteit mogelijk. Na de uitvinding van het wiel konden we een breed scala aan technologieën hierop baseren. Deze openden vervolgens weer nieuwe mogelijkheden voor innovatie. Koetsen, trams, handkarren, auto\textquotesingle s, bussen, vrachtwagens, treinen en vliegtuigen werden met wielen ontwikkeld. Deze vervoersmiddelen, en ook het wiel zelf, zullen verbeterd blijven worden door gebruikers en ingenieurs. Alleen de verbeteringen die de productiviteit verhogen worden overgenomen, terwijl degenen die het niet verbeteren worden verworpen. Technologische verbetering creëert nieuwe, intensievere arbeidsdelingen, die de specialisatie vergroten en zorgen voor verhoogde productiviteit. \autocite{76} Zolang mensen economiseren, zullen ze hun verstand blijven inzetten om betere oplossingen voor hun problemen te vinden.

We kunnen enig bewijs zien voor het argument dat technologische innovatie de motor is voor groei op lange termijn met de empirische observatie dat grotere bevolkingsgroepen snellere economische groei meemaken dan kleinere groepen. Als economische groei een product zou zijn van de beschikbaarheid van hulpmiddelen, dan zouden we verwachten dat een kleinere bevolking een grotere hoeveelheid hulpmiddelen per hoofd van de bevolking zou hebben. Dit zou het mogelijk maken om de productiviteit en de levensstandaard sneller te verhogen dan in een dichtbevolkt gebied. Als alleen hulpmiddelen economisch\index{economisch} welzijn zouden bevorderen, zouden we verwachten dat dunbevolkte gebieden hogere inkomens hebben dan dichterbevolkte gebieden. Maar als technologische innovatie de drijfveer is voor groei op lange termijn, dan kunnen we het tegenovergestelde verwachten. Grotere bevolkingsgroepen zouden leiden tot meer individuen die productieve ideeën bedenken. Aangezien deze ideeën niet concurrerend zijn, zouden ze zich verspreiden naar de hele bevolking, wat zou leiden tot hogere productiviteitsgroei. In een samenleving van 100 miljoen mensen zullen veel meer nieuwe ideeën zoals het wiel bedacht worden dan in een samenleving van 100 mensen. Stel je voor dat één op de 100 mensen elk jaar met een innovatief idee komt. De kleinere samenleving zou elk jaar één innovatie hebben om hun productiviteit te verbeteren, terwijl de grotere samenleving elk jaar 1.000.000 innovaties zou hebben. Aangezien deze niet concurrerend zijn, kan iedereen in de samenleving ze kopiëren en profiteren van de toegenomen productiviteit die ze met zich meebrengen.

De voorgaande bespreking is de essentie van een artikel van econoom Michael Kremer, die constateert dat bevolkingsgroeipercentages in de loop van de tijd positief correleren met de bevolkingsomvang. Als de drijfveer van economische groei de beschikbaarheid van fysieke hulpmiddelen zou zijn, dan zou je verwachten dat samenlevingen met een lagere bevolking sneller zouden kunnen groeien, omdat er per hoofd van de bevolking meer middelen beschikbaar zijn. Maar als de drijfveer van economische groei technologische vooruitgang zou zijn, dan zou je het tegenovergestelde verwachten: grotere samenlevingen produceren meer technologische ontdekkingen en behalen daardoor snellere economische en bevolkingsgroei.\autocite{77} In een andere test van dezelfde hypothese vergelijkt Kremer de bevolkingsdichtheid en economische groeipercentages in verschillende geografische regio\textquotesingle s die historisch gezien geïsoleerd waren. De data tonen aan dat de dichter bevolkte geografische gebieden een snellere economische groei hadden dan dunbevolkte gebieden, wat opnieuw het idee ondersteunt dat technologische innovatie en niet de economische hulpbronnen de economische groei stuwt. Een grotere bevolkingsdichtheid betekent dat er zich meer niet-concurrerende innovaties en technologieën zullen verspreiden naar de hele bevolking, wat de productiviteit verhoogt en de levensstandaard verbetert.

Een uniek aspect van ideeën en technologische innovaties is dat ze erg moeilijk kapot te maken zijn, in tegenstelling tot fysieke eigendommen en kapitaal\index{kapitaal}. Toen het wiel eenmaal was uitgevonden, zou het vernietigen van een specifiek wiel het idee van het wiel niet hebben vernietigd. Het idee zou voortleven in de gedachten van iedereen die het gezien had, en het kon onbeperkt opnieuw geproduceerd worden. Natuurlijke calamiteiten of door de mens veroorzaakte dingen zoals vandalisme en diefstal, hebben in de loop der millennia onmetelijk grote hoeveelheden kapitaal\index{kapitaal} vernietigd. Maar technologieën en ideeën zijn altijd veel moeilijker te vernietigen geweest. Ze blijven voortleven in de gedachten van mensen die ze hebben waargenomen, of in hun geschriften. En hoewel geschriften vernietigd kunnen worden, wat in de gedachten van mensen zit kan niet beperkt worden. Het is moeilijker om ideeën te doden dan om een persoon te doden of een object te vernietigen. Iemand kan gewelddadig aangevallen of gedood worden vanwege een idee of gedwongen worden dit af te wijzen onder fysieke marteling, maar je kunt hem niet beletten het te denken. Het laatste bastion van menselijke vrijheid zullen altijd de gedachten zijn die mensen in hun hoofd bewaren. Dit kan geen enkele kracht op de aarde overstemmen.

Zoals besproken in het vorige hoofdstuk, heeft fysiek kapitaal\index{kapitaal} ook last van het probleem van afnemende waarde, een onvermijdelijk gevolg van zijn fysieke aard. De kwaliteit van fysiek kapitaal\index{kapitaal} gaat constant achteruit, bovendien zijn er de risico\index{risico}\textquotesingle s van vernietiging waarmee het ook te maken krijgt. Niet alleen ontstaan materiële objecten uit ideeën, op lange termijn overleven ze ook alleen als ideeën omdat hun individuele fysieke manifestaties vergaan en worden vernietigd. De ideeën, technologieën en kennis die bijdragen aan het maken van bruggen, gebouwen, motoren, computers, wielen of medicijnen zijn allemaal economisch\index{economisch} gezien belangrijker dan welke individuele manifestatie van deze technologieën dan ook.

De introductie van de drukpers was een buitengewoon belangrijke technologie voor de mensheid, omdat het de massaproductie van ideeën mogelijk maakte. Het werd veel moeilijker om ideeën te vernietigen, aangezien ze zich verspreidden via een groeiend aantal kopieën. De uitvinding van digitale media en het internet was een ander hulpmiddel voor de capaciteit van de mensheid om ideeën en technologieën te bewaren. Het maakte kopiëren van informatie veel goedkoper. Een digitaal\index{digitaal} apparaat om makkelijk informatie op te slaan ter waarde van een paar dollar of het salaris van een paar uur werk voor de meeste mensen op de wereld, kan alle boeken van \textquotesingle s werelds grootste bibliotheek opslaan.

\hypertarget{technologische-innovatie-en-ondernemerschap}{%
\section{Technologische innovatie en ondernemerschap}\label{technologische-innovatie-en-ondernemerschap}}

Dit evolutionaire proces van selectie en variatie gaat eindeloos door met technologieën. Er zijn geen goede redenen om te verwachten dat dit zal stoppen. Het wordt uiteindelijk gedreven door de menselijke behoefte om te economiseren. Dit is een eeuwig probleem dat niet kan worden vermeden. Mensen zijn altijd aan het economiseren. Dat vereist het toepassen van het verstand om het productieproces\index{productieproces} te verbeteren. Technologische innovaties verhogen de productiviteit, maar ze stoppen het economiserend gedrag niet; mensen moeten nog steeds economiseren en zoeken naar manieren om hun productiviteit te verbeteren. De nieuwe innovatie opent simpelweg meer horizons om nieuwe innovaties te vinden.

Het overheersende model om technologische innovatie te begrijpen, is dat het een product is van wetenschappelijke vooruitgang die door wetenschappers wordt gerealiseerd. Hoewel dit model begrijpelijkerwijs populair is bij de universiteiten die het doceren, toont een nadere blik op de realiteit van technologische innovatie een veel dynamischer en marktgestuurd proces aan. Technologische innovaties zijn alleen echte innovaties als ze slagen voor de markttest en de productiviteit verhogen en daardoor een marktprijs opleveren die hoog genoeg is om de producent te compenseren voor het gebruik ervan. Falen op de markt betekent dat de toename van de productiviteit van de technologie de initiële kosten niet rechtvaardigt. Het verschil tussen een nieuwigheidje of speelgoed en technologische innovatie ligt puur in het vermogen van laatstgenoemde om de productiviteit te verhogen.

In \emph{The Economic Laws of Scientific Research} geeft Terence Kealey een zeer overtuigend beeld van de onlosmakelijke band tussen markten\index{markten} en technologische innovatie. \autocite{78} Kealey verwerpt het lineaire model voor technologische vooruitgang, waarin academische wetenschappelijke bevindingen worden toegepast om technologische innovaties te produceren, en biedt een schat aan overtuigend bewijsmateriaal dat het tegendeel bewijst. De toename van de productiviteit in de textielindustrie in de achttiende eeuw kwam door de uitvindingen van vaklieden die niets aan academici te danken hadden. Britse groei in landbouwproductiviteit in de negentiende eeuw ontstond niet door overheidssteun voor landbouwonderzoek en -ontwikkeling, maar door boeren en uitvinders. Het belangrijkste is dat de Industriële Revolutie niet is voortgekomen uit de laboratoria van wetenschappers, maar uit de werkplaatsen van arbeiders, soms analfabeet. Thomas Newcomen, die de eerste commerciële stoommachine uitvond, was een nauwelijks geletterde smid uit de provincie die geen kennis had van welke wetenschappelijke vooruitgang dan ook, waarvan verondersteld werd dat deze de industriële motor zou hebben gestart. Na een decennium van experimenteren, leidde zijn werk met pompen hem ertoe het proces van een pomp om te keren en er een motor van te produceren. Terwijl de pomp mechanische kracht gebruikt om vloeistoffen te verplaatsen, gebruikt een motor bewegende vloeistoffen om mechanische kracht te produceren. Het was een eenvoudig idee, geïnspireerd door de enorme economische beloning voor het produceren van een motor, en niet door theoretische wetenschappelijke ontdekkingen. Kealey illustreert dit aan de hand van voorbeelden als James Watt, Richard Trevithick en George Stephenson, en andere pioniers op het gebied van motoren.

\begin{blockquotebox}
    Het zal daarom duidelijk zijn dat de ontwikkeling van de stoommachine, het artefact dat meer dan enig ander de Industriële Revolutie belichaamt, niets te danken had aan wetenschap; het kwam voort uit bestaande technologie, en het werd gecreëerd door ongeschoolde en vaak afgezonderde mannen die praktisch gezond verstand en intuïtie toepasten om de mechanische problemen waar ze last van hadden aan te pakken, en wiens oplossingen een duidelijke economische beloning zouden opleveren.
    \par\vspace{1em}\noindent
    Als we terugkijken op de Industriële Revolutie in het algemeen, is het moeilijk te zien hoe wetenschap überhaupt veel zou hebben kunnen bieden aan technologie. Dit komt doordat de wetenschap zelf zo rudimentair was. Scheikundigen die de flogistontheorie aanhingen, of die dachten dat warmte een substantie was, of die probeerden om een perpetuum mobile te bouwen, waren waarschijnlijk niet erg nuttig voor ingenieurs. Sterker nog, tijdens een groot deel van de negentiende eeuw was het tegenovergestelde waar; wetenschappers haastten zich om de ingenieurs bij te benen. De beschrijvingen van bijvoorbeeld Carnots wetten van de thermodynamica ontstonden uit zijn frustratie met Watt\textquotesingle s verbeterde stoommachine, omdat die stoommachine alle regels van de hedendaagse natuurkunde brak. Watt\textquotesingle s motor was efficiënter dan de theorie zei dat hij kon zijn, dus Carnot moest de theorie veranderen.\footnotemark
\end{blockquotebox}
\footautocite{79}

Het is nauwkeuriger om te zeggen dat de uitvinding van de stoommachine de thermodynamica heeft gecreëerd, dan andersom. Een vergelijkbaar verhaal kunnen we zien met de uitvinding van het vliegtuig. De meerderheid van de wetenschappers aan het begin van de twintigste eeuw was er stellig van overtuigd dat vliegen onmogelijk was, \autocite{80} zelfs nadat het was gebeurd. Toch waren het twee broers met fietsenwinkels zonder enige wetenschappelijke scholing die het voor elkaar kregen. Daarna werd de natuurkunde gerevolutioneerd om de vlucht te verklaren en te rationaliseren. Technologische innovatie ontstaat uit de wens om doelen te bereiken en winst te behalen door anderen te dienen.

Verder illustreert Kealey dat technologische vooruitgang tijdens de Industriële Revolutie plaatsvond in Groot-Brittannië. Dit land had amper overheidssteun voor wetenschap. In landen als Frankrijk, dat officiële wetenschap royaal financierde, gebeurde dit niet.

\hypertarget{software}{%
\section{Software}\label{software}}

Naarmate de menselijke kennis vorderde, hebben onze ideeën geleid tot de creatie van steeds complexere machines om de producten te produceren die we waarderen. Toen het bedienen van machines steeds repetitiever en voorspelbaarder werd, begonnen mensen manieren te bedenken om de instructies voor machines te automatiseren. Weefgetouwen voor het maken van stoffen werden uitgerust met leidende patronen en ponskaarten die betrouwbare patronen in stof konden produceren zonder dat er bewuste en voortdurende menselijke supervisie vereist was. Sommige mechanische apparaten werden gebruikt om wiskundige berekeningen uit te voeren met een snelheid en betrouwbaarheid die mensen niet konden bereiken.

In het jaar 1822 werkte de Engelse polymath en uitvinder Charles Babbage aan de ontwikkeling van een ``verschilmachine''. Deze machine werd gebruikt voor het berekenen van polynomiale functies. \autocite{81} Hij kon de bouw niet voltooien, hoewel zijn ontwerp bewaard is gebleven en in 1991 bouwde het London Science Museum een werkende machine op basis van zijn ontwerp. In 1833 begon Babbage te werken aan een algemener ontwerp, de Analytische Machine. Dit ontwerp zou veel van de essentiële kenmerken van de moderne computer in zich dragen, een eeuw voordat moderne computerfabrikanten een commercieel succes werden.

Misschien het meest fascinerende aspect van Babbage\textquotesingle s ontwerp was dat het programmeerbaar was met geponste kaarten. Ada Lovelace, de dochter van Lord Byron, ontwikkelde in 1842 een algoritme om een reeks Bernoulli-getallen te berekenen met Babbage\textquotesingle s machine. Dit geeft haar een sterke claim op de titel van \textquotesingle s werelds eerste programmeur. \autocite{82} Hoewel Babbage en Lovelace er niet in slaagden om commerciële computers te ontwikkelen, waren ze van cruciaal belang in het bevorderen van de wetenschap en kunst van computerontwikkeling totdat het vruchten afwierp in de twintigste eeuw. De Analytische Machine van Babbage was te moeilijk en te duur om in de negentiende eeuw succesvol te bouwen en commercieel te gebruiken, gezien de industriële en technologische realiteit van die tijd; maar in de twintigste eeuw is het wel mogelijk geworden.

Elektriciteit zou een rol gaan spelen in de werking van deze machines, waardoor hun productiviteit en complexiteit toenam. Er waren zeer geavanceerde bedradingsschema\textquotesingle s en circuits nodig om ze te besturen. Omdat deze geavanceerde nieuwe elektrische machines moeilijke wiskundige problemen konden berekenen, werden ze ``computers'' genoemd. In 1941 bouwde de Duitse ingenieur Konrad Zuse wat nu wordt beschouwd als de eerste programmeerbare computer, de Z3.\autocite{83}

De instructies die de vroege computermachines bedienden, werden er in gecodeerd via elektrische circuits of ponskaarten. Om een vroege computer iets anders te laten doen, waren doorgaans aanpassingen aan de hardware en processen nodig, evenals ingewikkelde herbedrading. Tegen het einde van de jaren `40 werd het mogelijk om deze instructies elektronisch op te slaan in computers met de ENIAC (Electronic Numerical Integrator and Computer). In de jaren `50 en `60 werden computertalen ontwikkeld waarmee programma\textquotesingle s op een meer abstracte manier konden worden gespecificeerd, los van de architectuur van de computer. De ontwikkeling van deze gestandaardiseerde programmeertalen, en het groeiend aantal mensen wereldwijd dat ze kon lezen, begrijpen en schrijven, bracht een geheel nieuw type economisch\index{economisch} goed voort met enorme implicaties.

Software kan worden beschouwd als de puurste vorm van een technologisch goed. Het bestaat volledig uit informatie en heeft geen fysieke vorm, maar het verhoogt de productiviteit enorm. Het kan heel snel over de hele wereld worden gecommuniceerd met moderne communicatiemiddelen, en het is niet-exclusief en niet-schaars. Het gebruik van software in een industrieel proces maakt een verhoogde automatisering van de functies van de machines mogelijk, waardoor er minder menselijk toezicht en arbeid nodig is. Software maakt een veel betere organisatie van middelen en productieketens mogelijk, waardoor kosten worden verlaagd en efficiëntie wordt verhoogd.

Deze economische ontwikkeling heeft de afgelopen zeven decennia een opmerkelijke impact gehad op de wereld. Ideeën en technologieën kunnen nu worden gecodeerd via abstracte letters en nummers. Deze worden ingevoerd in hardware die de werking van een machine controleert en het in staat stelt steeds complexere taken uit te voeren. Voor het merendeel van de bevolking van negentiende-eeuws Groot-Brittannië moeten de ponskaarten die in obscure en zeer complexe machines werden gestopt, onbegrijpelijk onbeduidend hebben geleken. Vandaag de dag heeft software, de instructies die in standaardtalen zijn gecodeerd en machines opdracht geven functies uit te voeren, elke industrie ter wereld overgenomen. Het is onmogelijk om je een enkel gebied van economische productie\index{productie} voor te stellen dat zijn productiviteit niet heeft verhoogd door het gebruik van machines die op software draaien.

\hypertarget{eigendom-van-ideeuxebn}{%
\section{Eigendom van ideeën}\label{eigendom-van-ideeuxebn}}

Kunnen ideeën en technologie als eigendom worden beschouwd? Om deze vraag te beantwoorden, keren we terug naar de discussie in Hoofdstuk 2. Daar werd een onderscheid gemaakt tussen economische en niet-economische goederen. Beide types van goederen bieden nut voor individuen, maar economische goederen hebben waarde omdat ze schaars zijn. Schaarse goederen zijn goederen waarvan de voorraad dermate beperkt is dat het onmogelijk is om aan de vraag te voldoen. Deze schaarste\index{schaarste} dwingt mensen om keuzes te maken over hoe ze deze goederen consumeren en verdelen. Met andere woorden, schaarste\index{schaarste} dwingt mensen om waarde toe te kennen aan dingen. Ideeën zijn immaterieel en er is geen grens aan hun voorraad, dus het beschikbare aanbod kan altijd voldoen aan de vraag. Dit voorkomt de ontwikkeling van een marktwaarde voor ideeën, tenzij het individu dat het idee bezit een markt creëert door de toegang te beperken.

Er zijn twee manieren om schaarste\index{schaarste} te creëren in de toegang tot ideeën om een marktwaarde voor hen te genereren. De eerste is dat de persoon met de kennis ervoor kiest om deze niet openbaar te maken, en deze alleen te onthullen aan individuen die ervoor betalen. Handelsgeheimen, geheime recepten en iemands eigen technologische processen zijn voorbeelden van deze vrijwillige en vreedzame methode om eigendom te vestigen in technologie en ideeën. De tweede manier is om de kennis openbaar te maken, maar de dwangmaatregelen van de staat te gebruiken om te voorkomen dat anderen de kennis gebruiken om winst te maken. Voorbeelden hiervan zijn intellectuele eigendomswetten, zoals auteursrechten en patenten. Kinsella legt deze uit:

\begin{blockquotebox}
    Een patent is een toekenning door de staat die de patenthouder toestaat om het rechtssysteem van de staat te gebruiken om anderen te verbieden hun eigendom op bepaalde manieren te gebruiken -- door hun eigendom opnieuw in te richten volgens een bepaald patroon of ontwerp beschreven in het patent, of door hun eigendom (inclusief hun eigen lichamen) te gebruiken in een bepaalde reeks stappen beschreven in het patent.
    \par\vspace{1em}\noindent
    Auteursrechten hebben betrekking op \textquotesingle originele werken\textquotesingle, zoals boeken, artikelen, films en computerprogramma\textquotesingle s. Een auteursrecht is een concessie van de staat waarmee de houder van het auteursrecht kan voorkomen dat anderen hun eigendom -- bijvoorbeeld inkt en papier -- op bepaalde manieren gebruiken.
    \par\vspace{1em}\noindent
    In beide gevallen wijst de staat aan A het recht toe om het eigendom van B te controleren -- A kan B vertellen om bepaalde dingen niet te doen met zijn eigendom. Aangezien eigendom het recht op controle is, verleent intellectueel eigendom aan A mede-eigendom van B's eigendom.\footnotemark
\end{blockquotebox}
\footautocite{84}

Een uitstekende behandeling van dit onderwerp vanuit een juridisch en economisch\index{economisch} perspectief kan gevonden worden in Stephan Kinsella\textquotesingle s werk \emph{Against Intellectual Property}. \autocite{85} Een belangrijk inzicht is dat wanneer informatie en kennis over bepaalde productieprocessen openbaar worden, de enige manier om te voorkomen dat anderen deze gebruiken, is door beperkingen op te leggen aan de manieren waarop zij hun eigendom kunnen gebruiken. De enige manier om gepubliceerde informatie auteursrechtelijk te beschermen, is door het voor de eigenaren van het gepubliceerde werk illegaal te maken om hun eigen bezit van inkt en papier te gebruiken voor het opnieuw creëren van het auteursrechtelijk beschermde werk. Op dezelfde manier kunnen patenten alleen werken door beperkingen op te leggen, met de dreiging van overheidsgeweld, aan de mogelijkheden van producenten om hun eigen apparatuur op een vergelijkbare manier te gebruiken zoals beschreven in het patent.

Zowel patenten als auteursrechten vereisen het gebruik van dreiging tegen individuen die vreedzaam economische productie\index{productie} verrichten. In beide gevallen geeft de overheid\index{overheid} aan de rechthebbende van het auteursrecht of patent het recht om andermans bezit te controleren. Vanuit een juridisch oogpunt impliceren wetten over intellectueel eigendom de toewijzing van de aanspraak op het fysieke eigendom van anderen: de houder van het auteursrecht of patent eist controle over de eigendommen.

Zoals Wendy McElroy uitlegde in \emph{Contra Copyright, Again:}

\begin{blockquotebox}
    Mijn ideeën zijn als stapels geld die in een kluis zijn opgeborgen waar je niet bij kunt zonder in te breken en te stelen. Maar, als ik de kluis open gooi en mijn geld in de wind verspreid, zijn de mensen die het van de straat oprapen niet meer dieven dan de mensen die de woorden oppikken en gebruiken die ik in het publieke domein werp.\footnotemark
\end{blockquotebox}
\footautocite{86}

Hoofdstuk 4 gaat dieper in op hoe eigendom, volgens Menger, ``geen willekeurige uitvinding is, maar juist de enige praktisch mogelijke oplossing voor het probleem dat ons door de aard der dingen wordt opgelegd. Dit probleem komt voort uit het verschil tussen de behoeften aan, en beschikbare hoeveelheden van, alle economische goederen.'' \autocite{87} Begrip van de manier waarop menselijk handelen\index{menselijk handelen} resulteert in de ontwikkeling van het instituut van eigendom verklaart de willekeurige, onwerkbare en tegenstrijdige aard van het concept van intellectueel eigendom. Ideeën zijn niet zeldzaam, dus de vraag ernaar kan nooit hun aanbod overtreffen \emph{--} er is geen limiet aan hoeveel wielen kunnen worden geproduceerd vanuit het idee van het wiel. Het gebrek aan schaarste\index{schaarste} maakt de toepassing van het kader van eigendom ongeschikt voor ideeën, aangezien er geen conflict over schaarste\index{schaarste} te vermijden is. Dit maakt intellectueel eigendom onverenigbaar met eigendomsrechten.

Met de economische benadering van deze vragen, is het idee van intellectuele eigendomswetten intellectueel onverdedigbaar. Het wordt niet meer dan agressie van de instanties die deze wetten opleggen aan het eigendom van iedereen die er mogelijk mee in conflict komt. Het afschaffen van intellectuele eigendomswetten voorkomt niet dat producenten handelsgeheimen bewaren; het plaats de kosten van het geheimhouden ervan bij de producent en vereist dat hij alleen vreedzame methoden gebruikt om het af te dwingen. Ideeën hebben niets dat het opleggen van hun schaarste\index{schaarste} een aanvaardbare uitzondering maakt op het principe van non-agressie, dat in Hoofdstuk 16 gedetailleerder zal worden besproken. Zelfs als er een voordeel zou zijn voor een bepaald deel van de samenleving, of voor de samenleving als geheel, rechtvaardigt het niet het initiëren van agressie tegen vreedzame mensen.

Een grondiger onderzoek naar de vermeende voordelen van intellectueel eigendom onthult echter dat deze sterk overdreven zijn. Intellectuele eigendomswetten stimuleren steeds meer vernieuwers om monopolie-licenties te verkrijgen ten koste van innovatie om aan de vraag van consumenten te voldoen. Deze wetten vergroten de beloning van statelijke monopolie-licenties op ideeën. Dit leidt ertoe dat vernieuwers steeds meer middelen inzetten om dat doel te bereiken, in plaats van te proberen consumenten tevreden te stellen.

Dit is het meest zichtbaar in de farmaceutische en software-industrieën, waar grote bureaucratische bedrijven steeds meer gezien kunnen worden als enorme patenttrollen. Hun voornaamste doelstelling is het inhuren van advocaten, het verkrijgen van patenten, het aanklagen en zichzelf verdedigen tegen aanklachten. Ondertussen wordt het ontwikkelen van consumentensoftware en medicijnen steeds meer een secundaire focus.

Hoewel we leren om innovaties te waarderen omwille van de innovaties zelf, zijn waardevolle innovaties die waar consumenten genoeg waarde aan hechten om ze winstgevend te maken. Zonder intellectuele eigendomswetten is de enige manier om ideeën en innovaties te gelde te maken, dat houders van ideeën ervoor zorgen dat hun ideeën een grotere waarde hebben voor consumenten dan de beschikbare alternatieven.
 \autocite{88} Met intellectuele eigendomsrechten kunnen ondernemers hun concurrenten juridisch verbieden om te concurreren. Ze slagen door hun monopoliepositie over hun ideeën te behouden. Het voldoen aan de wensen van de consumenten wordt een secundaire zorg. Door het aantal aanbieders op de markt te beperken, komen de intellectuele eigendomsrechten in handen van de overheid\index{overheid}, ten koste van de tevredenheid van de consument.

Een veelvoorkomend argument van voorstanders van intellectuele eigendomsrechten is dat het voor een bepaalde periode belonen van innovators met monopoliewinsten hen zal aanzetten om meer te produceren dan ze anders zouden doen. De samenleving als geheel zou er beter aan doen deze vorm van agressie tegen vreedzame eigenaren toe te staan om vernieuwers te beschermen en hen aan te moedigen om nieuwe ideeën te bedenken. Maar de theoretische en empirische argumenten voor de grotere voordelen voor de samenleving van intellectuele eigendomswetten zijn erg zwak. In een uitstekende studie van het systeem van patenten en intellectuele eigendommen leveren Levine en Boldrin overtuigend bewijs dat intellectuele monopoliewetten innovatie tegengaan. De focus op patenten leidt de energie van bedrijven weg van innovatie naar rechtszaken en een wapenwedloop met patenten, waarbij concurrenten proberen zoveel mogelijk patenten te verwerven om als ruilmiddel\index{ruilmiddel} in rechtszaken te dienen en om elkaar met rechtszaken te dwarsbomen. De hoge kosten van de ontwikkeling van geneesmiddelen, meestal als rechtvaardiging voor monopoliewinsten aangehaald, komen voornamelijk voort uit de kosten van rechtszaken en de vereiste regelgevende goedkeuring voor geneesmiddelen en het verkrijgen van patenten.

Boldrin en Levine onderzoeken deze wetten en vinden weinig empirisch bewijs voor het idee dat intellectueel eigendom leidt tot meer innovatie of groei:

\begin{blockquotebox}
    Er is geen empirisch bewijs dat ze bijdragen aan het bevorderen van innovatie en productiviteit, tenzij productiviteit wordt geïdentificeerd met het aantal toegekende patenten -- dat, zoals bewijs aantoont, geen correlatie heeft met gemeten productiviteit. Deze disconnectie ligt aan de basis van wat het ``patentraadsel'' wordt genoemd: ondanks de enorme toename in het aantal patenten en in de sterkte van hun juridische bescherming, heeft de Amerikaanse economie noch een dramatische versnelling in het tempo van technologische vooruitgang, noch een grote toename in de uitgaven voor onderzoek en ontwikkeling gezien.
    \par\vspace{1em}\noindent
    In 1983 werden in de Verenigde Staten\index{Verenigde Staten} 59.715 patenten uitgegeven; in 2003 waren dat er 189.597; en in 2010 werden er 244.341 nieuwe patenten goedgekeurd. In minder dan 30 jaar is de stroom van patenten meer dan verviervoudigd. Daarentegen lieten innovatie en uitgaven aan onderzoek en ontwikkeling, of de productiviteitsgroei geen bijzondere stijging zien. Volgens het Bureau of Labor Statistics was de jaarlijkse groei van de totale factorproductiviteit in het decennium 1970 -- 1979 ongeveer 1,2 procent, terwijl het in de decennia 1990 -- 1999 en 2000 -- 2009 iets minder dan 1 procent was.\footnotemark
\end{blockquotebox}
\footautocite{89}

Het simplistische beeld van intellectuele monopolierechten is dat ze vernieuwers stimuleren. Maar bij nader inzien wordt duidelijk dat ze juist het tegenovergestelde effect hebben. Innovatie heeft altijd sterke motivatie nodig en wordt vergemakkelijkt door te bouwen op innovaties van andere mensen. Intellectuele monopoliewetten bieden niet zozeer een extra stimulans voor de vernieuwers, maar hinderen hen juist door hen te beletten voort te bouwen op het werk van anderen. De meeste uitvinders stuiten op hun uitvindingen terwijl ze hun eigen probleem proberen op te lossen, en de uitvinding zal hen waarde op zich bieden, ongeacht wat anderen ermee doen. Bovendien heeft de uitvinder een enorm voordeel als hij als eerste met een innovatie op de markt komt en deze kan verkopen zonder zijn toevlucht te hoeven nemen tot dwingende wetten op intellectueel eigendom. Door de eeuwen heen zijn de grootste uitvindingen, evenals de meest innovatieve werken van literatuur, muziek en kunst, geproduceerd zonder de noodzaak van auteursrechten of patenten. In feite kan men beargumenteren dat ze juist werden ontwikkeld vanwege de afwezigheid van auteursrechtwetten, waardoor hun makers goedkoop toegang hadden tot het werk van hen die hen inspireerden en de basis voor hun eigen creaties boden. Het is gebruikelijk voor voorstanders van intellectuele eigendomswetten om te focussen op de voordelen van grotere inkomsten voor de uitvinder, maar ze zijn zeer stil over het idee van de enorme kosten die dit met zich meebrengt voor het merendeel aan potentiële uitvinders die geen toegang hebben tot ideeën of hierop kunnen voortbouwen, zonder overdreven kosten te moeten betalen.

Een idee is het enige niet-schaarse productieve kapitaal\index{kapitaal}. Naarmate technologie en telecommunicatie goedkoper worden, wordt het kopiëren van productieve ideeën steeds eenvoudiger en goedkoper. Hoe goedkoper het wordt om goede ideeën te verspreiden en kopiëren, hoe productiever de wereld wordt. Intellectuele eigendomswetten leggen hogere kosten op aan de overdracht van ideeën. In de hedendaagse wereld profiteren vooral de mensen die werken op het gebied van intellectueel eigendom hiervan, maar niet de makers of producenten, en ook niet degenen die het kopiëren of de maatschappij in brede zin. ``Als ik verder heb gezien dan anderen, komt dat doordat ik op de schouders van reuzen stond'', was hoe Isaac Newton eer betoonde aan de vele mensen van wie hij geleerd had. In zijn tijd kostte het veel geld om de kennis van anderen te verkrijgen door dure manuscripten aan te schaffen. De drukpers, industrialisatie en het internet hebben de kosten van het verwerven van kennis drastisch verminderd en bijna alle kennis van de mensheid toegankelijk gemaakt voor iedereen met een telefoon van \$20 en een internetverbinding. Intellectuele eigendomswetten verhogen deze kosten weer, waardoor eeuwen van technologische vooruitgang in het verminderen van de kosten van het overdragen van kennis wordt teruggedraaid en talloze miljoenen genieën en producenten worden beroofd van kennis die ze zouden kunnen gebruiken om productiever te worden. Als de afgelopen eeuwen van vooruitgang het merendeel van de mensen op aarde toegang hebben gegeven tot de schouders van een zeer groot aantal reuzen, dan zijn de intellectuele eigendomswetten een belasting voor het staan op deze schouders. Het is moeilijk voor te stellen hoeveel creatiever en productiever de mensheid zou zijn als alle boeken ter wereld vrij en online beschikbaar waren.
