\hypertarget{geweld}{%
\chapter{Geweld}\label{geweld}}

\vspace{-1em}
\lettrine{A}lle menselijke economische handelingen die dit boek tot nu toe heeft behandeld, zijn op vrijwillige basis. Deel II besprak vrijwillige economische handelingen die individuen uit eigen initiatief ondernemen om de kwaliteit en kwantiteit van hun tijd te vergroten. Deel III heeft het marktsysteem uitgelegd dat voortkomt uit de sociale interacties die individuen vrijwillig aangaan om eveneens de kwaliteit en kwantiteit van hun tijd te verbeteren. In elk onderdeel hebben de betrokken individuen uit vrije wil gehandeld, zowel individueel als in samenwerking met anderen. Dat is echter niet de enige manier waarop mensen met elkaar om kunnen gaan. Ze kunnen ook de kwaliteit en de kwantiteit van hun tijd op aarde verbeteren door geweld te gebruiken of anderen met geweld te bedreigen. Ze kunnen het lichaam en het eigendom van anderen aanvallen met als doel hun eigendom af te pakken of mogelijk zelfs de ander tot slaaf te maken. Geweld, en de dreiging van geweld, leiden tot \textbf{dwang}: de oplegging van iemands wil aan een ander.

\hypertarget{non-agressieprincipe}{%
\section{Non-agressieprincipe}\label{non-agressieprincipe}}

Economie beschouwt geweld en dwang niet als irrelevant, maar bestudeert ze als vormen van menselijk handelen, waarvan de gevolgen kunnen worden onderzocht en vergeleken met vrijwillige overeenkomsten. Het fundamentele verschil tussen vrijwillige en onvrijwillige interactie is dat alle deelnemers in een vrijwillige interactie verwachten ervan te profiteren, terwijl iemand negatieve gevolgen kan verwachten in een onvrijwillige interactie (anders zouden ze niet gedwongen hoeven te worden). Hoewel vrijwillige ruilhandel niet altijd het beoogde resultaat oplevert voor alle betrokken partijen, garandeert het uitoefenen van dwang dat één partij zeker ongewenste gevolgen zal ondervinden. Mensen die geen voordeel blijken te halen uit hun gezamenlijke overeenkomst, kunnen hun verwachtingen herzien en de overeenkomst opzeggen of hun strategie aanpassen in de hoop op een betere uitkomst. Slachtoffers van dwang hebben deze mogelijkheid niet, omdat hun wensen niet worden gerespecteerd door de toepassing van geweld of de dreiging ervan. Interacties onder dwang kunnen voortduren zolang de dader de negatieve gevolgen van de agressie niet ondervindt.

De negatieve gevolgen van dwang en agressie kunnen vergeleken worden met natuurrampen of aanvallen door dieren. Net zoals voor deze situaties, hebben mensen al lang naar manieren gezocht om zichzelf tegen dwang en geweld te beschermen. Vergelijkbaar met hoe mensen werken, kapitaal\index{kapitaal} opbouwen, handelen en innoveren, leren ze zich ook te verdedigen en ontwikkelen ze steeds complexere en effectievere manieren om zichzelf en hun eigendom te beschermen tegen de agressie van anderen. Hoofdstuk 17 gaat in op verschillende verdedigingsstrategieën tegen agressie. De rest van dit hoofdstuk onderzoekt een specifieke vorm van agressie: overheidsagressie.

In ethische en economische zin is er een zeer belangrijk onderscheid tussen geweld, en het initiëren van geweld. Het initiëren van geweld schendt het eigendomsrecht op het lichaam of eigendom van het slachtoffer, wat leidt tot vijandigheid en mogelijk zelfs vergelding door het slachtoffer. Ook kan het leiden tot het uitsluiten van de dader door anderen binnen de samenleving. Dit maakt vreedzame samenwerking moeilijker en voorkomt de groei van de marktomvang en de arbeidsdeling\index{arbeidsdeling}. De mate waarin groepen mensen, klein of groot, het initiëren van geweld verwerpen, geeft aan hoe ver de marktorde is ontwikkeld en tot welke mate de samenleving kan profiteren van de arbeidsdeling\index{arbeidsdeling}. Hoe meer het initiëren van geweld door sommige leden van een groep wordt geaccepteerd, des te meer conflict er zal ontstaan en de samenwerking, die noodzakelijk is voor de marktorde, zal worden ondermijnd. Het legitimeren van agressie voor één individu of groep, maar niet voor anderen, is geen morele standaard die consequent op de samenleving toegepast kan worden.

Geweld kan echter wel ethisch acceptabel worden beschouwd wanneer het wordt ingezet ter zelfverdediging of om agressors af te weren of te bestraffen. Legitieme zelfverdediging kan daarom ook goed samen gaan met een brede marktorde. Leden van een marktorde kunnen allemaal samenwerken als ze allemaal akkoord gaan met één universele regel die op hen allen van toepassing is: de onwettigheid van het initiëren van agressie en de legitimiteit van zelfverdediging. Deze asymmetrie tussen geweld en het initiëren van geweld, en de implicaties voor de marktorde, vormen de basis voor het \textbf{non-agressieprincipe}, dat Rothbard als volgt omschrijft:

\begin{blockquotebox}
Niemand mag dreigingen uiten of geweld plegen (`agressie') tegen een ander persoon of diens eigendom. Geweld mag alleen ingezet worden tegen de persoon die zelf dit geweld pleegt; met andere woorden, enkel en alleen in verdediging tegen de agressieve daad van een ander. Kortom, geweld mag niet ingezet worden tegen een niet-agressor.\footnotemark
\end{blockquotebox}
\footautocite{179}

Het non-agressieprincipe, geformuleerd en gepopulariseerd door Rothbard en andere Oostenrijkse economen, vindt zijn historische wortels in diverse beschavingen en tijdperken doorheen de geschiedenis, zoals Edward Fuller heeft vastgelegd in zijn paper.

\begin{blockquotebox}
Een grote en diverse groep van de meest vooraanstaande denkers uit de geschiedenis hebben ideeën geuit die zeer vergelijkbaar zijn met het non-agressieprincipe. De beginselen van het principe waren bekend bij de oude Egyptenaren rond 2000 v.Chr., bij de oude Hindoes rond 1500 v.Chr., en bij de oude Hebreeërs rond 1000 v.Chr. Rond 500 v.Chr. formuleerden de oude Chinese en Griekse filosofen de onderliggende logica van het principe. Cicero kwam dicht bij het formuleren van het principe in zijn moderne vorm. Thomas van Aquino kwam uit op iets wat opvallend gelijk was aan het non-agressieprincipe na de vroege middeleeuwen, en de scholastische filosofen droegen het idee over naar de vroege moderne tijd. In de zeventiende eeuw bereikte het non-agressieprincipe de top van de Westerse filosofie.\footnotemark
\end{blockquotebox}
\footautocite{180}

Veel economische boeken, waaronder dit boek, gebruiken het verhaal van Robinson Crusoë op een onbewoond eiland als middel om de werkelijkheid van economische productie\index{productie} en de voordelen van vreedzame samenwerking te verduidelijken. Het verhaal vindt zijn oorsprong in een fictieve roman, \emph{Hayy Ibn Yaqdhan}, geschreven door de Arabische filosoof Ibn Tufayl. Dit uitgangspunt werd door hem gebruikt om te illustreren hoe een mens, zelfs wanneer geboren in totale afzondering van de mensheid, een begrip van moraliteit kan ontwikkelen.

\hypertarget{overheidsdwang}{%
\section{Overheidsdwang}\label{overheidsdwang}}

De meeste gangbare stromingen binnen de economie en politiek presenteren de overheid\index{overheid} als de oplossing voor het probleem van agressie binnen de samenleving. Gezien het feit dat geweld en agressie altijd aanwezig zullen zijn, is een instelling met een monopolie op geweld volgens hen de enige manier om op ieder vlak een beschaafde en vreedzame sociale orde te bewerkstelligen. Als alle inwoners de legitimiteit van de monopolist accepteren (vrijwillig of anderszins), worden gewelddadige handelingen, die gepleegd worden door een andere instelling, als illegaal en strafbaar beschouwd door de monopolist.

In de negentiende en twintigste eeuw draaiden de politieke en intellectuele debatten voornamelijk om de juiste rol van de staat in de samenleving, en niet zozeer om de legitimiteit of noodzaak ervan. Mises en de klassieke liberalen zagen het beschermen van haar burgers en hun eigendommen en het garanderen van hun veiligheid tegen agressie en diefstal als de juiste rol van de overheid\index{overheid}.

\begin{blockquotebox}
    De overheid\index{overheid} dient alle taken te vervullen waarvoor ze nodig is en waarvoor ze is opgericht. De overheid\index{overheid} moet het volk beschermen tegen het gewelddadige en frauduleuze gedrag van misdadigers, en ze moet het land verdedigen tegen buitenlandse vijanden. Dit zijn de functies van de overheid\index{overheid} binnen een vrij systeem, binnen het systeem van de vrijemarkteconomie.
    \par\vspace{1em}\noindent
    Onder socialisme is de overheid\index{overheid} van nature totalitair en er is niets buiten haar invloedssfeer en jurisdictie. In de markteconomie is de belangrijkste taak van de overheid\index{overheid} echter om het soepel functioneren van de markteconomie te beschermen tegen fraude of geweld van zowel binnen het land als van buitenaf.\footnotemark
\end{blockquotebox}
\footautocite{181}

Door het recht op eigendom te waarborgen, zal de overheid\index{overheid} individuen in staat stellen om te plannen voor de toekomst, hun tijdsvoorkeur\index{tijdsvoorkeur} te verlagen, kapitaal\index{kapitaal} te accumuleren, de productiviteit te verhogen en hun leven te verbeteren. Klassieke liberalen waarschuwen echter dat als een overheid\index{overheid} haar mandaat niet beperkt tot het beschermen van eigendom en de handhaving van wet en orde, dat zij dan meer kwaad dan goed zal doen. Haar interventies in de markteconomie zullen niet de beoogde doelen bereiken, voornamelijk vanwege het probleem van economische calculatie zonder duidelijk gedefinieerde eigendomsrechten (dit wordt besproken in Hoofdstuk 12). Als de overheid\index{overheid} eigenaar is van de kapitaalgoederen\index{kapitaalgoederen}, dan is er geen markt voor deze goederen en dus geen mogelijkheid om economische calculatie te verrichten met betrekking tot de alternatieve gebruiksmogelijkheden van deze middelen of om te bepalen hoe ze kunnen worden gealloceerd. Wanneer overheidsbureaucraten middels dwang beslissingen nemen over andermans eigendom, doen ze dat blindelings, zonder kennis van de belangrijkste factor die het inzetten van middelen bepaalt, namelijk de subjectieve voorkeuren van de betrokken individuen. Economische calculatie zonder eigendomsrechten resulteert in het verkeerd inzetten van middelen, verspilling en kapitaalvernietiging.

Een opmerkelijk aantal werken is gepubliceerd over de tekortkomingen van overheidsinterventie in de economie, zowel door economen uit de Oostenrijkse school als door mainstream etatistische economen.\autocite{182} Het vervolg van dit hoofdstuk zal de mislukkingen behandelen van enkele van de meest gangbare en breed ondersteunde vormen van overheidsinterventie in de individuele besluitvorming binnen een kapitalistische economie. Door gebruik te maken van de lens van menselijk handelen\index{menselijk handelen} en door inzicht te verkrijgen in de kenmerken van de opkomende marktorde, kunnen we de economische gevolgen onderzoeken van specifieke vormen van overheidsdwang.

Prijscontroles vormen misschien wel de populairste methode van overheidsinterventie in de economie en lijken een verleidelijke oplossing voor echte problemen, maar ze hebben enorme gevolgen, zowel positief voor sommigen als negatief voor velen. De logica achter prijscontroles lijkt eenvoudig en aantrekkelijk: indien de prijs\index{prijs} van een goed te hoog is, kan de overheid\index{overheid} een maximumprijs instellen, waardoor het verkopen tegen een hogere prijs\index{prijs} illegaal wordt en verkopers worden gedwongen hun waren tegen een lagere prijs\index{prijs} aan te bieden. Op deze manier kunnen mensen die de hogere prijs\index{prijs} niet kunnen betalen, toch de goederen tegen een lagere prijs aanschaffen. In \emph{Forty Centuries of Wage and Price Controls: How Not to Fight Inflation}, geven Robert Schuettinger en Eamonn Butler een uitgebreid historisch overzicht van de mislukkingen van prijscontroles door vierduizend jaar en over talloze plaatsen heen. Een alarmerend aantal regeringen heeft door de geschiedenis heen geprobeerd om met deze methode de prijzen van allerlei goederen, van voedsel tot huur, te reguleren. Er is geen enkel bewijs dat prijscontroles succesvol zijn geweest in het verlagen van prijzen; ze resulteren slechts in tekorten, zwarte markten\index{markten} en het ontstaan van uiterst inefficiënte manieren om schaarse voorraden te rantsoeneren.

Als prijscontroles dan toch enig effect hebben, is het dat ze mensen verhinderen te handelen tegen een prijs\index{prijs} waarvoor ze bereid waren te handelen. Wanneer handel wordt verboden tegen de marktprijs, zal de producent logischerwijs minder goed in staat zijn om het product op een winstgevende manier te produceren. Zonder de inkomsten van de originele hogere prijs\index{prijs} zal de producent niet genoeg van de benodigde middelen kunnen kopen om het product te produceren. Overheidsdwang kan de producent wel verbieden om boven de maximumprijs te verkopen, maar het kan hem niet dwingen te verkopen tegen die maximumprijs omdat hij er simpelweg voor kan kiezen om te stoppen met produceren. Het effect van deze interventie zal onvermijdelijk de levering van het product op de markt verminderen.

Een ander waarschijnlijk gevolg is de opkomst van een zwarte markt, waar goederen illegaal worden verkocht tegen hogere prijzen. De zwarte markt omzeilt de schade die overheidsinterventie veroorzaakt en geeft de mensen de mogelijkheid om de producten die ze nodig hebben alsnog te kopen. Maar het leidt wel tot verspilling voor de betrokken partijen. In plaats van schaarse middelen te wijden aan het produceren van het gewenste product, moeten producenten extra kosten maken die horen bij de illegale verkoop en distributie van hun product. Bovendien riskeren ze vervolging, inbeslagname en gevangenisstraffen. Hierdoor zullen speciale organisaties ontstaan die de verkoop van het product regelen en een groot deel van de winst ontvangen, in plaats van dat het wordt besteed aan investeringen in kapitaalgoederen\index{kapitaalgoederen} voor de productie\index{productie} van meer van het schaarse product. Het invoeren van prijscontroles verandert niet op magische wijze de economische waardering van goederen en hun productiekosten. Het maakt het alleen illegaal om de goederen te verhandelen tegen de prijzen die de producenten vragen. Dat betekent in feite een subsidie voor de criminele sector. Voor mensen die gewend zijn om buiten de wet te opereren, is dit lucratief. Middelen die naar de producenten hadden kunnen gaan om meer investeringen mogelijk te maken, gaan naar de misdaad\index{misdaad}.

Zodra tekorten na de invoering van prijscontroles onvermijdelijk optreden, zal de vraag naar het goed het aanbod beginnen te overtreffen. Dit zal nieuwe mechanismen om het aanbod onder consumenten te verdelen noodzakelijk maken. Een typisch voorbeeld is in de wachtrij staan. Hierbij besteden consumenten hun kostbare tijd aan het wachten in de rij totdat het product beschikbaar komt. Omdat tijd schaars is, zorgt in de rij staan er alleen maar voor dat de kosten van het goed worden overgebracht van geld naar tijd. Consumenten betalen nu voor het product met tijd die ze hebben verspild, iets wat niet door de producent kan worden gebruikt om meer van het goed te maken.

Voor sommige andere economische goederen zal gewelddadige overheidsinterventie proberen een hogere prijs\index{prijs} op te leggen op de markt dan de prijs\index{prijs} die voortkomt uit vrijwillige handel. Dit wordt voornamelijk toegepast bij lonen, waar overheden sinds lang hogere lonen voor werknemers proberen op te leggen. Het fiasco van het minimumloon wordt in detail uitgelegd in Hoofdstukken 1 en 4.

Het probleem van prijzen heeft bijna altijd zijn wortels in inflatie\index{inflatie}, wat het resultaat is van dwingende overheidsbemoeienis in de geldmarkt. Zoals besproken in Hoofdstuk 10, verschijnt geld op de markt als het goed met de beste verkoopbaarheid\index{verkoopbaarheid} door tijd -- het goed dat het best zijn waarde in de loop van de tijd behoudt. Aangezien men kan verwachten dat de beschikbare hoeveelheid het minst snel zal stijgen van alle andere marktgoederen, neigt de waarde van geld in waarde te stijgen, aangezien relatief meer van alle andere goederen dan van geld worden geproduceerd. In zo\textquotesingle n wereld zouden de prijzen van alle goederen de neiging hebben te dalen ten opzichte van het geld, terwijl de werkelijke lonen stijgen, zelfs als ze constant blijven of dalen in nominale termen. Als er echter dwingende interventie in de geldmarkt is, dan daalt de waarde van geld in de loop van de tijd, waardoor verkopers hun prijzen verhogen en de lonen van werknemers worden gedevalueerd. Het verhaal van inflatie\index{inflatie} is een al lang bestaand fenomeen, al werden er verschillende mechanismen toegepast: van Romeinse keizers die het goud\index{goud}- of zilvergehalte van hun munten\index{munten} verminderden en ze vervingen door koper en andere basismetalen, tot moderne overheden die grote hoeveelheden papiergeld drukken om rentetarieven te manipuleren, zodat er meer krediet\index{krediet} kan worden gecreëerd dan dat er spaarmiddelen\index{spaarmiddelen} in de samenleving zijn.\autocite{183} Inflatie ontneemt de welvaart van spaarders, devalueert hun geld en verhoogt de prijzen van de goederen die ze kopen, terwijl de overheid\index{overheid} met weinig beperkingen kan uitgeven.

De gangbare economische stromingen zullen mogelijk de problemen van monetaire inflatie\index{inflatie} voor geldbezitters en de economie in het algemeen erkennen, maar ze zien overheidsuitgaven als een goede zaak die deze problemen kan verlichten en helpen om maatschappelijke doelen te bereiken. Ze houden echter geen rekening met de werkelijke kosten. Begrip van opportuniteitskosten, de subjectieve aard van economie en de problemen van economische calculatie leiden tot de tegenovergestelde conclusie. Elke uitgave die door de overheid\index{overheid} wordt gedaan, moet worden gefinancierd met geld dat wordt ontnomen van productieve leden van de markteconomie, ten koste van hun eigen uitgaven. Mensen richten hun uitgaven om het best te voldoen aan hun behoeften. Het dwingend belasten van het inkomen van deze mensen om het geld vervolgens aan andere doeleinden te besteden, kan hun welzijn niet verhogen, omdat ze ervoor hadden gekozen dat geld zelf aan iets anders uit te geven. Geen enkele overheidsuitgave kan als vrijwillig worden beschouwd zolang inflatie\index{inflatie} en belasting niet vrijwillig zijn. Daarom kunnen overheidsuitgaven het best worden beschouwd als consumptieve uitgaven voor de mensen in overheidsinstellingen; het zijn geen investeringen.

Een andere veelvoorkomende vorm van overheidsinterventie is het verstrekken van subsidies aan individuen of het aanschaffen van specifieke goederen voor hen. Een simplistische kijk hierop veronderstelt dat overheidsuitgaven kosteloos zijn en dat overheden subsidies kunnen gebruiken om het welzijn van burgers te verbeteren. Echter, economische analyse laat zien dat dit niet het geval is. Overheidssubsidies verstoren de markt en sturen de beslissingen van mensen in een andere richting dan waar hun eigen beweegredenen en economische berekening hen normaal gesproken zouden leiden. Subsidies resulteren vaak in overproductie en overconsumptie van goederen, wat niet zou gebeuren als mensen vrijelijk zouden kunnen kiezen. Wanneer subsidies worden verstrekt op basis van iemands economische situatie, ontstaat er een sterkere prikkel voor mensen om die situatie te behouden om in aanmerking te komen voor de subsidies. Sociale bijstand kan bijvoorbeeld mensen met een laag inkomen aanmoedigen om op dat lage niveau te blijven. Subsidies voor werklozen kunnen leiden tot een grotere neiging tot werkloosheid. Bovendien, omdat subsidies gefinancierd worden ten koste van werkenden, kunnen ze de motivatie om te werken ondermijnen.

Overheidsvoorzieningen worden vaak gezien als een oplossing voor het probleem van ontoereikendheid of onbetaalbaarheid van goederen of diensten die ze biedt. Terwijl private bedrijven zich richten op winst, wordt door sommigen betoogd dat de overheid beter werk kan leveren door zich te concentreren op inclusie. Deze argumentatie wordt gebruikt ter ondersteuning van de overheidsvoorziening van goederen en diensten zoals onderwijs, water en gezondheidszorg. Echter, dit negeert de basisprincipes van kapitalistische economische productie. Winsten zijn niet alleen een middel voor hebzuchtige individuen om rijk te worden; ze vormen de coördinatie van de gehele structuur van marktproductie en stellen producenten in staat om de kosten en baten van verschillende opties te berekenen, terwijl ze streven naar het dienen van anderen en het behalen van optimale winsten voor henzelf. Het wegnemen van de winstmotivatie uit economische productie leidt niet tot onbaatzuchtige, overvloedige en betaalbare productie; het resulteert juist in een gebrek aan economische berekening, wat leidt tot aanzienlijke verspilling. Producten kunnen ongewenst zijn, wat leidt tot een verspilling van middelen, of als ze wel gewild zijn, kan het ontbreken van een vrije marktprijs leiden tot overmatige consumptie en problemen met beschikbaarheid voor gebruikers. \textquotesingle Gratis\textquotesingle{} door de overheid\index{overheid} aangeboden wegen eindigen bijvoorbeeld vaak vol met verkeer, wat grote vertragingen veroorzaakt voor reizigers die waarschijnlijk duurder zijn dan wat ze zouden betalen voor de aanleg van privéwegen. Door de overheid\index{overheid} aangeboden gezondheidszorg, in plaatsen als Canada, is berucht om het feit dat patiënten zeer lang moeten wachten voordat ze door een dokter kunnen worden ontvangen. Het is opmerkelijk dat in Canada, waar ze een vrije markt hebben in gezondheidszorg voor dieren, een ziek Canadees huisdier sneller een dokter ziet dan zijn eigenaar zelf.\autocite{184}

Overheidsuitgaven zijn schadelijk voor de economie, niet alleen omdat ze de berekening van winst en verlies van individuen verstoren, maar ook omdat ze gefinancierd moeten worden door belastingen, zowel direct als indirect via inflatie. Het belasten van producenten om overheidsuitgaven te financieren bestraft economische productie\index{productie}, wat de motivatie om eraan deel te nemen vermindert. Bovendien leidt de waardedaling van spaartegoeden\index{spaartegoeden} tot een afname van de stimulans om te sparen, terwijl belastingen op kapitaalwinsten de motivatie om te investeren verminderen. Door een persoon minder in staat te stellen om voor zijn of haar toekomst te zorgen, werken overheden de drijvende kracht achter de menselijke beschaving\index{beschaving} tegen, namelijk het verlagen van de tijdsvoorkeur\index{tijdsvoorkeur}.

Of het nu gaat om prijscontroles, landbouwsubsidies of belastingen, elke vorm van overheidsinterventie leidt tot gedwongen veranderingen in menselijk gedrag die onder vrije omstandigheden anders zouden zijn. Als mensen vrij gelaten zouden worden, zouden zij hun tijd en rijkdom besteden aan doelen die zij het meest waardevol achten, of dat nu productief is voor henzelf of voor anderen. Aangezien economische waarde zelf subjectief is, impliceert het wijzigen van iemands handelingen een afwijking van de door hem gekozen koers, die subjectief gezien minder zijn voorkeur geniet.

Gelukkig hebben de meeste mainstream economen, met name sinds het einde van de Sovjet-Unie, enig begrip getoond van de problemen met overheidsinterventie in het economische systeem en de verstorende gevolgen ervan. Zelfs het Samuelsoniaanse leerboek bevat nu een discussie over de problemen van overheidsinterventie in markten. Echter blijft de rechtvaardiging voor interventie onverminderd voortbestaan -- nu wordt het gepresenteerd alsof overheden 'marktfalen' oplossen, een term die mainstream economen gebruiken om een resultaat van vrije menselijke interactie aan te duiden waar ze het niet mee eens zijn.

\hypertarget{rechtvaardigingen-voor-overheidsgeweld}{%
\section{Rechtvaardigingen voor overheidsgeweld}\label{rechtvaardigingen-voor-overheidsgeweld}}

De moderne rechtvaardigingen voor overheidsingrijpen worden gewoonlijk gepresenteerd in de vorm van marktfalen: als mensen vrij gelaten worden om keuzes te maken, zullen ze inferieure en suboptimale resultaten opleveren. De initiële fout van deze benadering is dat de markt wordt gepresenteerd als een entiteit die niet in staat is om het gewenste resultaat te leveren. In werkelijkheid is de markt een overkoepelende term die mensen gebruiken om te verwijzen naar de handelingen van individuen die hun vrije wil uitoefenen om de tevredenheid in hun eigen leven te maximaliseren. De term `marktfalen' stelt een alwetende centrale planner voor die in staat is om te beslissen wat het optimale resultaat zou zijn van de vrije interacties van individuen, en vervolgens de acties van deze vrije individuen veroordeelt als inferieur en noodzakelijkerwijs te veranderen. Hoewel dit als voordelig voor het maatschappelijk welzijn wordt gepresenteerd, houdt deze benadering methodologisch gezien simpelweg in dat de wil van een centrale planner voorrang krijgt op de wensen van alle vrij handelende individuen.

De rijkelijk beloonde maar onleesbare artikelen die mainstream economen de afgelopen decennia aan de lopende band hebben gepubliceerd, bevatten een overdaad aan overheidspropaganda die zich voordoet als economische analyse. Hun benadering volgt een voorspelbaar script: een econoom verricht een grote hoeveelheid theoretisch, wiskundig of experimenteel werk om vervolgens te concluderen dat individuen die vrij handelen iets produceren dat suboptimaal is voor de samenleving als geheel, wat zij dan `marktfalen' noemen. Ze negeren gemakshalve de ethische vraag wat een academicus, die afhankelijk is van overheidssubsidies om te overleven, het recht geeft om te oordelen over de handelingen van anderen, en welk doel zij eigenlijk zouden moeten nastreven. De collectivistische benadering van economie gaat ervan uit dat waarde objectief is en dat een onpartijdige centrale planner precies kan weten hoeveel dat is. Het ontneemt individuen ook het recht om hun eigen beslissingen te nemen met betrekking tot hun eigendom, kapitaal en consumptie. Door economie voor te stellen als een objectieve wiskundige functie, terwijl economische waarde geen meeteenheid heeft, kunnen door de overheid gefinancierde economen alle benodigde cijfers tevoorschijn toveren om elke vorm van agressie tegen privaat eigendom te rechtvaardigen.

De oorsprong van deze analyse van het marktfalen komt voort uit het standaardmodel van de neoklassieke economie, wat probeerde om het marktproces wiskundig te modelleren. In plaats van de Oostenrijkse methode van individuele handelingen te volgen als basis voor het begrijpen van economie, probeerden moderne economen, in een gewaagde vertoning van cultwetenschap, de natuurkunde te kopiëren. Sinds de overname van de academische wereld door de overheid\index{overheid} in de jaren \textquotesingle30, probeert het merendeel van de economie voornamelijk natuurkundige concepten toe te passen op de ideeën van John Maynard Keynes\index{John Maynard Keynes} om zo rechtvaardigingen voor overheids- en centrale bankbeleid te verkrijgen. Wiskundige economen probeerden een wiskundig model uit de natuurkunde op de economische realiteit te leggen, maar wanneer ze werden geconfronteerd met een van de talloze onoverkomelijke obstakels tot het wiskundig modelleren van menselijke handelingen, werd de economische realiteit vereenvoudigd zodat het toepassen van de wiskundige modellen mogelijk werd. Enkele van de meest opmerkelijke aannames die voor de economie zijn bedacht, zijn 1) dat alle actoren in een marktsysteem volledige kennis moeten hebben; 2) dat ze op een rationele manier eigenbelang nastreven; en 3) dat er een toestand van perfecte concurrentie bestaat, met een oneindig aantal kopers en verkopers voor elke markt. Deze aannames houden uiteraard geen stand in de echte wereld, maar mainstream economen zien de onnauwkeurigheid van de aannames als bewijs van marktfalen in plaats van simpelweg te beseffen dat dergelijke wiskundige modellen nutteloos zijn!

Nadat het `marktfalen' is vastgesteld, stellen economen, zonder enig bewijs of analyse, dat overheidsinterventie deze geobserveerde inefficiëntie van de markt kan corrigeren. Ze publiceren deze onzin vervolgens in hoog aangeschreven tijdschriften, krijgen banen als docent aan universiteiten en ontvangen onderscheidingen en prijzen voor het bedenken van de rechtvaardigingen voor overheidsdwang en --- agressie tegen privaat eigendom. In de academische fiatwereld zijn er geen kosten verbonden aan het maken van fouten en zijn er zelfs beloningen voor het maken van fouten die de macht van de overheid vergroten.\autocite{185} De hele fiateconomie kan worden gezien als een uitgebreide oplichtingspraktijk, waarbij stromannen worden geplaatst en neergehaald en waarbij de ruimte die ze innamen wordt overgenomen alsof die van niemand was.

\subsection{Informatieasymmetrie}

Tot de meer modieuze argumenten voor economische interventie van de afgelopen decennia behoort de drogreden van `informatieasymmetrie'. Volgens de Keynesiaanse school vol onleesbaar onderzoek, beschikken individuen die deelnemen aan een transactie niet over volledige kennis van alles wat met de transactie te maken heeft, wat vaak leidt tot slechte resultaten. Dit is een volledig triviaal gegeven: het is onmogelijk voor een persoon om alles te weten wat de ander weet. Deze gedachte wordt echter gepresenteerd als bewijs om dwingende overheidsinterventie bij het maken van transacties mogelijk te maken. Toch vinden er elke dag wereldwijd miljarden markttransacties plaats, en de overgrote meerderheid daarvan is tot tevredenheid van beide partijen. Men hoeft niet alles te weten om te begrijpen wat een gunstige ruil\index{ruil} inhoudt -- men hoeft alleen maar te weten wat hun eigen voorkeur is voor de verhandelde goederen. Natuurlijk wordt in deze argumentatie ook geen rekening gehouden met het vraagstuk van hoe de dwingende regelgevende autoriteit toegang kan krijgen tot informatie die niet voor beide partijen beschikbaar is, en hoe zij die informatie kan gebruiken om met het dreigement van geweld een betere oplossing voor beide partijen af te dwingen. Hoe kan een regulerende instantie, die verantwoordelijk is voor alle transacties in een samenleving, alle benodigde informatie voor elke transactie hebben als de handelende partijen zelf niet over voldoende informatie beschikken? En wiens belang zal deze centrale planner vooropstellen?

Als informatieasymmetrie echt een probleem vormt op de markt, kan het het beste met vrijwillige inspanningen opgelost worden. Het voorbeeld dat fiateconomen graag over informatieasymmetrie aanhalen, is de markt voor gebruikte auto\textquotesingle s -- en toch is er, om dit probleem op te lossen, een grote industrie ontwikkeld die informatie levert over gebruikte auto's. Autokopers geven de voorkeur aan auto\textquotesingle s met een document met daarin de voertuighistorie. Eigenaren van auto's kiezen op hun beurt vrijwillig om zich voor deze diensten aan te melden om de waarde van hun auto te verhogen voor potentiële kopers. Op deze manier lost de markt het zogenaamde marktfalen op een volledig vrijwillige manier op. Er is een veelvoud aan diensten die productinformatie leveren ontstaan in alle soortgelijke industrieën zodat consumenten toegang kunnen krijgen tot informatie over producten: filmrecensies, restaurantrecensies, elektronicarecensies, enzovoort. Als deze industrieën, onder het voorwendsel van onvolmaakte informatie, door overheidsinterventie en regelgeving werden belemmerd om te groeien, hoe zou dat dan de consumenten en producenten ten goede zijn gekomen?

\subsection{Irrationaliteit}

Binnen de nieuwe opkomende pseudowetenschap van de gedragseconomie is `irrationaliteit' een ander populair cluster van drogredenen dat wordt gebruikt om overheidsdwang te rechtvaardigen. Gedragseconomen stellen willekeurige en irrelevante criteria voor wat rationeel gedrag zou moeten zijn en testen vervolgens hun universitaire studenten -- als menselijke proefkonijnen en plaatsvervangers voor de hele mensheid -- om te zien of deze criteria door mensen wordt voldaan. Nadat de studenten er niet in zijn geslaagd een resultaat te leveren dat voldoet aan zijn verzonnen criteria van rationaliteit, keuren ze de mensheid vol arrogantie af als irrationeel. Ten slotte concluderen ze dat alleen de dwingende interventie van de overheid\index{overheid} dit gedrag kan corrigeren.

Economische rationaliteit kan echter niet worden bestudeerd in de context van een laboratoriumexperiment omdat het inherent subjectief en marginaal is. Het betreft de besluiten van individuen op het moment en de plaats waar deze keuzes gemaakt moeten worden, terwijl in een laboratoriumomgeving alle beslissingen betrekking hebben op het lab, niet op de echte wereld. De wereld is immers enorm complex en heeft talloze factoren die niet in een lab kunnen worden nagebootst. Er is geen enkele reden om de volledig verzonnen experimenten van gedragseconomen te accepteren als een nauwkeurige weergave van de echte wereld, en de prikkels van een proefpersoon als equivalent van de stimuli die worden ervaren in de echte wereld. Zelfs als men ze zou accepteren, blijft een grotere vraag: hoe kunnen mensen irrationeel zijn, maar gedragseconomen rationeel? Als menselijke vooringenomenheden invloed hebben op de rationaliteit, waarom zouden gedragseconomen hier dan van zijn vrijgesteld? Nog belangrijker, waarom zouden de regelgevers die ingrijpen in deze markten\index{markten} immuun zijn voor deze irrationaliteit? Hoe veel destructiever zou het zijn als deze irrationaliteit wordt opgelegd op een dwangmatige macroschaal in plaats van beperkt te blijven tot degenen die er welwillend en vrijwillig voor kiezen om met hen in zee te gaan?

\subsection{Imperfecte competitie}

Omdat het neoklassieke economische model perfecte concurrentie veronderstelt, is een andere manier waarop markten\index{markten} falen imperfecte concurrentie: het falen van markten\index{markten} doordat er geen \textit{oneindig} aantal kopers en verkopers actief is. Uiteraard is dat een onmogelijk te halen norm. Zolang het aantal kopers en verkopers niet oneindig is in een markt -- wat natuurlijk nooit het geval is -- kan de markt worden bestempeld als lijdend onder imperfecte concurrentie of monopolisering. Deze situatie kan volgens overheidseconomen alleen worden verholpen door de overheid\index{overheid} een monopolie op geweld te geven die alle marktdeelnemers dwingt om volgens haar orders op een monopolievrije manier te opereren. Markten neigen echter niet naar het vormen van monopolies, behalve door het gebruik van dwingend geweld. Simpel gezegd, individuele producenten die buitensporige prijzen vragen, kunnen niet voorkomen dat concurrenten hen onderbieden -- tenzij ze geweld gebruiken. In decennia van onderzoek naar deze vraag, ben ik nog nooit een enkel voorbeeld tegengekomen van een monopolist wiens monopoliestatus op de markt vreedzaam in plaats van door dwingend ingrijpen is veiliggesteld. Het zijn altijd overheidsregels en -voorschriften die monopolies creëren, aangezien zij de enige barrière zijn die een vreedzaam particulier bedrijf\index{bedrijf} kunnen stoppen. Het ironische hieraan is dat overheidsmandaten specifieke industrieën in monopolies veranderen, wat dan het idee normaliseert dat deze industrie onvermijdelijk alleen als een monopolie kan functioneren, waardoor het een `natuurlijk monopolie' wordt. Er is alleen niets natuurlijks aan monopolies, en de overheidsregulering ervan is een probleem dat zich voordoet als een oplossing, zoals Thomas Di Lorenzo uitlegt in een artikel dat grondig de vorming van monopolies als rechtvaardiging voor overheidsdwang weerlegt:\autocite{186}

\begin{blockquotebox}
Het is een mythe dat de theorie van natuurlijke monopolies eerst werd ontwikkeld door economen, en vervolgens door wetgevers werd gebruikt om monopolies bij wettelijk recht te `rechtvaardigen'. De waarheid is dat de monopolies tientallen jaren voordat de theorie werd geformaliseerd al waren gecreëerd door op interventie gerichte economen. Ze gebruikten de theorie vervolgens als een ex post rechtvaardiging voor overheidsinterventie. Toen de eerste overheidsgesteunde monopolies werden opgezet, begreep de grote meerderheid van de economen dat grootschalige, kapitaalintensieve productie\index{productie} niet leidde tot een monopolie, maar dat een absoluut wenselijk aspect van het concurrentieproces was.
\end{blockquotebox}

Bepaalde voorbeelden van monopolies die vaak door economen worden aangehaald, verwijzen naar producenten die erin slaagden hun marktaandeel te vergroten door een aanzienlijk verbeterd product aan te bieden tegen een lagere prijs\index{prijs} dan de concurrenten. In dit geval beschermen monopoliewetten niet de consument tegen een monopolistische producent; ze beschermen de inefficiënte producenten tegen efficiëntere. Ook laten ze de inefficiënte producenten winstgevend blijven zonder de meest efficiënte productiemechanismen toe te passen die wel door de marktleider zijn omarmd.

\subsection{Externaliteiten en publieke goederen}

Een aantal van de meest voorkomende argumenten voor overheidsdwang zijn de misvattingen over \textquotesingle externe invloeden\textquotesingle{} en \textquotesingle publieke goederen\textquotesingle, die worden gepresenteerd als unieke goederen die door hun aard alleen adequaat kunnen worden geleverd met behulp van overheidsdwang. De meeste reguliere economische leerboeken zullen toegeven dat het vrije marktkapitalisme het beste maatschappelijke organisatieprincipe is voor de productie\index{productie} en verdeling van private goederen. Deze leerboeken presenteren \textquotesingle publieke goederen\textquotesingle{} echter als een speciaal soort goed waarvoor de markten\index{markten} ontoereikend zijn.

Externaliteiten zijn positieve of negatieve economische gevolgen die iemand ondervindt als gevolg van de consumptie\index{consumptie}- of productiebeslissingen van een ander persoon. Negatieve externaliteiten kunnen de vorm aannemen van vervuiling of economische verliezen. Positieve externaliteiten kunnen de vorm aannemen van economisch\index{economisch} voordeel uit activiteiten die anderen ondernemen, zoals een hotel of restaurant dat geniet van bovengemiddelde inkomsten dankzij een sportevenement dat in de buurt plaatsvindt. Een ander voorbeeld kan zijn dat een vastgoedbedrijf de prijzen van zijn eigendommen ziet stijgen door de opening van een stadspark op nabijgelegen grond. Hierdoor is het omliggende vastgoed aantrekkelijker geworden voor potentiële kopers. Een beroep op het bestaan van externaliteiten is onvoldoende rechtvaardiging voor het toepassen van overheidsdwang. Externaliteiten zijn ofwel schendingen van eigendomsrechten, in welk geval ze kunnen worden opgelost door de rechtspraak, of het zijn situaties die kunnen worden beschouwd als een onvermijdelijk gevolg van het leven in een samenleving. In het tweede geval kunnen externaliteiten geen rechtvaardiging zijn voor het gebruik van geweld of gerechtelijke stappen. Een voorbeeld van het eerste is vervuiling. Als een fabriek begint met het lozen van afval op aangrenzende eigendommen, schendt het simpelweg het eigendom van zijn buren. De vervuilende daad is het begin van agressie, en de grondeigenaar die het slachtoffer is, kan juridische stappen ondernemen tegen de fabriek. Zo heeft vrijwel elke economische activiteit in de markteconomie invloed op anderen. Als je het laatste stuk taart bij je lokale bakker koopt, betekent dit dat anderen het niet kunnen kopen. Als je er netjes uitziet en je fatsoenlijk gedraagt -- in tegenstelling tot stinken en er onverzorgd uitzien -- heeft dit een positieve impact op mensen die met je omgaan. Iedereen kan belang hebben bij de beslissingen van een ander en daardoor een positief of negatief nut ervaren, maar dat rechtvaardigt nooit het starten van agressie.\autocite{187} Hoofdstuk 5 legde de basis voor privaat eigendom uit en hoe het de enige consistente morele standaard is waarmee een samenleving vredig en productief kan functioneren. Deelnemen aan de markteconomie betekent dat je economische interactie zal hebben met een zeer groot aantal mensen en elke dag talloze externaliteiten door hun beslissingen zal ondervinden. Dit economische en sociale systeem kan alleen vreedzaam functioneren als alle leden hun soevereiniteit over hun eigendom uitoefenen en de soevereiniteit van anderen over hun eigendom accepteren. Zolang iemand bij het gebruik van zijn eigendom het eigendom van een ander niet schaadt, kan er absoluut geen legitieme basis bestaan voor het initiëren van geweld tegen die persoon, ook al is de ander het niet eens met hoe die persoon gebruik maakt van zijn eigendom. Dit is een morele standaard die universeel kan worden toegepast. Een morele standaard waarbij mensen echter controle kunnen uitoefenen over het eigendom van niet-aggressors, zal onvermijdelijk leiden tot eindeloze conflicten en het uiteenvallen van de basis waarop de beschaafde samenleving rust: het privaat eigendom.

Publieke goederen worden gedefinieerd als goederen die niet-exclusief en niet-ri\-va\-li\-se\-rend zijn --- termen die nauw verweven zijn met het concept van externaliteit. Niet-exclusief refereert aan het feit dat iemand ongehinderd van een goed kan profiteren terwijl iemand anders ervoor betaalt. Met andere woorden, de voordelen van het goed komen zowel de persoon die ervoor betaalt als een persoon die er niet voor betaalt ten goede. Dit zal mensen aanmoedigen om niet voor het goed te betalen, wat zal resulteren in een suboptimale productie\index{productie}, oftewel onderproductie. Als een overheid\index{overheid} iedereen dwingt om voor het goed te betalen, kan het aan iedereen in de noodzakelijke hoeveelheid worden geleverd. De fatale, onuitgesproken aanname hierbij is echter dat economen of centrale planners de optimale productie\index{productie} van een goed voor de hele samenleving kunnen bepalen. Ze nemen beslissingen namens iedereen, volledig op de hoogte van de betrokken afwegingen en de gemaakte opportuniteitskosten voor ieder persoon. Economische calculatie kan echter alleen worden uitgevoerd wanneer kapitaalgoederen\index{kapitaalgoederen} privaat worden verhandeld, zodat de prijzen kunnen dienen als betrouwbare signalen voor de markt. Publieke goederen worden aan de marge geleverd, en ze vereisen de inzet van arbeid en kapitaalgoederen\index{kapitaalgoederen} op basis van economische calculatie. Abstracte overwegingen over hun waarde zijn irrelevant als ze niet in een prijs\index{prijs} kunnen worden vertaald door de vrije handelingen van individuen als werknemers en kapitalisten.

Een regulier economieboek presenteert het leger en politie, stadsparken, wegen, vuurtorens, brandweer en politie als voorbeelden van niet-exclusieve goederen. Als je naar een stad zou verhuizen waarin je geen cent betaalt voor de productie\index{productie} van deze goederen, zou je er nog steeds van profiteren. Het leger zou jou en iedereen in de stad veilig houden en je zou kunnen genieten van de parken en wegen zonder een cent bij te dragen. Je goederen zouden getransporteerd worden op boten die baat hebben bij vuurtorens waar je niet aan hebt bijgedragen. De brandweer zou een brand in je huis blussen, terwijl de politie criminelen in je buurt zou arresteren, waardoor je een veiliger bestaan hebt. Omdat de samenleving niet kan voorkomen dat je voordeel geniet van deze goederen, ontstaat het probleem van de profiteur -- iedereen wil graag profiteren van deze goederen zonder bij te dragen aan hun voorziening. Reguliere economen concluderen daarom dat deze goederen zonder overheidsdwang ondermaats zouden worden geleverd.

Toch staat de geschiedenis vol met voorbeelden waarbij dit soort goederen zonder problemen toch vrijwillig werden geleverd, evenals voorbeelden waarbij verkwisting en tekorten het resultaat waren van gewelddadige overheidsinterventies. Een vrijwillige levering hoeft niet altijd door organisaties met winstbejag te worden verzorgd. Talloze vormen van liefdadigheid of vrijwilligerswerk kunnen cruciale goederen leveren zonder te worden genoodzaakt om gewelddadige dwang toe te passen. Talloze openbare parken zijn door de eigenaren aan hun geboortestad gedoneerd. Private parken zijn ook talrijk in veel gebieden waar private organisaties de biodiversiteit en mooie gebieden beheren en ze beschermen met de inkomsten die met entreeprijzen en diverse ervaringen en producten worden opgehaald. Deze private natuurgebieden beslaan ongeveer 200.000 km2 in Zuid-Afrika -- ongeveer een zesde deel van de totale oppervlakte van het land.\autocite{188} Er wordt daar betaald om van de natuurgebieden te genieten. Historische bronnen staan vol met vuurtorens die gebouwd werden door private instanties die havens exploiteerden en gefinancierd werden door de heffingen die aan aanmerende boten werden opgelegd.\autocite{189} Het feit dat enkele boten die voorbij de haven voeren, konden profiteren van het licht zonder bij te dragen aan de constructie, was geen belemmering voor de bouw ervan, omdat het voor havengebruikers alsnog economisch\index{economisch} nuttig genoeg kon zijn om ervoor te betalen. In het uiterste geval kunnen de eigenaars van vuurtorens het licht ook uitzetten om profiteurs te dwingen een betalingsregeling te treffen.\autocite{190}

Het fundamentele probleem met het argument rondom externaliteiten is dat het de realiteit negeert van hoe marginale economische beslissingen worden genomen. Bij het besluit om een product te kopen, maakt een persoon een economische calculatie van de kosten en baten van deze marginale aankoop. Als de voordelen de kosten overtreffen, koopt hij het product. Het is voor hem niet relevant of anderen hiervan profiteren of niet. Zolang het product niet leidt tot het schaden van het eigendom van anderen, heeft het individu geen reden om de voor- of nadelen van anderen te berekenen. Hij zal zichzelf niet benadelen, alleen maar om ervoor te zorgen dat anderen niet profiteren. Als de vuurtoren nuttig is voor de haven, zullen de boten die er aanmeren waarschijnlijk bereid zijn om meer te betalen voor het gebruik ervan dan de constructie kost, en dus zal de haveneigenaar de vuurtoren bouwen.

Openbare wegen lijden wereldwijd aan files en verval. De overheden die ze bouwen, kunnen grond in beslag nemen en aan de eigenaren ervan de prijs\index{prijs} betalen die ze willen. Centrale planners worden dus niet geconfronteerd met een nauwkeurige kostenberekening voor het belangrijkste middel dat ze beheren, wat betekent dat ze er niet de volledige marktprijs voor hoeven te betalen. Het resultaat is een overproductie van wegen wat leidt tot het verbruik van grote hoeveelheden grond voor wegen, de vermindering van de bruikbare ruimte van een stad en het dwingt mensen om steeds meer te moeten rijden naarmate de stad zich uitbreidt. In tegenstelling tot de etatistische clichés, zou een wereld waarin overheden geen wegen aanleggen, geen wereld zijn zonder wegen. Het zou simpelweg betekenen dat de wegenbouwers de volledige kosten zouden moeten betalen, en dat de opbrengst van het gebruik van de weg en van de herbestemming van de grond voor alternatieve doeleinden hoog genoeg zou moeten zijn om die kosten te rechtvaardigen. Zo'n berekening is niet mogelijk wanneer overheden grond kunnen confisqueren om wegen te bouwen of een verkoopprijs kunnen opleggen aan eigenaren van grond. Marginaal gezien zal een dergelijk beleid overheden toestaan om grond te verwerven tegen een kostprijs die lager is dan de marktprijs. Bovendien wordt de economische calculatie uitgevoerd door mensen met een gevestigd belang bij het uitvoeren van meer projecten, omdat dat meer geld in hun handen betekent. Als geen enkele instantie grond kan kopen tegen een prijs\index{prijs} die het zelf verordonneert, zal de grond worden toegewezen aan economische doeleinden en zal het niet teveel worden gebruikt voor eenzelfde doeleinde. Veel wegen worden privaat gebouwd. Door de gebruikers ervan rechtstreeks te belasten, worden de wegen veel functioneler, omdat ze de gebruikers de kosten van files besparen door een prijs\index{prijs} in rekening te brengen die het verkeer vlot laat doorstromen. Walter Block's werk over de economie van wegen is hier heel nuttig.\autocite{191} Het volgende hoofdstuk onderzoekt veiligheid en defensie, en waarom ze reguliere economische goederen zijn die geen speciale voorzieningen vereisen.

Het principe van niet-rivaliteit verwijst naar goederen waarvan het verbruik door één persoon het voordeel voor andere consumenten niet vermindert. Dit zijn goederen die aan de hele samenleving kunnen worden geleverd of aan niemand. Een vuurtoren, straatlampen en nationale defensie zijn klassieke voorbeelden. Een vuurtoren komt alle schepen die een zeehaven passeren ten goede, zelfs als een schip niet aanmeert in de haven en dus geen aanlegvergoeding betaalt aan de eigenaren. Schepen die een vuurtoren passeren, kunnen allemaal het licht zien en ervan profiteren, tegelijkertijd verminderen ze het licht niet voor elkaar. Ook profiteren alle voetgangers van straatverlichting, zonder dat gebruik ervan het licht wegneemt bij anderen. Een leger dat een land beschermt tegen buitenlandse indringers beschermt iedereen in de samenleving, en het toevoegen van een extra persoon aan deze samenleving vermindert de veiligheid en beveiliging van andere leden niet. Het leger houdt ofwel buitenlandse legers tegen voor het algemeen belang van alle burgers, ofwel doet dat niet.

Bij nader inzien blijkt dit echter ook een foutieve rationalisatie voor het initiëren van agressie. Als een goed echt niet-rivaliserend is, zou het een niet-economisch\index{economisch} goed zijn. Rivaliteit is altijd aanwezig bij economische goederen, en de oplossing voor dat probleem zijn eigendomsrechten en het beginsel van non-agressie. Straatverlichting maakt eenvoudigweg deel uit van de straat en behoort toe aan de eigenaar ervan, die hiervoor kosten in rekening brengt als onderdeel van de vergoeding voor toegang tot de straat. Zelfs in het geval van openbare wegen in stedelijke gebieden -- die van niemand zijn -- profiteren mensen die aan de straat wonen, klanten en bezoekers het meest van het licht. Als het voordeel van straatverlichting voor hen de investering waard is, kunnen zij individueel of gezamenlijk investeren via vrijwillige verenigingsvormen. Het feit dat ze misschien niet kunnen voorkomen dat voorbijgangers ook voordeel hebben, rechtvaardigt geen agressie tegen alle leden van de samenleving om de straatverlichting te bekostigen. Als de bewoners van één straat kunnen verwachten dat de rest van de samenleving hun straten financiert, zullen de inwoners van alle straten hetzelfde verwachten. In plaats van individueel te beslissen of de kosten opwegen tegen de voordelen, plaatst de collectivistische oplossing een centrale planner aan het hoofd om die beslissing voor de gehele maatschappij te nemen. Sommige mensen krijgen verlichting, waarvoor ze heel weinig betalen, en anderen worden gedwongen te betalen voor lichten die ze niet gebruiken als voor hun straat geen verlichting nodig wordt geacht. Uiteindelijk is door het bestaan van eigendomsrechten niets niet-rivaliserend. Er is een limiet aan het aantal mensen dat een weg kan gebruiken en voordeel kan hebben van de verlichting ervan, en deze rivaliteit is wat de wegbeheerder motiveert om de infrastructuur van de weg te optimaliseren. Dit komt ten goede aan hem en de gebruikers die hij op de weg wil hebben.

Het is ook onjuist om te veronderstellen dat nationale defensie niet-rivaliserend is. Veiligheid en bescherming tegen agressie zijn beide private goederen, en elk individu verhoogt de last voor de aanbieder van deze diensten. Hoe groter het gebied dat beveiligd moet worden, hoe hoger de kosten. Hoe meer mensen in het gebied wonen, hoe meer mogelijke doelwitten en hoe meer veiligheidsrisico\textquotesingle s voortkomen uit het gedrag van elke toegevoegde persoon, wiens acties de veiligheid van anderen in gevaar kunnen brengen.

In al deze voorbeelden liggen de wortels van het foutief economische redeneren in het negeren van marginale analyse. Het is verleidelijk om over nationale defensie, justitie, wegen, verlichting en dergelijke in absolute en algemene termen te spreken, maar in de economische realiteit bestaat er alleen marginaliteit, en individuen die de beslissing nemen over de inzet van kapitaal\index{kapitaal} om deze goederen marginaal te produceren. Of het nu gaat om een soldaat, politieagent, rechter, weg of lantaarnpaal, er worden alleen individuele eenheden ingezet, met economische kosten en baten. Alleen door economische calculatie met eigendomsrechten kunnen deze middelen productief en rationeel worden ingezet.

\hypertarget{rationaliteit-in-economie}{%
\section{Rationaliteit in economie}\label{rationaliteit-in-economie}}

De oorsprong van de analyse van het marktfalen komt voort uit het standaardmodel van neoklassieke economie. Zoals ik al eerder heb opgemerkt, heeft dit model geprobeerd het marktproces wiskundig te modelleren. Om dit te doen, hebben economen enkele belachelijke aannames gedaan: dat alle deelnemers in een marktsysteem volledige kennis moeten hebben en rationeel op eigen belang zijn gericht; en dat er een staat van perfecte concurrentie is, met een oneindig aantal kopers en verkopers voor elke markt. De afgelopen zeventig jaar aan economie hebben vermeende genieën, die overheidssalarissen ontvangen, hun tijd besteed aan het prikken van gaten in dit belachelijke wiskundige model, om vervolgens te beweren dat ze de mogelijkheid van goed functionerende markten\index{markten} hebben weerlegd.

Stel je een econoom voor die een wiskundig model creëert voor de vlucht van een vogel. Om het model berekenbaar te maken, maakt hij vereenvoudigende aannames, zoals dat het gewicht van de vogel gelijkmatig over zijn lichaam verdeeld is. Met deze aanname kan een soort vereenvoudigd model van vogelvlucht worden geconstrueerd dat geschikt is voor examenvragen. Economen die zich bezighouden met marktfalen zullen dan uitvoerig debatteren over de aanname en trots verkondigen dat ze hebben bewezen dat... vogels niet vliegen! Ze verwerpen niet alleen dit model van vogelvlucht als onnauwkeurig -- ze verwerpen ook het echte verschijnsel dat het onnauwkeurige model probeert weer te geven, zelfs al zien ze elke dag vliegende vogels. Net zoals het er niet toe doet dat vogels daadwerkelijk kunnen vliegen, doet het er ook niet toe dat miljarden mensen wereldwijd dagelijks deelnemen aan wederzijds voordelige markttransacties. Voor de door fiat\index{fiat} (overheid\index{overheid}) betaalde academici wordt de waarheid bepaald door degene die hun salaris tevoorschijn tovert, niet door het weerspiegelen van de werkelijkheid. Zolang een econoom een fout kan aanwijzen in de belachelijke wiskundige modellen van andere economen, zullen alle gangbare leerboeken de heilige mantra\textquotesingle s trouw herhalen: `Markten falen!' en `De overheid\index{overheid} lost dit op!'

Dit wordt duidelijk na het lezen van Vernon Smith\textquotesingle s fascinerende boek \emph{Rationality in Economics}, een experimenteel econoom die economische modellen in klaslokalen testte, maar hij was voor het grootste deel van zijn lange carrière geen Oostenrijkse econoom.\autocite{192} Maar toen hij experimenteerde met economische deelnemers, kwam Smith tot dezelfde conclusies die de Oostenrijkse economen decennia eerder hadden bereikt. Zelfs in kunstmatige laboratoriumomgevingen konden Smith\textquotesingle s proefpersonen voordelig handel drijven en prijzen ontdekken. En dat deden zij zonder te hoeven voldoen aan de aannames van het neoklassieke model en zonder een welwillende centrale planner nodig te hebben die hen de voorwaarden oplegde. Markten hoeven dus niet te voldoen aan de aannames van het neoklassieke economische model om te kunnen werken; het is eerder het neoklassieke model dat deze aannames nodig heeft om berekenbaar te zijn. De echte wereldmarkten hebben deze modellen net zo hard nodig als dat de zon de astronomie nodig heeft om op te komen.

Deze realisatie heeft Vernon Smith ertoe gebracht om voort te bouwen op het werk van Friedrich Hayek om een onderscheid te maken tussen de resultaten van menselijk ontwerp en menselijk handelen\index{menselijk handelen}, en hoe elk op zijn eigen manier rationeel kan worden begrepen.\autocite{193} `Constructieve rationaliteit' is de term die Smith gebruikte om dingen aan te duiden die bewust zijn ontworpen door menselijke intelligentie --- een voorbeeld hiervan is het ontwerp van een auto of vliegtuig. Ingenieurs hebben elk detail van hun ontwerp uitgewerkt en geproduceerd. Daarentegen gebruikte Smith de term `ecologische rationaliteit' om te verwijzen naar fenomenen die voortkomen uit menselijke handelingen en interactie -- door middel van een evolutionair proces van variatie en selectie -- zonder dat een specifieke ontwerper de contouren van het ontwerp bepaalt. Een voorbeeld hiervan zijn vliegroutes, die niet worden ontworpen door een planner van bovenaf, maar in plaats daarvan ontstaan uit een uitgebreid proces van variatie en selectie. In dit geval proberen talloze luchtvaartmaatschappijen vele verschillende routes en plannen voor aansluitende vluchten, maar uiteindelijk bepaalt de consumentenkeuze welke routes winstgevend zijn en welke niet. Luchtvaartmaatschappijen maken vervolgens gebruik van marktfeedback -- het bouwen van nieuwe vliegvelden, het lanceren van nieuwe lijnen, het optimaliseren van bepaalde verbindingen -- om het zeer geavanceerde wereldwijde web van vliegroutes te produceren dat onze planeet bedekt. Hayek introduceerde het concept van spontane orde om te verwijzen naar deze fenomenen, die de complexe output van het werk van een ontwerper lijken, maar in werkelijkheid zijn ze het product van menselijk handelen\index{menselijk handelen} en interactie onder een set van overeengekomen, abstracte regels.

Hayeks krachtige inzicht is dat veel van de orde in ons leven en de instellingen die we nodig hebben voor ons overleven, producten zijn van menselijke interactie die spontaan ontstaan -- niet het product van bewust menselijk ontwerp. Taal is misschien wel het beste voorbeeld hiervan. Hoewel sommige moderne talen, zoals het Esperanto, rationeel zijn ontworpen, hebben de meeste talen ter wereld geen ontwerper of grondlegger. Deze talen zijn in de loop van duizenden jaren ontstaan en ontwikkeld, en generaties van mensen hebben ze geleerd en kleine toevoegingen en wijzigingen aangebracht, waarvan sommige zijn gebleven terwijl andere verdwenen. Hayek, de Oostenrijkers en Smith betogen dat de kapitalistische markteconomie ook geen product is van iemands ontwerp, maar het complexe opkomende verschijnsel dat evolueert uit de handelingen van mensen die functioneren onder een reeks abstracte regels. Niemand ontwerpt markten\index{markten} of roept ze in het leven door fiat\index{fiat}; ze ontstaan in een wereld waarin individuen vrij zijn om de economische handelingen te ondernemen waarover in het tweede deel van dit boek wordt gesproken. In een sociale orde waarin mensen op rechtmatige wijze eigendom hebben verworven en eigenaar blijven van hun lichamen, zijn ze in staat om te werken, kapitaal\index{kapitaal} te accumuleren, hun gebruik van energiebronnen te verhogen om aan hun behoeften te voldoen, en de staat van de technologie die ze gebruiken te verbeteren. In een maatschappij waar mensen elkaars eigendom respecteren en het begin van agressie afwijzen, kan er met elkaar gehandeld worden, en daaruit ontstaat geld, de arbeidsdeling\index{arbeidsdeling} en het moderne kapitalistische systeem. Er is geen bewuste ontwerper die de ontwikkeling van een markteconomie stuurt; het is de spontane orde die ontstaat uit de naleving van de abstracte regels die in de moderne beschaving\index{beschaving} gelden.

Economen uit de reguliere economische stroming uit de twintigste eeuw missen dit punt volledig. In plaats hiervan stellen zij zich voor dat markten\index{markten} het product zijn van rationeel ontwerp, zoals een auto, tafel, of ten minste iets dat verbeterd kan worden met bewust centraal gestuurd ontwerp. De fatale misvatting hier is, in Hayeks taal, dat door te proberen de markteconomie te verbeteren en te herstellen met gecentraliseerde planning, de fundamentele abstracte regels die de basis vormen van de markteconomie, zullen worden ondermijnd en verstoord door dwang. In die zin biedt Hayek de Oostenrijkse visie op de taak van de econoom:

\begin{blockquotebox}
De eigenaardige taak van economie is om mensen te laten zien hoe weinig ze eigenlijk weten over wat ze denken te kunnen ontwerpen. Voor de naïeve geest die orde alleen kan opvatten als het product van bewuste ordening, kan het absurd lijken dat onder complexe omstandigheden orde, en aanpassing aan het onbekende, effectiever bereikt kan worden door beslissingen te decentraliseren, en dat een deling van autoriteit daadwerkelijk de mogelijkheid van algehele orde zal vergroten. Toch leidt die decentralisatie er in feite toe dat er met meer informatie rekening wordt gehouden.\footnotemark
\end{blockquotebox}
\footautocite{194}

Reguliere fiateconomen zijn erg behendig in het aandragen van zwaarwichtige rechtvaardigingen voor waarom markten\index{markten} falen, waarom mensen irrationeel zijn, en waarom alleen interventie kan slagen in het verbeteren van zaken. Maar bij nadere inspectie blijkt dat markten\index{markten} functioneren, ondanks de bezwaren van economen en dat het echte marktfalen optreedt wanneer dwang, onder verleidelijk altruïstische voorwendsels, wordt gebruikt om deze markten\index{markten} te proberen te repareren. Misschien zijn het niet de marktdeelnemers, maar de economen die irrationeel zijn en die weigeren de natuurlijke orde van de markt te zien, zelfs terwijl ze erop vertrouwen in hun dagelijkse leven. Maar dat is geen eerlijke beschuldiging, want de werkelijkheid is dat het levensonderhoud van de moderne econoom afhankelijk is van het aanvallen van de markteconomie en het rationaliseren van overheidsinterventies. Het produceren van onzinnig onderzoek om de agressie door de overheid\index{overheid} te rechtvaardigen is, hoe betreurenswaardig ook, de rationele koers van handelen voor een professionele econoom in een wereld waarin het academische domein gekaapt is door de staat.
